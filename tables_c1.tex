\documentclass[dv_diss_main.tex]{subfiles}





%---------------------------------------------%
%----------------Figures----------------------
%---------------------------------------------%



\newcommand{\rddplot}{
NOTE--This plot aggregate data into bins of half percentage points and estimate a third order polynomial regression between the running variable and the bins on each side of the cut-off. 
}


\newcommand{\rdd}{
NOTE--This table reports the estimates of political alignment from equation (2). The sample includes post electoral years of all municipalities with close elections during the period 1998-2003. The outcome variables are measure as a three year changes. Controls refers to state fixed effects, election-year fixed effects, and baseline political characteristics (incumbency status, previous political alignment, previous political party). Mean dep var refers to the sample average of the outcome variable for the non-aligned municipalities. 
}



\newcommand{\sharerev}{
Percentage of revenues is the sample average share of each source of revenue on total revenues for the non-aligned counterparts
}

\newcommand{\shareemp}{
Percentage of employment is the sample average share of each sector on total employment for the non-aligned counterparts
}

\newcommand{\event}{
NOTE--The figure plots the coefficients obtained from the estimation of equation (3) discussed in Section 4. The sample includes all municipalities with close elections during the period 1998-2003. The unit of observation is  the municipal-election pair, for each pair I follow the outcome measures in [-4 +4] years window. The outcome variables are measure in inverse hyperbolic sine points. The tick(thin) lines are 90\%(95\%) confidence intervals. The specification controls by municipality-election and election-year fixed effects. 
}

%inverse hyperbolic sine (IHS) transformation


\newcommand{\stars}{
Standard errors clustered at municipality level.  *** p<0.01, ** p<0.05, * p<0.1.
}

\newcommand{\census}{
}

\newcommand{\household}{
}



\begin{document}

\begin{table}[h]
    \small
    \begin{center}
        \caption{Effect of Alignment on Intergovernmental Transfers}\label{tab:1trans} 
        \resizebox{!}{!}{
\resizebox{!}{!}{
	\begin{tabular}{lcccccccc}

	
	\hline
	\hline
	
%	\multicolumn{1}{l}{} &  \multicolumn{4}{c}{Vote margin bandwidth}  \\
\\
\multicolumn{1}{l}{} &  \multicolumn{4}{c}{Earmarked Transfers Growth}  &  \multicolumn{4}{c}{Prb(Earmarked Transfers $>$ 0)}  \\
\cmidrule(l{1mm}r{1mm}){2-5} \cmidrule(l{1mm}r{1mm}){6-9} 	
											&  (1) 
											&  (2) 
											&  (3) 
											&  (4)   
											&  (5) 
											&  (6) 
											&  (7) 
											&  (8) \\


	\midrule
	
Political alignment         &        .651\sym{*}  &        .420\sym{**} &        .594\sym{**} &        .290\sym{**}  &      -.0170         &      .00239         &      .00901         &       .0224\sym{**} \\
                            &       (.38)         &       (.18)         &       (.26)         &       (.12)     &       (.01)         &       (.01)         &       (.01)         &       (.01)         \\
\\
Mean dep var           &        1.38         &        1.38         &        1.44         &        1.44        &            .97         &         .97         &         .98         &         .98         \\

R$^2$          		&         .01         &         .79         &         .01         &         .80    	 &         .01         &         .21         &         .00         &         .22         \\
Controls		    	& 					   & $\checkmark$  		&						& $\checkmark$ & 			  & $\checkmark$  		&						& $\checkmark$  \\
Bandwidth    	&        5         &        5         &        11         &        11   	&        5         &        5         &        11         &        11   \\

Obs      									&        1313         &        1313         &        2639         &        2639   	&        1313         &        1313         &        2639         &        2639   \\

%\multicolumn{1}{l}{} &  \multicolumn{4}{c}{Prb(Ramo-33 transfers $>$ 0)}  \\
%\cmidrule(l{1mm}r{1mm}){2-5}
%Political alignment  	     
%\\
%Control's mean dep var                          \\
%R$^2$          					
%\midrule



\bottomrule							

\end{tabular}%
}
}
    \end{center}
    \begin{tablenotes}
    \vspace{0.5em}
    \footnotesize	
	\textit{Note. }This table reports the estimates of political alignment from equation (2). The sample includes post electoral years of all municipalities with close elections during the period 1998-2003. The outcome variables are measure as a three year changes. Controls refers to state fixed effects, election-year fixed effects, and baseline political characteristics (incumbency status, previous political alignment, previous political party). Mean dep var refers to the sample average of the outcome variable for the non-aligned municipalities.
    \end{tablenotes}
\end{table}

\newpage

\begin{table}[h]
    \small
    \begin{center}
        \caption{{Effect of Alignment on Source of Revenue}}\label{tab:2revenues}
        \resizebox{!}{!}{

\resizebox{!}{!}{
{
\def\sym#1{\ifmmode^{#1}\else\(^{#1}\)\fi}
\begin{tabular}{lcccccc}
	\hline
	\hline
	\\
	\multicolumn{1}{l}{} &  \multicolumn{1}{c}{\shortstack{ Earmarked\\transfers}} 
	                     &  \multicolumn{1}{c}{\shortstack{Revenue\\transfers}} 
	                     &  \multicolumn{1}{c}{\shortstack{Taxes\\ \& services}} 
	                     &  \multicolumn{1}{c}{\shortstack{Public\\debt}} 
	                     &  \multicolumn{1}{c}{\shortstack{Other\\revenues}}
	                     &  \multicolumn{1}{c}{\shortstack{Total\\revenues}}  
	                     \\
	
	\cmidrule(l{1mm}r{1mm}){2-6} \cmidrule(l{1mm}r{1mm}){7-7} 
											&  (1) 
											&  (2) 
											&  (3) 
											&  (4)  
											& (5) 
											& (6)  \\
\midrule
\multicolumn{7}{l}{\bf Panel A. Bandwith 5 pp N=1313 }  \\
\\	

Political alignment 
			&        .420\sym{**} &       .0422         &       .0808         &        .287         &        .229         &        .121\sym{***}  \\
            &       (.18)         &       (.07)         &       (.10)         &       (.36)         &       (.15)         &       (.05)          \\
\\
Control mean dep var 	
		 &        1.38         &         .33         &         .36         &         .61         &         .36         &         .56         \\
\% of revenues    	
		 &         .22         &         .53         &        .062         &        .033         &         .15         &           1         \\
R$^2$          		
		&         .79         &         .24         &         .22         &         .48         &         .41         &         .68         \\

\midrule
\multicolumn{7}{l}{\bf Panel B. Bandwith 11 pp N=2639 }  \\
\\

Political alignment 
                    &        .290\sym{**} &       .0921\sym{*}  &       .0496         &       .0478         &       .0759         &        .103\sym{***}\\
                    &       (.12)         &       (.05)         &       (.07)         &       (.23)         &       (.09)         &       (.03)         \\
\\
Control mean dep var 	
				    &        1.44         &         .35         &         .38         &         .60         &         .38         &         .57         \\

\% of revenues    	
				    &         .22         &         .54         &        .059         &        .029         &         .15         &           1         \\
R$^2$          		
					&         .80         &         .31         &         .18         &         .46         &         .39         &         .67         \\

\bottomrule							




\end{tabular}
}
}}
    \end{center}
    \begin{tablenotes}
    \vspace{0.5em}
    \footnotesize
	\textit{Note. }This table reports the estimates of political alignment from equation (2). The sample includes post electoral years of all municipalities with close elections during the period 1998-2003. The outcome variables are measure as a three year changes. Controls refers to state fixed effects, election-year fixed effects, and baseline political characteristics (incumbency status, previous political alignment, previous political party). Mean dep var refers to the sample average of the outcome variable for the non-aligned municipalities.
    \end{tablenotes}
    
\end{table}     

\newpage

\begin{table}[h]
    \small
    \begin{center}
        \caption{{Effect of Alignment on Private Formal Employment and Earnings}}\label{tab:3femp}
        \resizebox{!}{!}{
\resizebox{!}{!}{
	\begin{tabular}{lcccccccc}

	
	\hline
	\hline
	
%	\multicolumn{1}{l}{} &  \multicolumn{4}{c}{Vote margin bandwidth}  \\
\\
\multicolumn{1}{l}{} &  \multicolumn{4}{c}{Private Employment}  &  \multicolumn{4}{c}{Private Wages}  \\
\cmidrule(l{1mm}r{1mm}){2-5} \cmidrule(l{1mm}r{1mm}){6-9} 	
											&  (1) 
											&  (2) 
											&  (3) 
											&  (4)   
											&  (5) 
											&  (6) 
											&  (7) 
											&  (8) \\


	\midrule
	
Political alignment         &       -.121\sym{**} &       -.116\sym{**} &       -.105\sym{***}&      -.0950\sym{**} &      .00192         &     -.00213         &    			  .00159         &     -.00243         \\
          
                          &       (.05)         &       (.05)         &       (.04)         &       (.04)         &       (.01)         &       (.01)         &       (.01)         &       (.01)         \\
\\
Mean dep var          &        .091         &        .091         &        .071         &        .071         &        .082         &        .082         &        .079         &        .079         \\

R$^2$          	  &         .01         &         .10         &         .01         &         .06         &         .00         &         .48         &         .00         &         .44         \\
Controls		    	& 					   & $\checkmark$  		&						& $\checkmark$ & 			  & $\checkmark$  		&						& $\checkmark$  \\
Bandwidth    	&        5         &        5         &        11         &        11   	&        5         &        5         &        11         &        11   \\

Obs      		&        1294         &        1294         &        2587         &        2587         &        1297         &        1297         &        2595         &        2595         \\

%\multicolumn{1}{l}{} &  \multicolumn{4}{c}{Prb(Ramo-33 transfers $>$ 0)}  \\
%\cmidrule(l{1mm}r{1mm}){2-5}
%Political alignment  	     
%\\
%Control's mean dep var                          \\
%R$^2$          					
%\midrule



\bottomrule							

\end{tabular}%
}
}
    \end{center}
    \begin{tablenotes}
    \vskip
    \vspace{0.5em}
    \footnotesize	
	\textit{Note. }This table reports the estimates of political alignment from equation (2). The sample includes post electoral years of all municipalities with close elections during the period 1998-2003. The outcome variables are measure as a three year changes. Controls refers to state fixed effects, election-year fixed effects, and baseline political characteristics (incumbency status, previous political alignment, previous political party). Mean dep var refers to the sample average of the outcome variable for the non-aligned municipalities.
    \end{tablenotes}
    
\end{table}     

\newpage

\begin{table}[h]
    \small
    \begin{center}
        \caption{{Effects on 100 $\times$ Probability of Employment by Formal and Informal Sector}}\label{tab:censuslfs}
        \resizebox{!}{!}{
\resizebox{1\textwidth}{!}{
\begin{tabular}{lcccc}
\hline
\hline
&  \multicolumn{2}{c}{100 $\times$ Prob. of employed} &  \multicolumn{2}{c}{Emp. growth (98-08)}\\
\cmidrule(l{1mm}r{1mm}){2-3} \cmidrule(l{1mm}r{1mm}){4-5} 
&  (1)      & (2)           & (3)          & (4)         \\
\midrule
\multicolumn{5}{l}{\bf Panel A. Bandwith 5 pp}  \\
\\	

Political Alignment                     
                                    &   -3.384\sym{***} &   -3.412\sym{***}    
                                    &   -.0765          &   -.0598  
                                    
                                    \\
                                     
                                     &  (1.06)  &   (.93)   
                                     &   (.08)  &   (.06)          
                                     \\ 
                                    \\
R$^2$                               
                                    &         .02         &         .10 
                                    &     .00         &         .27          \\
Observations                            
                                    &  1,004424         &     1,004424  
                                    &  467         &         467   
                                    \\
\midrule
\multicolumn{7}{l}{\bf Panel B. Bandwith 11 pp }  \\
\\	
Political Alignment             
                                &  -3.061\sym{***}   &      -3.156\sym{***}
                                &    -.0614         &      -.0347 
                                    
                                    \\
                                   
                                    &     (.95)         &       (.85)        
                                    &     (.04)         &       (.03)    
                                    \\ 
                                    \\
                                    
R$^2$                               
                                    
                                    &         .00         &         .24      
                                    &         .01         &         .20   \\
Observations                            
                                   
                                    &  2,102277   &     2,102277  
                                    &    945     &         945                      \\
\midrule

Controls                 &         & $\checkmark$                        &         & $\checkmark$                       \\ 

\bottomrule

\end{tabular}
}

}
    \end{center}
    \begin{tablenotes}
    \vskip
    \vspace{0.5em}
    \footnotesize	
	\textit{Note. }This table reports the estimates of political alignment from equation (2). The estimates include municipalities with close elections between 1998 and 2003. The outcome variables are measure as a three year changes. Controls refers to state fixed effects, election-year fixed effects, and baseline political characteristics (incumbency status, previous political alignment, previous political party). Mean dep var refers to the sample average of the outcome variable for the non-aligned municipalities. Columns 3 and 4 use as outcome variable the change in private municipal employment between the 2008 and 1998 round of the economic census.
    \end{tablenotes}
    
\end{table} 

\newpage

\begin{table}[h]
    \begin{center}
        \caption{Effects on 100 $\times$ Probability of Employment by Formal and Informal Sector}\label{tab:lfs1}
        
\resizebox{\textwidth}{!}{ 

\begin{tabular}{lcccccc}

	
	\hline
	\hline
	
%	\multicolumn{1}{l}{} &  \multicolumn{4}{c}{Vote margin bandwidth}  \\
\\
\multicolumn{1}{l}{} &  \multicolumn{2}{c}{Formal employment}  &  \multicolumn{2}{c}{Informal employment}  &  \multicolumn{2}{c}{Total employment}    \\
%&  \multicolumn{2}{c}{Unemployment}  &  \multicolumn{2}{c}{Labor force} 
\cmidrule(l{1mm}r{1mm}){2-3} \cmidrule(l{1mm}r{1mm}){4-5} \cmidrule(l{1mm}r{1mm}){6-7}
											&  (1) 
											&  (2) 
											&  (3) 
											&  (4)   
											&  (5) 
											&  (6)  \\
\midrule
\multicolumn{7}{l}{\bf Panel A. Bandwith 5 pp N=1004424 }  \\
\\	


Political alignment         

     &      -2.360\sym{***}&      -2.100\sym{***}&      -1.024         &      -1.312         &      -3.384\sym{***}&      -3.412\sym{***}  \\
            &       (.48)         &       (.61)         &       (.86)         &       (.79)         &      (1.06)         &       (.93)             \\
\\
R$^2$        &         .02         &         .15         &         .02         &         .10         &         .00         &         .24                 \\


\midrule
\multicolumn{7}{l}{\bf Panel B. Bandwith 11 pp N=2102277 }  \\
\\	

Political alignment         
      &      -1.707\sym{***}&      -1.976\sym{***}&      -1.354\sym{*}  &      -1.180\sym{*}  &      -3.061\sym{***}&      -3.156\sym{***}       \\
            &       (.42)         &       (.39)         &       (.70)         &       (.69)         &       (.95)         &       (.85)         &        \\

R$^2$            &         .02         &         .15         &         .02         &         .10         &         .00         &         .24                \\
\midrule
\\

Municipality FE 	
            & $\checkmark$  							   
            & $\checkmark$  		
            & $\checkmark$ 						
            & $\checkmark$ 
            & $\checkmark$ 			  
            & $\checkmark$   \\
Election Year $\times$\ Year FE 	
            & $\checkmark$  							   
            & $\checkmark$  		
            & $\checkmark$ 						
            & $\checkmark$ 
            & $\checkmark$ 			  
            & $\checkmark$   \\
Individual $\times$\ Year FE 	
            & 					   
            & $\checkmark$  		
            &						
            & $\checkmark$ 
            & 			  
            & $\checkmark$   \\

\bottomrule							

\end{tabular}
}



    \end{center}
    \begin{tablenotes}
    \vskip
    \vspace{0.5em}
    \footnotesize	
	\textit{Note. }This table reports the estimates of political alignment from equation (2). The sample includes post electoral years of all municipalities with close elections during the period 1998-2003. The outcome variables are measure as a three year changes. Controls refers to state fixed effects, election-year fixed effects, and baseline political characteristics (incumbency status, previous political alignment, previous political party). Mean dep var refers to the sample average of the outcome variable for the non-aligned municipalities.
    \end{tablenotes}
    
\end{table}     
   
   
 %Re-election
 %Night light, vol ele
    
\newpage


\begin{table}[h]
    \small
    \begin{center}
        \caption{{Effect of Alignment by Pre-election Economic Cycle}}\label{tab:goodbad}
        \resizebox{!}{!}{
\resizebox{!}{!}{
\begin{tabular}{lcccc}
\hline
\hline
 &  \multicolumn{2}{c}{Public spending} &  \multicolumn{2}{c}{Private-Employment} \\
\cmidrule(l{1mm}r{1mm}){2-3} \cmidrule(l{1mm}r{1mm}){4-5} 
&  (1)      & (2)           & (3)          & (4)         \\
\midrule
\multicolumn{3}{l}{\bf Panel A. High pre-election growth}  \\
\\	

Political Alignment                     
               &     .113         &        .131\sym{**}  &       -.130\sym{**}  &       -.137\sym{**}   \\
               &       (.08)         &       (.06)   &       (.06)         &       (.06)            \\
                                    \\

Control mean dep var        &         .63         &         .63    &        .050         &        .050            \\
R$^2$                       &         .01         &         .49    &         .01         &         .10         \\
Observations                &        1198         &        1146    &        1248         &        1248          \\            

\midrule
\multicolumn{3}{l}{\bf Panel B. Low pre-election growth }  \\
\\	
Political Alignment             
        &        .155\sym{**} &       .0934\sym{**}  &      -.0711         &      -.0670\sym{*}     \\
        &       (.06)         &       (.05)          &       (.04)         &       (.04)             \\
                                    \\

Control mean dep var        &         .51         &         .51         &        .089         &        .089     \\
R$^2$                       &         .01         &         .52         &         .00         &         .09          \\
Observations                &        1441         &        1441        &        1568         &        1568        \\


\midrule

Controls                    &                &   $\checkmark$                        &         & $\checkmark$                       \\ 

\bottomrule

\end{tabular}
}

}
    \end{center}
    \begin{tablenotes}
    \vskip
    \vspace{0.5em}
    \footnotesize	
	\textit{Note. }This table reports the estimates of political alignment from equation (2). The sample includes post electoral years of all municipalities with close elections during the period 1998-2003. The outcome variables are measure as a three year changes. Controls refers to state fixed effects, election-year fixed effects, and baseline political characteristics (incumbency status, previous political alignment, previous political party). Mean dep var refers to the sample average of the outcome variable for the non-aligned municipalities.
    \end{tablenotes}
    
\end{table}     

\newpage

\begin{table}[h]
    \small
    \begin{center}
        \caption{{Effect of Alignment on Employment by Tradable and Non-tradable}}\label{tab:tradable}
        \resizebox{!}{!}{
\resizebox{!}{!}{
\begin{tabular}{lcccc}
\hline
\hline
&  \multicolumn{4}{c}{Private-Employment}\\
&  \multicolumn{2}{c}{Tradable} &  \multicolumn{2}{c}{Non-Tradable}\\
\cmidrule(l{1mm}r{1mm}){2-3} \cmidrule(l{1mm}r{1mm}){4-5} 
&  (1)      & (2)           & (3)          & (4)         \\
\midrule
\multicolumn{3}{l}{\bf Panel A. Bandwidth 5 pp, N= 1220}  \\
\\	

Political Alignment                     
            &       -.157\sym{**} &       -.146\sym{**} &      -.0145         &      -.0235         \\
            &       (.07)         &       (.07)         &       (.06)         &       (.06)         \\
                                    \\

Control mean dep var       &        .020         &        .020         &         .15         &         .15         \\
R$^2$                      &         .01         &         .09         &         .00         &         .07         \\

\midrule
\multicolumn{3}{l}{\bf Panel B. Bandwidth 11 pp, N= 2421 }  \\
\\	
Political Alignment             
        &       -.134\sym{***}&       -.139\sym{***}&     -.00995         &      .00153         \\
        &       (.05)         &       (.05)         &       (.04)         &       (.04)         \\
                                    \\

Control mean dep var        &      -.0026         &      -.0026         &         .16         &         .16         \\
R$^2$                       &         .01         &         .05         &         .00         &         .05         \\


\midrule

Controls                    &                &   $\checkmark$                        &         & $\checkmark$                       \\ 

\bottomrule

\end{tabular}
}

}
    \end{center}
    \begin{tablenotes}
    \vskip
    \vspace{0.5em}
    \footnotesize	
	\textit{Note. }This table reports the estimates of political alignment from equation (2). The sample includes post electoral years of all municipalities with close elections during the period 1998-2003. The outcome variables are measure as a three year changes. Controls refers to state fixed effects, election-year fixed effects, and baseline political characteristics (incumbency status, previous political alignment, previous political party). Mean dep var refers to the sample average of the outcome variable for the non-aligned municipalities.
    \end{tablenotes}
    
\end{table}     


\newpage

\begin{table}[h]
    \small
    \begin{center}
        \caption{{Effect of Alignment on Employment by Pre-election Share of Government Dependent Sectors}}\label{tab:gdsectors}
        \resizebox{0.8\linewidth}{!}{
\resizebox{\linewidth}{!}{
\begin{tabular}{lcccc}
\hline
\hline
&  \multicolumn{2}{c}{Public Spending} &  \multicolumn{2}{c}{Priv Employment} \\
\cmidrule(l{1mm}r{1mm}){2-3} \cmidrule(l{1mm}r{1mm}){4-5} 
&  (1)      & (2)           & (3)          & (4)         \\

\midrule
\multicolumn{5}{l}{\bf Panel A. High share of GD sectors, N=1300} \\
\\
Political Alignment                     
            &        .134\sym{*}  &       .0995\sym{*} &      -.0682         &      -.0754\sym{*}  \\
         &       (.08)         &       (.06)       &       (.05)         &       (.04)             \\
                                    \\

Control mean dep var      &         .57         &         .57    &        .067         &        .067              \\
R$^2$                     &         .01         &         .50      &         .00         &         .08            \\

\midrule
\multicolumn{5}{l}{\bf Panel B. Low share of GD sectors,  N=1275 }  \\
\\	
Political Alignment             
            &        .154\sym{*}  &        .114\sym{**}  &       -.110\sym{**} &       -.118\sym{**} \\
               &       (.08)         &       (.05)     &       (.05)         &       (.06)          \\
                                    \\

Control mean dep var        &         .56         &         .56       &        .077         &        .077         \\
R$^2$                       &         .01         &         .50       &         .01         &         .12       \\


\midrule

Controls                    &                &   $\checkmark$                        &         & $\checkmark$                       \\ 

\bottomrule

\end{tabular}
}

}
    \end{center}
    \begin{tablenotes}
    \vskip
    \vspace{0.5em}
    \footnotesize	
	\textit{Note. }This table reports the estimates of political alignment from equation (2). The sample includes post electoral years of all municipalities with close elections during the period 1998-2003. The outcome variables are measure as a three year changes. Controls refers to state fixed effects, election-year fixed effects, and baseline political characteristics (incumbency status, previous political alignment, previous political party). Mean dep var refers to the sample average of the outcome variable for the non-aligned municipalities.
    \end{tablenotes}
    
\end{table}     

\newpage

\begin{table}[h]
    \small
    \begin{center}
        \caption{{Effect of Alignment on Consumption}}\label{tab:growth}
        \resizebox{!}{!}{
\resizebox{!}{!}{
\begin{tabular}{lcccc}
\hline
\hline

&  \multicolumn{2}{c}{ Night lights} &  \multicolumn{2}{c}{ \shortstack{Electricity \\ consumption} }\\
\cmidrule(l{1mm}r{1mm}){2-3} \cmidrule(l{1mm}r{1mm}){4-5} 
&  (1)      & (2)           & (3)          & (4)         \\
\midrule
\multicolumn{2}{l}{\bf Panel A. Bandwidth 5 pp}  \\

Political Alignment                     
            &       .0738\sym{*}  &       .0376         &       .0905         &       .0359         \\
            &       (.04)         &       (.03)         &       (.07)         &       (.06)         \\
                                    \\

Control mean dep var    &        .028         &        .028         &         .11         &         .11         \\
R$^2$                   &         .00         &         .75         &         .01         &         .12         \\
Observations             &        1313         &        1313         &         887         &         887         \\
\midrule
\multicolumn{2}{l}{\bf Panel B. Bandwidth 11 pp}  \\
Political Alignment             
            &       .0363         &       .0171         &       .0328         &     .000         \\
            &       (.03)         &       (.02)         &       (.05)         &       (.05)         \\
         
                                    \\

Control mean dep var    &        .022         &        .022         &         .13         &         .13         \\
R$^2$                   &         .00         &         .73         &         .00         &         .07         \\
Observations            &        2638         &        2638         &        1704         &        1704         \\

\midrule

Controls                    &                &   $\checkmark$                        &         & $\checkmark$                       \\ 

\bottomrule

\end{tabular}
}

}
    \end{center}
    \begin{tablenotes}
    \vskip
    \vspace{0.5em}
    \footnotesize	
	\textit{Note. }This table reports the estimates of political alignment from equation (2). The sample includes post electoral years of all municipalities with close elections during the period 1998-2003. The outcome variables are measure as a three year changes. Controls refers to state fixed effects, election-year fixed effects, and baseline political characteristics (incumbency status, previous political alignment, previous political party). Mean dep var refers to the sample average of the outcome variable for the non-aligned municipalities.
    \end{tablenotes}
    
\end{table}     

\newpage
 
\begin{table}[h]
    \begin{center}
        \caption{{Effects on 100 $\times$ Probability of Being Employed, Unemployed and Part of the Labor Force}}\label{tab:lfs2}
        \resizebox{!}{!}{
\resizebox{\textwidth}{!}{ 

\begin{tabular}{lcccccc}

	
	\hline
	\hline
	
%	\multicolumn{1}{l}{} &  \multicolumn{4}{c}{Vote margin bandwidth}  \\
\\
\multicolumn{1}{l}{} &  \multicolumn{2}{c}{Total employment} &  \multicolumn{2}{c}{Unemployment}  &  \multicolumn{2}{c}{Labor force}     \\
%&  \multicolumn{2}{c}{Unemployment}  &  \multicolumn{2}{c}{Labor force} 
\cmidrule(l{1mm}r{1mm}){2-3} \cmidrule(l{1mm}r{1mm}){4-5} \cmidrule(l{1mm}r{1mm}){6-7}
											&  (1) 
											&  (2) 
											&  (3) 
											&  (4)   
											&  (5) 
											&  (6)  \\


\midrule
\multicolumn{7}{l}{\bf Panel A. Bandwith 5 pp N=1004424 }  \\
\\	

Political alignment         

   &      -3.384\sym{***}&      -3.412\sym{***}&        .448         &        .534         &      -2.936\sym{***}&      -2.878\sym{***} \\
    &      (1.06)         &       (.93)         &       (.36)         &       (.35)         &       (.84)         &       (.70)         \\

\\
R$^2$    &         .00         &         .24         &         .00         &         .01         &         .00         &         .25         \\



\midrule
\multicolumn{7}{l}{\bf Panel B. Bandwith 11 pp N=2102277 }  \\
\\	

Political alignment         
     &      -3.061\sym{***}&      -3.156\sym{***}&        .509\sym{*}  &        .557\sym{*}  &      -2.552\sym{***}&      -2.599\sym{***} \\
    
    &       (.95)         &       (.85)         &       (.29)         &       (.28)         &       (.79)         &       (.70)   \\

R$^2$           &         .00         &         .24         &         .00         &         .01         &         .00         &         .25            \\
\midrule
\\

Municipality FE 	
            & $\checkmark$  							   
            & $\checkmark$  		
            & $\checkmark$ 						
            & $\checkmark$ 
            & $\checkmark$ 			  
            & $\checkmark$   \\
Election Year $\times$\ Year FE 	
            & $\checkmark$  							   
            & $\checkmark$  		
            & $\checkmark$ 						
            & $\checkmark$ 
            & $\checkmark$ 			  
            & $\checkmark$   \\
Individual $\times$\ Year FE 	
            & 					   
            & $\checkmark$  		
            &						
            & $\checkmark$ 
            & 			  
            & $\checkmark$   \\

\bottomrule							

\end{tabular}
}


}
    \end{center}
    \begin{tablenotes}
    \vskip
    \vspace{0.5em}
    \footnotesize	
	\textit{Note. }This table reports the estimates of political alignment from equation (2). The sample includes post electoral years of all municipalities with close elections during the period 1998-2003. The outcome variables are measure as a three year changes. Controls refers to state fixed effects, election-year fixed effects, and baseline political characteristics (incumbency status, previous political alignment, previous political party). Mean dep var refers to the sample average of the outcome variable for the non-aligned municipalities.
    \end{tablenotes}
    
\end{table}     

\newpage

\begin{table}[h] 
    \small
    \begin{center}
        \caption{{Effect of Alignment on Probability of Winning Subsequent Elections}}\label{tab:votes}
        \resizebox{!}{!}{
\resizebox{!}{!}{
\begin{tabular}{lcccc}
\hline
\hline

&  \multicolumn{2}{c}{\shortstack{\\Prob of winning \\ next election}} &  \multicolumn{2}{c}{ \shortstack{\\Prob of winning \\ next two elections} }\\
\cmidrule(l{1mm}r{1mm}){2-3} \cmidrule(l{1mm}r{1mm}){4-5} 
&  (1)      & (2)           & (3)          & (4)         \\
\midrule
\multicolumn{3}{l}{\bf Panel A. Bandwidth 5 pp, N=1313}  \\

Political Alignment                     
             &       .0648         &       .0410         &       .0828         &       .0925         \\
            &       (.09)         &       (.09)         &       (.06)         &       (.06)         \\
                                    \\

Control mean dep var    &         .25         &         .25         &        .077         &        .077         \\
R$^2$                   &         .03         &         .15         &         .04         &         .19         \\

\midrule
\multicolumn{3}{l}{\bf Panel B. Bandwidth 11 pp, N=2639}  \\
Political Alignment             
            &        .156\sym{**} &        .134\sym{**} &        .128\sym{***}&        .130\sym{***}\\
           &       (.06)         &       (.06)         &       (.05)         &       (.05)         \\
         
                                    \\

Control mean dep var   &         .31         &         .31         &         .11         &         .11         \\
R$^2$                 &         .03         &         .11         &         .03         &         .12         \\

\midrule

Controls                    &                &   $\checkmark$                        &         & $\checkmark$                       \\ 

\bottomrule

\end{tabular}
}

}
    \end{center}
    \begin{tablenotes}
    \vskip
    \vspace{0.5em}
    \footnotesize	
	\textit{Note. }This table reports the estimates of political alignment from equation (2). The sample includes post electoral years of all municipalities with close elections during the period 1998-2003. The outcome variables are measure as a three year changes. Controls refers to state fixed effects, election-year fixed effects, and baseline political characteristics (incumbency status, previous political alignment, previous political party). Mean dep var refers to the sample average of the outcome variable for the non-aligned municipalities.
    \end{tablenotes}
    
\end{table}     

\newpage

\begin{table}[h]
    \small
    \begin{center}
        \caption{{Effect of Alignment on Public Employment and Wages}}\label{tab:public}
        \resizebox{!}{!}{
\resizebox{!}{!}{
\begin{tabular}{lcccc}
\hline
\hline

&  \multicolumn{2}{c}{\shortstack{Public \\ Employment}} &  \multicolumn{2}{c}{ \shortstack{\\ Public \\ Wage bill}  } \\

\cmidrule(l{1mm}r{1mm}){2-3} \cmidrule(l{1mm}r{1mm}){4-5} 
&  (1)      & (2)           & (3)          & (4)         \\
\midrule
\multicolumn{5}{l}{\bf Panel A. Bandwidth 5 pp, N= 1313}  \\
\\	

Political Alignment                     
            &        .309\sym{*}  &      -.0243         &        .208\sym{*}  &        .128         \\
            &       (.18)         &       (.05)         &       (.11)         &       (.10)         \\
                                    \\

Control mean dep var    &         .46         &         .46         &         .51         &         .51        \\
R$^2$                   &         .00         &         .90         &         .01         &         .32         \\

\midrule
\multicolumn{5}{l}{\bf Panel B. Bandwidth 11 pp, N= 2639}  \\
\\	
Political Alignment             
            &        .191         &     -.00319         &        .121         &       .0935         \\
            &       (.14)         &       (.05)         &       (.08)         &       (.07)         \\
                                    \\

Control mean dep var       &         .50         &         .50         &         .54         &         .54         \\
R$^2$                      &         .00         &         .89         &         .00         &         .35         \\


\midrule

Controls                    &                &   $\checkmark$                        &         & $\checkmark$                       \\ 

\bottomrule

\end{tabular}
}

}
    \end{center}
    \begin{tablenotes}
    \vskip
    \vspace{0.5em}
    \footnotesize	
	\textit{Note. }This table reports the estimates of political alignment from equation (2). The sample includes post electoral years of all municipalities with close elections during the period 1998-2003. The outcome variables are measure as a three year changes. Controls refers to state fixed effects, election-year fixed effects, and baseline political characteristics (incumbency status, previous political alignment, previous political party). Mean dep var refers to the sample average of the outcome variable for the non-aligned municipalities.
    \end{tablenotes}
    
\end{table}     

\newpage

\begin{table}[h]
    \small
    \begin{center}
        \caption{{Effect of Alignment on Public Investment}}\label{tab:infra2}
        \resizebox{!}{!}{
\resizebox{!}{!}{
\begin{tabular}{lcccc}

\hline
\hline

& \multicolumn{2}{c}{Bandwith 5pp }  & \multicolumn{2}{c}{Bandwith 11pp }   \\                                                                                       
\cmidrule(l{1mm}r{1mm}){2-3} \cmidrule(l{1mm}r{1mm}){4-5} 
                              & (1)      & (2)           & (3)          & (4)  \\
\hline
                                         

Political Alignment                     
             &        .383         &        .406\sym{**} &        .363         &        .282\sym{***}\\
             &       (.33)         &       (.16)         &       (.22)         &       (.10)         \\
            \\

Control mean dep var        &        1.20         &        1.20         &        1.14         &        1.14         \\
R$^2$                       &         .19         &         .19         &         .20         &         .20         \\
Observations                &        1313         &        1313         &        2639         &        2639         \\
Controls		    	    & 					  &  $\checkmark$  		&					  & $\checkmark$      \\
\bottomrule
\end{tabular}
}}
    \end{center}
    \begin{tablenotes}
    \vskip
    \vspace{0.5em}
    \footnotesize	
	\textit{Note. }This table reports the estimates of political alignment from equation (2). The sample includes post electoral years of all municipalities with close elections during the period 1998-2003. The outcome variables are measure as a three year changes. Controls refers to state fixed effects, election-year fixed effects, and baseline political characteristics (incumbency status, previous political alignment, previous political party). Mean dep var refers to the sample average of the outcome variable for the non-aligned municipalities.
    \end{tablenotes}
    
\end{table}     

\newpage

\begin{table}[h]
    \small
    \begin{center}
        \caption{{Effect of Alignment on Public Spending by Categories}}\label{tab:infra}
        \resizebox{1\linewidth}{!}{
\resizebox{!}{!}{
{
\def\sym#1{\ifmmode^{#1}\else\(^{#1}\)\fi}
\begin{tabular}{lcccccc}
	\hline
	\hline
	\\
	\multicolumn{1}{l}{} &  \multicolumn{1}{c}{\shortstack{Infrastructure \\ Investment}} 
	                     &  \multicolumn{1}{c}{\shortstack{Service \\ Contracts }} 
	                     &  \multicolumn{1}{c}{\shortstack{Subsidies \\ \& Transfers}} 
	                     &  \multicolumn{1}{c}{\shortstack{Public workers \\ salaries}} 
	                     &  \multicolumn{1}{c}{\shortstack{Other\\spending}}
	                     &  \multicolumn{1}{c}{\shortstack{Total\\spending}}  
	                     \\
	
	\cmidrule(l{1mm}r{1mm}){2-6} \cmidrule(l{1mm}r{1mm}){7-7} 
											&  (1) 
											&  (2) 
											&  (3) 
											&  (4)  
											& (5) 
											& (6)  \\
\midrule
\multicolumn{7}{l}{\bf Panel A. Bandwith 5 pp N=1313 }  \\
\\	

Political alignment 
 &        .406\sym{**} &        .252\sym{**} &        .270\sym{**} &        .128         &       .0525         &        .121\sym{***}\\
            &       (.16)         &       (.11)         &       (.11)         &       (.10)         &       (.09)         &       (.05)         \\\\
Control mean dep var 	
	 &        1.20         &         .28         &         .64         &         .51         &         .45         &         .56         \\
\% of spending    	
		 &         .19         &         .15         &         .14         &         .32         &         .22         &           1         \\
R$^2$          		
	    &         .83         &         .42         &         .71         &         .36         &         .57         &         .68         \\

\midrule
\multicolumn{7}{l}{\bf Panel B. Bandwith 11 pp N=2639 }  \\
\\

Political alignment 
        &        .282\sym{***}&        .159\sym{**} &       .0650         &       .0935         &       .0739         &        .103\sym{***}\\
        &       (.10)         &       (.07)         &       (.08)         &       (.07)         &       (.06)         &       (.03)         \\
\\
Control mean dep var 	
		 &        1.14         &         .31         &         .57         &         .54         &         .46         &         .57         \\

\% of revenues    	
		 &         .20         &         .15         &         .14         &         .31         &         .21         &           1         \\
R$^2$          		
		 &         .81         &         .40         &         .69         &         .35         &         .54         &         .67         \\

\bottomrule							




\end{tabular}
}
}

}
    \end{center}
    \begin{tablenotes}
    \vskip
    \vspace{0.5em}
    \footnotesize	
	\textit{Note. }This table reports the estimates of political alignment from equation (2). The sample includes post electoral years of all municipalities with close elections during the period 1998-2003. The outcome variables are measure as a three year changes. Controls refers to state fixed effects, election-year fixed effects, and baseline political characteristics (incumbency status, previous political alignment, previous political party). Mean dep var refers to the sample average of the outcome variable for the non-aligned municipalities.
    \end{tablenotes}
    
\end{table}     

\newpage

\begin{table}[h]
    \small
    \begin{center}
        \caption{{Effect of Alignment on Inputs of Infrastructure Spending: Construction Jobs}}\label{tab:cons_jobs}
        \resizebox{!}{!}{
\resizebox{!}{!}{
\begin{tabular}{lcccc}
\hline
\hline

&  \multicolumn{2}{c}{ Employment} &  \multicolumn{2}{c}{ \shortstack{Wages} }\\
\cmidrule(l{1mm}r{1mm}){2-3} \cmidrule(l{1mm}r{1mm}){4-5} 
&  (1)      & (2)           & (3)          & (4)         \\
\midrule
\multicolumn{4}{l}{\bf Panel A. Bandwidth 5 pp, N= 1313}  \\
\\	

Political Alignment                     
            &       -.211         &      -.0930         &        .299\sym{**} &       .0769         \\
            &       (.16)         &       (.15)         &       (.15)         &       (.06)         \\
                                    \\

Control mean dep var    &         .18         &         .18         &         .40         &         .40         \\
\% total jobs           &         .10         &         .10         &                     &                     \\
R$^2$                   &         .01         &         .16         &         .01         &         .77         \\

\midrule
\multicolumn{4}{l}{\bf Panel B. Bandwidth 11 pp, N= 2639}  \\
\\	
Political Alignment             
            &       -.256\sym{**} &       -.222\sym{**} &        .111         &       .0366         \\
            &       (.10)         &       (.10)         &       (.09)         &       (.04)         \\
                                    \\

Control mean dep var        &         .15         &         .15         &         .40         &         .40         \\
\% total jobs               &         .11         &         .11         &                     &                     \\
R$^2$                       &         .01         &         .12         &         .00         &         .64         \\


\midrule

Controls                    &                &   $\checkmark$                        &         & $\checkmark$                       \\ 

\bottomrule

\end{tabular}
}

}
    \end{center}
    \begin{tablenotes}
    \vskip
    \vspace{0.5em}
    \footnotesize	
	\textit{Note. }This table reports the estimates of political alignment from equation (2). The sample includes post electoral years of all municipalities with close elections during the period 1998-2003. The outcome variables are measure as a three year changes. Controls refers to state fixed effects, election-year fixed effects, and baseline political characteristics (incumbency status, previous political alignment, previous political party). Mean dep var refers to the sample average of the outcome variable for the non-aligned municipalities.
    \end{tablenotes}
    
\end{table}

\newpage

\begin{table}[h]
    \small
    \begin{center}
        \caption{{Effect of Alignment on Outputs of Infrastructure Spending: Stock of Public Infrastructure}}\label{tab:infrastructure}
        \resizebox{!}{!}{
\resizebox{!}{!}{
\begin{tabular}{lcccc}

\toprule

\multicolumn{1}{l}{} &  \multicolumn{4}{c}{ $\Delta$ 1995-2010 log points difference } \\
\cmidrule(l{1mm}r{1mm}){2-3} \cmidrule(l{1mm}r{1mm}){4-5} 
& (1) & (2) & (3) & (4) \\

\midrule

\multicolumn{4}{l}{\textit{\bf Public infrastructure}}   \\  


\textit{Sewerage service}   &  2.765   &
						   0.841   &
						   0.551   &
						   -0.248   \\
						   
						&   \tiny{(3.575)}   &
						    \tiny{(2.930)}   &
						    \tiny{(2.369)}   &
						    \tiny{(1.940)}   \\


\textit{Electric lighting}   &  3.014   &
						   3.147   &
						   3.278*   &
						   2.368   \\
						   
						&   \tiny{(2.682)}   &
						    \tiny{(2.808)}   &
						    \tiny{(1.739)}   &
						    \tiny{(1.721)}   \\
						



\textit{Piped water}   &  5.398*   &
						   4.378   &
						   2.860   &
						   2.329   \\
						   
						&   \tiny{(2.840)}   &
						    \tiny{(3.136)}   &
						    \tiny{(1.907)}   &
						    \tiny{(1.912)}   \\




\multicolumn{4}{l}{\textit{\bf Private assets}}   \\  

\textit{Overcrowding}   &  -2.620***   &
						   -1.925**   &
						   -1.747**   &
						   -1.320**   \\
						   
						&   \tiny{(0.982)}   &
						    \tiny{(0.941)}   &
						    \tiny{(0.706)}   &
						    \tiny{(0.616)}   \\




\textit{Concrete floor}   &  3.857   &
						   1.887   &
						   3.319*   &
						   1.594   \\
						   
						&   \tiny{(2.494)}   &
						    \tiny{(2.395)}   &
						    \tiny{(1.738)}   &
						    \tiny{(1.600)}   \\

Observations				&	461 &
						    461 &
						    934 &
						    934 \\



\midrule

{\bf Controls}  			   		    	 &	      &
										     $\checkmark$ &
										     &
										     $\checkmark$ \\

{\bf Bandwith}  			   		    	 &	5      
											 &  5
											 &  11
										     &  11 \\

\bottomrule

\end{tabular}%
}
}
    \end{center}
    \begin{tablenotes}
    \vskip
    \vspace{0.5em}
    \footnotesize	
	\textit{Note. }This table reports the estimates of political alignment from equation (2). The sample includes post electoral years of all municipalities with close elections during the period 1998-2003. The outcome variables are measure as a three year changes. Controls refers to state fixed effects, election-year fixed effects, and baseline political characteristics (incumbency status, previous political alignment, previous political party). Mean dep var refers to the sample average of the outcome variable for the non-aligned municipalities.
    \end{tablenotes}
    
\end{table} 

\newpage

\begin{table}[h]
    \small
    \begin{center}
        \caption{{Effect of Alignment on Homicides}}\label{tab:violence}
        \resizebox{!}{!}{
\resizebox{!}{!}{
\begin{tabular}{lcccc}
\hline
\hline

&  \multicolumn{2}{c}{Prob (Homicide$>$0)} &  \multicolumn{2}{c}{Homicide rate}\\
\cmidrule(l{1mm}r{1mm}){2-3} \cmidrule(l{1mm}r{1mm}){4-5} 
&  (1)      & (2)           & (3)          & (4)         \\
\midrule
\multicolumn{4}{l}{\bf Panel A. Bandwidth 5 pp, N= 1428}  \\
\\	

Political Alignment                     
            &      -.0599         &       .0313         &       -.178         &       .0963         \\
            &       (.05)         &       (.05)         &       (.16)         &       (.15)         \\
                                    \\

Control mean dep var      &       .0054         &       .0054         &       -.099         &       -.099         \\
R$^2$                    &         .00         &         .34         &         .00         &         .31         \\

\midrule
\multicolumn{4}{l}{\bf Panel B. Bandwidth 11 pp, N= 2871}  \\
\\	
Political Alignment             
            &      -.0132         &       .0406         &      -.0840         &       .0754         \\
            &       (.04)         &       (.04)         &       (.11)         &       (.11)         \\
                                    \\

Control mean dep var       &       .0081         &       .0081         &       -.086         &       -.086         \\
R$^2$                      &         .00         &         .32         &         .00         &         .29         \\


\midrule

Controls                    &                &   $\checkmark$                        &         & $\checkmark$                       \\ 

\bottomrule

\end{tabular}
}

}
    \end{center}
    \begin{tablenotes}
    \vskip
    \vspace{0.5em}
    \footnotesize	
	\textit{Note. }This table reports the estimates of political alignment from equation (2). The sample includes post electoral years of all municipalities with close elections during the period 1998-2003. The outcome variables are measure as a three year changes. Controls refers to state fixed effects, election-year fixed effects, and baseline political characteristics (incumbency status, previous political alignment, previous political party). Mean dep var refers to the sample average of the outcome variable for the non-aligned municipalities.
    \end{tablenotes}
    
\end{table} 

\newpage

\begin{table}[h]
    \small
    \begin{center}
        \caption{{Effect of Alignment on Corruption}}\label{tab:corruption}
        \resizebox{!}{!}{
\resizebox{!}{!}{
\begin{tabular}{lcccccc}
\hline
\hline
 &  \multicolumn{2}{c}{Prb(Audited)}  &  \multicolumn{2}{c}{Prb(corruption$>$10\%)} &  \multicolumn{2}{c}{Prb(malfeaseance$>$10\%)} \\
\cmidrule(l{1mm}r{1mm}){2-3} \cmidrule(l{1mm}r{1mm}){4-5}  \cmidrule(l{1mm}r{1mm}){6-7} 
&  (1)      & (2)           & (3)          & (4)     & (5)  &  (6)    \\
\midrule
\multicolumn{4}{l}{\bf Panel A. Bandwidth 5 pp}  \\
\\	

Political Alignment                     
              &       .0112         &       .0126         &        .111         &       .085         &       -.580\sym{***}&       -.455\sym{**} \\
              &       (.01)         &       (.01)         &       (.21)         &       (.19)         &       (.19)         &       (.22)         \\
                                    \\

Control mean dep var        &        .017         &        .017         &         .30         &         .30         &         .35         &         .35         \\
R$^2$                       &         .00         &         .15         &         .04         &         .53         &         .19         &         .69         \\
Observations               &        4137         &        4134         &          83         &          83         &          83         &          83         \\

\midrule
\multicolumn{4}{l}{\bf Panel B. Bandwidth 11 pp }  \\
\\	
Political Alignment             
       &      .00551         &      .00433         &      -.0774         &      -.0853        &       -.420\sym{***}&       -.439\sym{***}\\
       &       (.01)         &       (.01)         &       (.13)         &       (.12)         &       (.14)         &       (.13)         \\
                                    \\

Control mean dep var        &        .018         &        .018         &         .30         &         .30         &         .22         &         .22         \\
R$^2$                       &         .00         &         .15         &         .02         &         .40         &         .10         &         .42         \\
Observations                 &        8652         &        8640         &         186         &         186         &         186         &         186         \\


\midrule

Controls                    &                &   $\checkmark$                        &         & $\checkmark$     &         & $\checkmark$                       \\ 

\bottomrule

\end{tabular}
}

}
    \end{center}
    \begin{tablenotes}
    \vskip
    \vspace{0.5em}
    \footnotesize	
	\textit{Note. }This table reports the estimates of political alignment from equation (2). The sample includes post electoral years of all municipalities with close elections during the period 1998-2003. The outcome variables are measure as a three year changes. Controls refers to state fixed effects, election-year fixed effects, and baseline political characteristics (incumbency status, previous political alignment, previous political party). Mean dep var refers to the sample average of the outcome variable for the non-aligned municipalities.
    \end{tablenotes}
\end{table}

\end{document}
