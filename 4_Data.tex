\documentclass[dv_diss_main.tex]{subfiles}

\begin{document}


\section{Data and Sample}

The objective is to estimate the economic effects of political alignment when politicians can distort the allocation of intergovernmental transfers. To do so, I assemble a municipality-year dataset that combines data on local public finance, electoral results, employment, and wages (public and private) from social security records. I also use alternative datasets to complement the main analysis, namely: labor force surveys, economic censuses, remote sensing data, and federal audits to local governments. This section describes each source of information and the sample used to obtain the main estimates.


\subsection{Data and Measurement}
\subsubsection{Public Finance}  Municipalities produce yearly balance sheets classifying both revenues and spending across different subcategories. The Mexican National Institute of Statistics and Geography (INEGI) collects this data. The revenues side provides information about distinct categories, from intergovernmental transfers to local property taxes and fees for services provided by local governments. On average, 82\% of local revenues come from intergovernmental transfers. In comparison, about 10\% come from taxes (mostly property taxes) and other public services provided by local governments. The revenue data allows me to distinguish which sources of revenue increase due to political alignment. Also, I can observe different components of spending data. The two most important in terms of their average share in total spending are the wage bill (40\%) and the public investment (30\%).

\subsubsection{Elections} Electoral data comes from Centro de Investigación para el Desarrollo (CIDAC) and the state electoral authorities. The election data provides information on the number of votes for each party or coalition for the universe of municipal, state, and presidential elections.\footnote{ I manually collect the elections not provided by CIDAC by requesting the data from the electoral institutions of each state.}
It is important to note that the party affiliation recorded in the electoral data is defined months before the election. Therefore, the measure of alignment is not affected by politicians deciding their political party affiliation after knowing the electoral results. The latter alleviates any concern related to the manipulation of alignment status. 


\subsubsection{Main Employment and Wage Data} To measure employment at the municipal-year level, I combine administrative records from the Mexican Institute of Social Security (IMSS) and the Institute of Social Security of Public Workers (ISSTE). Since both data sources correspond to social security records, they capture the universe of formal employees and employers. IMSS collects data on formal private-sector workers/employers and the ISSTE on formal public sector employees. Formal workers, in this context, are all workers who contribute to the social security system and, therefore, receive health insurance and pension contributions.

In addition to employment counts, the IMSS data allows for the measurement of aggregate wage bills and, therefore, average wages. Also, it provides this data by sector, firm size, gender, and age groups. The data from ISSTE does not report wages; I circumvent this by using the total wage bill of local governments from the public finance data.


\subsubsection{Alternative Employment Data} The main drawback of the data from IMSS and ISSTE is that they remain mute about what is happening to the informal sector and, therefore, to total employment. I use two alternative sources of information to infer the effects of alignment on aggregate (formal and informal) employment and, to some extent, on the informal sector.
The first source of information is the Mexican Economic Census collected every five years by INEGI. It provides detailed municipal-level information for the universe of non-agricultural establishments, both formal and informal. I use this dataset to measure employment and wage bill growth between the 1998 and 2008 rounds of the economic census.  I follow \cite{asher2017politics} by assigning the result of the earliest election that took place between the two rounds of the economic census to each intercensal growth rate. The main drawback of the economic census is that it does not allow us to measure the change in employment precisely before and after every election, which leads estimates to be biased towards zero. 

%This implies less precise estimates\footnote{For example, consider two municipalities that has an election n 1998 (2000) will have 10 (8) years of exposure by 2008; since between that period it if treatment effects vary with time of exposure, this will increase the dispersion of the estimates. Similarly, for municipalities where the earliest election occurred in 1998 will have as base year an electoral year, while a municipality where the earliest election occurred in 2000 will use as base year a non-electoral year. The fact that electoral years are different from non-electoral years implies that one should expect more imprecise estimates when using different base years across specifications.} and downward bias in our estimates.\footnote{The downward bias comes from the fact that treatment changes in between the two census rounds, for example: a municipality that is politically aligned right after 1998 may become non politically aligned before 2003,, creating a downward bias on the effect of being politically aligned the entire period in between the two census rounds.}

The second source of information corresponds to the labor force surveys collected by INEGI. In particular, I use the National Urban Employment Survey (ENEU), which is available at a quarterly level for 1998-2004. This survey is representative of about 48 metropolitan areas and collects a wide variety of socio-demographic and labor market information for both formal and informal workers. I use this data source to complement the analysis of the administrative records, mainly regarding the effects of political alignment on informality and labor force participation. 

The household surveys are not the preferred dataset for two main reasons. The first reason is that ENEU has little overlap with the primary sample used in the estimates. Specifically, it only provides 40\% of the municipality-year observations used in the main estimates. When I limit the sample to municipalities that appear before and after a close election, this percentage declines to 20\%. The second reason is that household surveys are not representative at the municipality level; this implies a large within municipality variation that would limit the ability to detect small effects.\footnote{
I compute the correlation between the growth rate of formal employment captured by IMSS and the ENEU between 1998-2003, uncovering that at the state level, the correlation is 0.57, while at the municipal level, the correlation is 0.2; \cite{bosch2014trade} find similar results.}


\subsubsection{Economic Activity}
To measure aggregate economic activity I use night lights luminosity and electricity consumption. Both are scaled by population, measured in log points, and available at the municipality level. The data on night lights comes from the National Oceanic and Atmospheric Administration (NOAA). Night lights data provides a luminosity measure for every square kilometer of Mexican territory on a scale from 0 to 63. The fact that this information is censored from above may limit the power to find statistically significant effects in cities with several pixels censored at 63. To circumvent this problem, I measure total municipal luminosity growth considering only those pixels that were below 63  by 2003. This is equivalent to assuming that the censored pixels' growth rate is equivalent to neighboring non-censored pixels.

The second source of data is the aggregate consumption of electricity, which comes from the ministry of energy and regulation. This information corresponds to the total energy consumed by both establishments and households. It can be interpreted as a local measure of economic activity that can increase either because residents work more or because they increase their consumption from higher government transfers.


\subsubsection{Other Datasets} I also use other datasets to explore some potential mechanisms and perform balance tests over the data. The sources include: i) The rollout of the number of beneficiaries from Seguro Popular, ii) Monthly payments and beneficiaries from PROGRESA, iii) Official reports from audits of local governments performed by an autonomous watchdog agency (Auditoría Superior de la Federación), and iv) Annual homicides data from INEGI.


\subsection{Sample}

The final dataset  provides variation at the municipal-year level for the sample of elections between 1998 and 2003. Since mayors have three-year term limits and the outcomes are measured as a three-year growth rate, this implies that I follow the dynamics of employment and public finance outcomes for the period of 1996-2006. The key advantages of this study period are: First, it allows me to estimate the effects of political alignment during the expansion and weak oversight of intergovernmental transfers. Second, it ended one year before the sudden and steep increase in violence experienced by Mexico after 2006, which some have argued was the result of political alignment with the president's party.

I exclude municipalities from the state of Oaxaca because most do not choose their mayor through elections; rather, they use a traditional governance structure, which makes it infeasible to construct the running variable.\footnote{Traditional governance structures in Oaxaca are valid after a constitutional reform that took place in 1995. This governance structure is known as \textit{"Usos y costumbres}". The municipalities organized by this type of "polity" are ruled by assemblies rather
than mayors. The election of these assemblies is informal and local leadership
can be arranged by rotating appointments without elections taking place.} Also, I limit the observations to those municipalities for which there is information available on formal private-sector employment for the study period 1998-2006. The final sample considers 1097 out of 2446 municipalities, which employ 99\% of all formal employees in Mexico and host 80\% of the Mexican population. This filter implies that the estimates are representative of mid-sized and large municipalities making the results of this study relevant from a macroeconomic point of view.\footnote{IMSS data records employment for 1,850 out of 2446 municipalities in Mexico. The rest of the municipalities either have none or few formal employees (e.g., less than 10) and therefore that IMSS group them into a larger neighboring municipality.}

\end{document}
