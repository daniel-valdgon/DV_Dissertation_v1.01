\documentclass[dv_diss_main.tex]{subfiles}





%---------------------------------------------%
%----------------Figures----------------------
%---------------------------------------------%



\newcommand{\rddplot}{
NOTE--This plot aggregate data into bins of half percentage points and estimate a third order polynomial regression between the running variable and the bins on each side of the cut-off. 
}


\newcommand{\rdd}{
NOTE--This table reports the estimates of political alignment from equation (2). The sample includes post electoral years of all municipalities with close elections during the period 1998-2003. The outcome variables are measure as a three year changes. Controls refers to state fixed effects, election-year fixed effects, and baseline political characteristics (incumbency status, previous political alignment, previous political party). Mean dep var refers to the sample average of the outcome variable for the non-aligned municipalities. 
}



\newcommand{\sharerev}{
Percentage of revenues is the sample average share of each source of revenue on total revenues for the non-aligned counterparts
}

\newcommand{\shareemp}{
Percentage of employment is the sample average share of each sector on total employment for the non-aligned counterparts
}

\newcommand{\event}{
NOTE--The figure plots the coefficients obtained from the estimation of equation (3) discussed in Section 4. The sample includes all municipalities with close elections during the period 1998-2003. The unit of observation is  the municipal-election pair, for each pair I follow the outcome measures in [-4 +4] years window. The outcome variables are measure in inverse hyperbolic sine points. The tick(thin) lines are 90\%(95\%) confidence intervals. The specification controls by municipality-election and election-year fixed effects. 
}

%inverse hyperbolic sine (IHS) transformation


\newcommand{\stars}{
Standard errors clustered at municipality level.  *** p<0.01, ** p<0.05, * p<0.1.
}

\newcommand{\census}{
}

\newcommand{\household}{
}



\begin{document}

\begin{figure}[H]
\begin{center}
    \begin{tabular}[c]{ccc}
    
    \normalsize{\bf Panel A: Aggregate Spending} & & \normalsize{\bf Panel B: Shares by Categories} \\
    {\includegraphics[height=1.8in,width=2.9in]{figures/graph_milspend_aggregate.png}} & & {\includegraphics[height=1.8in,width=2.9in]{figures/graph_composition_shares.png}} \\[0.1in]
    
    \end{tabular}
   \caption{Military Spending}\label{fig:share_comp}
\end{center}
    
    
    \footnotesize{\textit{Note. } The national level statistics are calculated by aggregating the microdata on military procurement contracts available from NARA and USA Spending. Spending is in real terms by deflating the nominal value of a contract by the CPI. The classification of the spending into the three categories is based on the Federal supply classification code. The statistics are calculated by using more than $20$ millions of new contracts or modifications of existing contracts.}


\end{figure}
\newpage

\begin{figure}[H]
\begin{center}
    \begin{tabular}[c]{ccc}
    
    \normalsize{\bf Panel A: Total Spending} & & \normalsize{\bf Panel B: Spending in Goods} \\
    {\includegraphics[height=1.5in,width=2.8in]{figures/map_cbsa_totspend.png}} & & {\includegraphics[height=1.5in,width=2.8in]{figures/map_cbsa_goodsspend.png}} \\[0.1in]
    
    \normalsize{\bf Panel C: Spending in Services} & & \normalsize{\bf Panel D: Spending in R\&D} \\
    {\includegraphics[height=1.5in,width=2.8in]{figures/map_cbsa_servicesspend.png}} & & {\includegraphics[height=1.5in,width=2.8in]{figures/map_cbsa_rdspend.png}} \\[0.1in]
    
    \multicolumn{3}{c}{\includegraphics[height=0.2in,width=3.8in]{figures/map_cbsa_legspend.png}} \\[0.1in]
      \end{tabular}
     \caption{Military Spending - Geographic Distribution}
\end{center}
   
    \footnotesize{\textit{Note. } The quartile to which a MSA belongs is assigned based on the average military spending in real terms that the MSA receives over the period $1966-2019$. The classification of the spending into the three categories is based on the Federal supply classification code. The maps only show the $296$ MSAs included in our sample.}
    \label{fig:map_spend}
\end{figure}
\newpage

\begin{figure}[H]
    \begin{center}
    \begin{tabular}[c]{ccc}
    
    \normalsize{\bf Panel A: Total Spending} & & \normalsize{\bf Panel B: Spending in Goods} \\
    {\includegraphics[height=1.5in,width=2.8in]{figures/graph_Rn_wages_lv_Rn_ms_lv.png}} & & {\includegraphics[height=1.5in,width=2.8in]{figures/graph_Rn_wages_lv_Rn_ms_goods_lv.png}} \\[0.1in]
    
    \normalsize{\bf Panel C: Spending in Services} & & \normalsize{\bf Panel D: Spending in R\&D} \\
    {\includegraphics[height=1.5in,width=2.8in]{figures/graph_Rn_wages_lv_Rn_ms_services_lv.png}} & & {\includegraphics[height=1.5in,width=2.8in]{figures/graph_Rn_wages_lv_Rn_ms_rd_lv.png}} \\[0.1in]
    
    
    
    \end{tabular}
        \caption{Fiscal Multipliers - Earnings}
    \end{center}
    

 
    \footnotesize{\textit{Note. } Panel A reports the estimates from equation \eqref{eq:fm_base}. The remaining panels the estimates from equation \eqref{eq:fm_comp}. We use the instrumental variable approach with the instrument calculated as in equation \eqref{eq:fm_iv}. The unit of observations are MSAs in different years. The balanced panel used for computing the estimates includes $296$ MSAs for the period $1980-2015$. The instrument is calculated using observations between $1966$ and $1980$. Standard errors are clustered at MSA-level. The shaded areas report the $90\%$ confidence intervals.}

    \label{fig:fm_earnings}
\end{figure}
\newpage

\begin{figure}[H]
    \begin{center}
    \begin{tabular}[c]{ccc}
    
    \normalsize{\bf Panel A: Total Spending} & & \normalsize{\bf Panel B: Spending in Goods} \\
    {\includegraphics[height=1.5in,width=2.8in]{figures/graph_Rn_emp_lv_Rn_ms_lv.png}} & & {\includegraphics[height=1.5in,width=2.8in]{figures/graph_Rn_emp_lv_Rn_ms_goods_lv.png}} \\[0.1in]
    
    \normalsize{\bf Panel C: Spending in Services} & & \normalsize{\bf Panel D: Spending in R\&D} \\
    {\includegraphics[height=1.5in,width=2.8in]{figures/graph_Rn_emp_lv_Rn_ms_services_lv.png}} & & {\includegraphics[height=1.5in,width=2.8in]{figures/graph_Rn_emp_lv_Rn_ms_rd_lv.png}} \\[0.1in]
    
    \end{tabular}
    \caption{Fiscal Multipliers - Employment}
    \end{center}
    
    \footnotesize{\textit{Note. } Panel A reports the estimates from equation \eqref{eq:fm_base}. The remaining panels the estimates from equation \eqref{eq:fm_comp}. We use the instrumental variable approach with the instrument calculated as in equation \eqref{eq:fm_iv}. The unit of observations are MSAs in different years. The balanced panel used for computing the estimates includes $296$ MSAs for the period $1980-2015$. The instrument is calculated using observations between $1966$ and $1980$. Standard errors are clustered at MSA-level. The shaded areas report the $90\%$ confidence intervals.}
    \label{fig:fm_employment}
\end{figure}
\newpage

\begin{figure}[H]
    \begin{center}
        \begin{tabular}[c]{c}
   
    {\includegraphics[height=2.5in,width=4.5in]{figures/graph_share_laborintensity.png}} \\[0.1in]

    
    \end{tabular}
    \caption{Share of Spending in Labor-intensive Industries}
    \end{center}
    
    
    \footnotesize{\textit{Note. } The classification of the spending into the three categories is based on the Federal supply classification code. The classification between low and high labor-intensive industries is based on the value added data collected from the BEA.}
    \label{fig:shlabint_comp}
\end{figure}
\newpage

\begin{figure}[ht]
    \begin{center}
    \begin{tabular}[c]{ccc}
    
    \normalsize{\bf Panel A: Total Spending} & & \normalsize{\bf Panel B: Spending in Goods} \\
    {\includegraphics[height=1.5in,width=2.8in]{figures/graph_Rn_wages_lv_spill100_Rn_ms_lv_spill100.png}} & & {\includegraphics[height=1.5in,width=2.8in]{figures/graph_Rn_wages_lv_spill100_Rn_ms_goods_lv_spill100.png}} \\[0.1in]
    
    \normalsize{\bf Panel C: Spending in Services} & & \normalsize{\bf Panel D: Spending in R\&D} \\
    {\includegraphics[height=1.5in,width=2.8in]{figures/graph_Rn_wages_lv_spill100_Rn_ms_services_lv_spill100.png}} & & {\includegraphics[height=1.5in,width=2.8in]{figures/graph_Rn_wages_lv_spill100_Rn_ms_rd_lv_spill100.png}} \\[0.1in]
    
    \end{tabular}
    \caption{``Outflow'' Effects - Earnings}
    \end{center}
    
    
    
    \footnotesize{\textit{Note. } Estimates are computed from  equation \eqref{eq:fm_spill}. We use the instrumental variable approach with the instrument calculated as in equation \eqref{eq:fmspill_iv}. The unit of observations are MSAs in different years. The balanced panel used for computing the estimates includes $284$ MSAs for the period $1980-2015$. The instrument is calculated using observations between $1966$ and $1980$. Standard errors are clustered at MSA-level. The shaded areas report the $90\%$ confidence intervals.}
    \label{fig:earn_spill}
\end{figure}
\newpage

\begin{figure}[ht]
    \begin{center}
        \begin{tabular}[c]{ccc}
    
    \normalsize{\bf Panel A: Total Spending} & & \normalsize{\bf Panel B: Spending in Goods} \\
    {\includegraphics[height=1.5in,width=2.8in]{figures/graph_Rn_consexp_lv_Rn_ms_lv.png}} & & {\includegraphics[height=1.5in,width=2.8in]{figures/graph_Rn_consexp_lv_Rn_ms_goods_lv.png}} \\[0.1in]
    
    \normalsize{\bf Panel C: Spending in Services} & & \normalsize{\bf Panel D: Spending in R\&D} \\
    {\includegraphics[height=1.5in,width=2.8in]{figures/graph_Rn_consexp_lv_Rn_ms_services_lv.png}} & & {\includegraphics[height=1.5in,width=2.8in]{figures/graph_Rn_consexp_lv_Rn_ms_rd_lv.png}} \\[0.1in]

    
    \end{tabular}
    \caption{Private Consumption Crowding-out Effect}
    \end{center}
    
    
    
    
    \footnotesize{\textit{Note. } The unit of observations are states in different years. The balanced panel used for computing the estimates includes $51$ states for the period $1998-2015$. The instrument is calculated using observations between $1966$ and $1980$. Standard errors are clustered at state-level. The shaded areas report the $90\%$ confidence intervals.}

    \label{fig:crowd_privcons}
\end{figure}
\newpage


\newpage
\end{document}


\begin{comment}
   
\begin{figure}[ht]
    \begin{center}
        \begin{tabular}[c]{ccc}
    
    \normalsize{\bf Panel A: Total Spending} & & \normalsize{\bf Panel B: Spending in Goods} \\
    {\includegraphics[height=1.5in,width=2.8in]{figures/graph_mpc_ms.png}} & & {\includegraphics[height=1.5in,width=2.8in]{figures/graph_mpc_goods.png}} \\[0.1in]
    
    \normalsize{\bf Panel C: Spending in Services} & & \normalsize{\bf Panel D: Spending in R\&D} \\
    {\includegraphics[height=1.5in,width=2.8in]{figures/graph_mpc_services.png}} & & {\includegraphics[height=1.5in,width=2.8in]{figures/graph_mpc_rd.png}} \\[0.1in]
    
    \multicolumn{3}{c}{\includegraphics[height=0.15in,width=2.8in]{figures/graph_mpc_legend.png}} \\[0.1in]

    
    \end{tabular}
    \end{center}
    
    
    
    \caption{Fiscal Multiplier by High- vs. Low-MPC States}
    
    \footnotesize{\textit{Note. } The unit of observations are states in different years. The balanced panel used for computing the estimates includes $27$ states for the period $1980-2015$. The instrument is calculated using observations between $1966$ and $1980$. Standard errors are clustered at state-level. The shaded areas report the $90\%$ confidence intervals.}

    \label{fig:mpc}
\end{figure}
\end{comment}
    