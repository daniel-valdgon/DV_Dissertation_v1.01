\documentclass[dv_diss_main.tex]{subfiles}

\begin{document}

\section{Mechanisms} 

This section investigates the potential mechanisms that explain why politically aligned municipalities show a slower increase in job creation despite receiving a disproportional increase in public revenues. 
I argue that public spending crowded out private-sector jobs by reallocating production factors (labor and capital) towards rent-seeking activities. 
The basic argument is that the additional revenues received by aligned municipalities are spent on contracts to the private sector that increase profits rather than the quantity of goods or services provided. The increase of profits is consistent with aligned politicians funneling private transfers to voters as strategy to increase their parties' re-election probabilities. 
This increase in rents attracts entrepreneurs and capital from non-rent seeking industries. This reallocation has an opportunity cost to the aggregate economy if non-rent seeking industries have a positive externality on the aggregate growth rate of total employment. The higher profits from reallocating jobs to the formal sector increase the likelihood that municipalities attract high productivity entrepreneurs who would otherwise have been employed in different industries. In turn, this increases investment and employment in the local economy. 

This channel has two implications: First, the effect should be weaker in economies where production factors did not reallocate because of the increase in the public sector demand. Second, the increase in spending on contracts to the private sector does not materialize into a proportional increase in production of goods or services to the public sector as part of the increased spending is diverted towards higher rents. 


\subsection{Reallocation and Rent Seeking}

\subsubsection{Reallocation Effects} 

A standard Keynesian macroeconomic framework predicts that increases in public spending lead to higher private employment when production factors are underutilized. In this case, the increase in public sector demand would not reallocate production factors from non-rent seeking industries, but it would put production factors into work that would otherwise not be utilized in the economy. On the contrary, if production factors are scarce, one would expect a higher opportunity cost from the increase in public sector demand that results from the reallocation of economic activity towards rent seeking activities. 

I perform several tests to argue that reallocation is one of the main reasons behind the slowdown in the private-sector economic growth. The first test evaluates whether the impact of politically motivated spending is different in expansion, when production factors are scarce, than in recessions, when production factors are idle. To measure expansions and recessions at the municipal level, I use the growth rate of formal employment during the three years before an election takes place. I define expansions (recessions) as those municipalities where the pre-electoral growth rate was above (below) the median pre-election employment growth rate.  Panel A of Table~\ref{tab:goodbad} shows the estimates over the expansions subsample, while Panel B shows the estimates over the recessions subsample. On one hand, the results of columns 1 and 2 of both panel A and B suggest that the amount of politically motivated spending is relatively similar in both subsamples. On the other hand, I find a substantial difference in the impact of alignment on employment depending on the economic cycle. In particular, columns 3 and 4 show that alignment reduces the employment growth rate by 13 percentage points during economic expansions, while the point estimate is halved, 6.7 percentage points, during  non-expansionary periods.  

A second test is to split the results between the tradeable and non-tradeable sectors.  Non-tradeable industries rely more on local demand. Therefore, it is expected that they benefit from increases in local public spending. This demand effect may partially offset the crowding-out effect that results from the reallocation of production factors towards rent seeking activities. On the other hand, tradeable industries, which rely less on local demand and depend on international markets, are expected to be fully affected by the crowding-out effects. 

Table~\ref{tab:tradable}  shows the effect of political alignment for tradeable and non-tradable industries. consistent with a competition for production factors mechanism, the tradable sector is the one that experiences the bulk of the negative effect of political alignment. The results suggest that the effect of alignment on employment is about six times larger in the tradeable sector. Column 2 shows that political alignment reduces the employment growth rate by 14.6 percentage points in the tradeable sector but only 2.3 percentage points in the non-tradeable sector.

The third test of this hypothesis is to evaluate whether the pre-election size of the industries that supply the public sector is important in explaining my results. I compute the pre-election share of private-sector jobs that work in industries that disproportionally supply the public sector. I use the input-output matrix of Mexico to separate the sectors between those that dis-proportionally supply the government and those with relatively low dependence on government demand. Based on this sector categorization, I split the municipalities into two groups, those with a high and low pre-election share of industries that supply the public sector. The idea is that the larger the share of government-dependent industries in an economy, the better the capacity of the local economy to accommodate to the increase in public spending without demanding a severe reallocation of workers from other highly productive industries. 

Panel A and B of Table~\ref{tab:gdsectors} show the effect of alignment on public spending and employment growth on local economies with a relatively high and low share of government-dependent (GD) sectors. The results of the effect of political alignment on spending are similar across both sets of municipalities, which indicates that central politicians do not consider the size of the GD sectors when allocating transfers. Specifically, I find that political alignment increases the spending growth rate by 9.9 percentage points in municipalities with a high share of GD jobs (Column 2 of Panel A) and 11.4 percentage points in municipalities with a relatively low share of GD jobs  (Column 4 of Panel B). Although both types of municipalities receive a similar spending shock, I find that the impact of alignment on the employment growth rate tends to be stronger (point estimate is 57\% stronger) in municipalities with a relatively low share of GD jobs. In particular, alignment reduces the employment growth rate by 11.8 (7.5) percentage points in municipalities with a low (high) share of GD sectors.  This provides suggestive evidence that the impact of alignment on employment is related to the relative size of the GD sector in a municipality; a relatively large GD sector may imply less competition for production factors with other productive sectors in the economy. 










\subsubsection{Rent Seeking} The main argument behind this mechanism is that the reallocation of production factors lead to rent seeking. By definition, the idea of rent seeking is that certain production factors obtain higher rents without providing equivalent value added in the goods and/or services that they produce. Providing direct evidence on rent seeking would require contract level data on the goods and services produced by public sector contractors, which is not available in Mexico for the study period.

However, I perform three indirect test that suggest that the contracts were directed to provide higher private rents that could increase political returns rather than using the spending to increase employment in sectors that provide higher externalities for the whole economy. Three facts confirm this story. First,
I do not observe a reduction in measures of total consumption. Table~\ref{tab:growth} reports the effect of political alignment on two different measures that capture total (public and private) consumption at a local level: night lights luminosity and electricity consumption. It is reassuring to see similar point estimates in both measures. Both points towards positive but non-statistically significant changes in aggregate consumption. This is consistent with the hypothesis that the disproportional resources from higher intergovernmental transfers are being allocated to households through means other than formal employment. 

A second argument suggesting that voters are better off is presented in Table~\ref{tab:lfs2}. It estimates the effect of alignment on the probability of being employed, unemployed, and part of the labor force. Each measure is computed using the population between 15 and 65 years of age. The decline in employment discussed in section 5 is decomposed into two types of transitions: from employment to out of the labor force (columns 5 and 6), and from employment to unemployment (columns 3 and 4). Panel A suggest that political alignment reduced labor force participation by 2.8-2.9 percentage points, but had no significant effect on unemployment. This implies that the observed negative effects of political alignment on total employment correspond to an increase in the labor force participation rather than an increase in the probability of being unemployed. This takeaway is consistent with interpreting the negative effect of alignment on employment as a slowdown in the net job creation rather than a destruction of jobs, as one would expect the destruction of jobs to generate a larger increase in the probability of being unemployed. 

Table~\ref{tab:votes} presents a third piece of evidence suggesting that money from intragovernmental transfers may have been used to generate private rents with political purposes. Specifically, this table displays the estimates of the effect of political alignment on the probability of winning the subsequent elections. Table~\ref{tab:votes} shows the probability of winning the next election (column 1 and 2) and two subsequent elections in a row (column 3 and 4). Although the estimates in panel A are not statistically significant, the consistency between the estimates is remarkable. In the case of panel B, I find a strong incumbency advantage for mayors who were initially politically aligned. Looking at panel B, I find that the probability of winning the next election is 40 percent (=13/31) higher for politically aligned municipalities. Also this municipalities duplicate the probability of winning two elections in a row (13/11).
\end{document}