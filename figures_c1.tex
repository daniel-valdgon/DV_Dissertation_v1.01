\documentclass[dv_diss_main.tex]{subfiles}





%---------------------------------------------%
%----------------Figures----------------------
%---------------------------------------------%



\newcommand{\rddplot}{
NOTE--This plot aggregate data into bins of half percentage points and estimate a third order polynomial regression between the running variable and the bins on each side of the cut-off. 
}


\newcommand{\rdd}{
NOTE--This table reports the estimates of political alignment from equation (2). The sample includes post electoral years of all municipalities with close elections during the period 1998-2003. The outcome variables are measure as a three year changes. Controls refers to state fixed effects, election-year fixed effects, and baseline political characteristics (incumbency status, previous political alignment, previous political party). Mean dep var refers to the sample average of the outcome variable for the non-aligned municipalities. 
}



\newcommand{\sharerev}{
Percentage of revenues is the sample average share of each source of revenue on total revenues for the non-aligned counterparts
}

\newcommand{\shareemp}{
Percentage of employment is the sample average share of each sector on total employment for the non-aligned counterparts
}

\newcommand{\event}{
NOTE--The figure plots the coefficients obtained from the estimation of equation (3) discussed in Section 4. The sample includes all municipalities with close elections during the period 1998-2003. The unit of observation is  the municipal-election pair, for each pair I follow the outcome measures in [-4 +4] years window. The outcome variables are measure in inverse hyperbolic sine points. The tick(thin) lines are 90\%(95\%) confidence intervals. The specification controls by municipality-election and election-year fixed effects. 
}

%inverse hyperbolic sine (IHS) transformation


\newcommand{\stars}{
Standard errors clustered at municipality level.  *** p<0.01, ** p<0.05, * p<0.1.
}

\newcommand{\census}{
}

\newcommand{\household}{
}



\begin{document}

\begin{figure}[h]
	\begin{center}
			\includegraphics[width=0.8\linewidth]{figures/2_Unexpected.png}
			\caption{The Economic Size of Ramo-33}\label{fig:unexpected}
	\end{center}
	\vspace{0.5em}
	\begin{figurenotes}
    \footnotesize	
	\textit{Note. }The figure show the population weighted average of revenues as a share of local GDP in 1998. Ramo-33 for the year after 1998 and PRONASOL for years before 1998.
    \end{figurenotes}
\end{figure}

\newpage

\begin{figure}[h]
	\begin{center}
			\includegraphics[width=0.8\linewidth]{figures/2_Panel_a_fig_sizable.png}
			\caption{The Economic Size of Ramo-33}\label{fig:iceconsizer33}
	\end{center}
	\vspace{0.5em}
	\begin{figurenotes}
    \footnotesize	
	\textit{Note. }The figure show the population weighted average of revenues as a share of local GDP in 1998. Ramo-33 for the year after 1998 and PRONASOL for years before 1998.
	
    \end{figurenotes}
\end{figure}

\newpage

\begin{figure}[h]
    \begin{center}
			\includegraphics[width=0.8\linewidth]{figures/Panel_b_fig_sizable.png}
            \caption{The Size of Ramo-33 in the Local Public Finances}\label{fig:icbudgsize33}
			%\subcaption{\footnotesize Size of pre-policy GDP 1998s}    
    \end{center}
	\vspace{0.5em}
	\begin{figurenotes}
    \footnotesize	
	\textit{Note. }The figure show the population weighted average of infrastructure investment as share of total  spending or revenues. Ramo-33 for the year after 1998 and PRONASOL for years before 1998.
	\end{figurenotes}
\end{figure}

\newpage

\begin{figure}[h]
	\begin{center}
			\includegraphics[width=0.8\linewidth]{figures/Pol_econ.png}
			%\subcaption{\footnotesize Size of pre-policy GDP 1998s}
			\caption{The Non-compliance with Ramo-33 \textit{de jure} Allocation}\label{fig:pol}
	\end{center}
	\vspace{0.5em}
	\begin{figurenotes}
    \footnotesize	
	\textit{Note. }The figure show the population weighted average or municipalities receiving transfers and the coeficient of variation (standard deviation / mean) of the distribution of yearly growth rates of Ramo-33 for the year after 1998 and PRONASOL for years before 1998.
	\end{figurenotes}
\end{figure}

%\subsection{Validity of the Research Design}

\newpage

\begin{figure}[h]
	\begin{center}
			\includegraphics[width=0.8\linewidth]{figures/HistogramGraph.png}
			%\subcaption{\footnotesize Size of pre-policy GDP 1998s}
			\caption{Distribution of Mayoral Elections Along the Vote Margin $V_{m,e}$ 1998-2003}\label{fig:hist}
	\end{center}
	\vspace{0.5em}
	\begin{figurenotes}
	\footnotesize	
     \textit{Note. }This figure shows the histogram of the governor's party vote margin on the mayoral elections used in our estimates (1998-2003).
	\end{figurenotes}
\end{figure}

\newpage

\begin{figure}[h!]
    \begin{center}
			\includegraphics[width=0.8\linewidth]{figures/McCraryGraph.png}
			%\subcaption{\footnotesize Size of pre-policy GDP 1998s}
			\caption{McCrary Density Estimates of the Vote Margin $V_{m,e}$ 1998-2004}\label{fig:mccrary}
    \end{center}
	\vspace{0.5em}
	\begin{figurenotes}
    \footnotesize	
     \textit{Note. }This figure shows a estimate of the density of governor's party vote margin on mayoral elections used in our estimates (1998-2003). Each bubble groups all elections that took place in half percentage points spread bins. The dark lie is the point estimate of the density function and the light lines a 95\% confidence interval.
	\end{figurenotes}
\end{figure}

\newpage

\begin{figure}[h]
	\begin{center}
			\includegraphics[width=0.8\linewidth]{figures/Balance_1998_2003.png}
			%\subcaption{\footnotesize Size of pre-policy GDP 1998s}
			\caption{Balance on Predetermined (1990s) Covariates}\label{fig:baselinebalance}
	\end{center}
	\vspace{0.5em}
	\begin{figurenotes}
    \footnotesize	
	\textit{Note. }The reported coefficients come from separated regressions that estimated the causal effect of political alignment on predetermined covariates using a variant of equation (2) that only controls linearly for the running variable on either side of the cut-off.  When a municipality has more than one close election I consider only first reported election from the studied period (1998-2003).  All reported outcomes are measure circa 1990 using populatin and economic census. 
	\end{figurenotes}
\end{figure}

\newpage

\begin{figure}[h]
    \begin{center}
			\includegraphics[width=0.8\linewidth]{figures/map00s_v2.png}
			%\subcaption{\footnotesize Size of pre-policy GDP 1998s} 
			\caption{Spatial Distribution of Political Alignment in Close Elections}\label{fig:map}
    \end{center}
	\vspace{0.5em}
	\begin{figurenotes}
    \footnotesize	
	\textit{Note. }The figure maps municipalities ruled by aligned and opposition parties for the sample used to obtain our main estimates (see section 4), where elections were decided by less than 5 percentage points.
	\end{figurenotes}
\end{figure}

\newpage

\begin{figure}[h]
	\begin{center}
	      % include first image
    	\stackunder{\includegraphics[width=.7\linewidth]{figures/3_rdplot_pf_2_1_0_as_main.png}}{\\ \textbf{A. Post-election Period}}
    	\\
        \stackunder{\includegraphics[width=.7\linewidth]{figures/3_rdplot_pf_2_1_0_as_placebo.png}}{ \\ \textbf{B. Pre-election Period}}
        \caption{Growth Rate of Earmarked Transfers and Governor's Vote Margin}\label{fig:rddplotearmarks}
	\end{center}
    \vspace{0.5em}
    \begin{figurenotes}
    \footnotesize	
	\textit{Note. }The lines correspond to the estimates of the outcome variable  (defined in the y axis) on the a third order polynomial of the governor's vote margin. Each circle shows the mean of the observations that correspond to the specific half percentage point bin. 
	\end{figurenotes}

\end{figure}

\newpage

\begin{figure}[h]
	\begin{center}

  % include first image
    \stackunder{\includegraphics[width=.7\linewidth]{figures/3_rdplot_pf_1_0_0_as_main.png}}{\\ \textbf{A. Post-election Period}}
    \\
    \stackunder{\includegraphics[width=.7\linewidth]{figures/3_rdplot_pf_1_0_0_as_placebo.png}}{\\ \textbf{B. Pre-election Period}}
    \caption{Growth Rate of Total Spending and Governor's Vote Margin}\label{fig:rddplottotspending}
	\end{center}
	\vspace{0.5em}
    \begin{figurenotes}
    \footnotesize	
	\textit{Note. }The lines correspond to the estimates of the outcome variable  (defined in the y axis) on the a third order polynomial of the governor's vote margin. Each circle shows the mean of the observations that correspond to the specific half percentage point bin. 
	\end{figurenotes}

\end{figure}

\newpage

\begin{figure}[h]
	\begin{center}
	    \includegraphics[width=0.8\linewidth]{figures/sdevent_itt_rdd_pf_1_0_0_ln_bd_11_th1.png}
	    \caption{Event Study Total Spending After Political Alignment}
	%\subcaption{\footnotesize Size of pre-policy GDP 1998s}
	\end{center}
	 
	 \vspace{0.5em}
    \begin{figurenotes}
    {\footnotesize	
	\textit{Note. }This plot aggregate data into bins of half percentage points and estimate a third order polynomial regression between the running variable and the bins on each side of the cut-off. }
	\end{figurenotes}

\end{figure}

\newpage

\begin{figure}[h]
	\begin{center}
  % include first image
        \stackunder{\includegraphics[width=.7\linewidth]{figures/3_rdplot_f_emp_main_.png}}{ \\   \textbf{A. Post-election Period}}
        \\
        \stackunder{\includegraphics[width=.7\linewidth]{figures/3_rdplot_f_emp_placebo_.png}}{ \\ \textbf{B. Pre-election Period}}
        \caption{Growth Rate of Formal Employment and Governor's Vote Margin}\label{fig:rddplotformalemp}
	\end{center}
	\vspace{0.5em}
    \begin{figurenotes}
    {\footnotesize	
    \textit{Note. }The lines correspond to the estimates of the outcome variable  (defined in the y axis) on the a third order polynomial of the governor's vote margin. Each circle shows the mean of the observations that correspond to the specific half percentage point bin. }
	\end{figurenotes}

\end{figure}

\newpage

\begin{figure}[h]
	\begin{center}
    	\includegraphics[width=0.8\linewidth]{figures/sdevent_itt_rdd_f_emp_bd_11_th1.png}
    	%\subcaption{\footnotesize Size of pre-policy GDP 1998s} 
    	\caption{Event Study Formal Employment After Political Alignment }
	\end{center}
	\vspace{0.5em}
    \begin{figurenotes}
    {\footnotesize	
	\textit{Note. }This plot aggregate data into bins of half percentage points and estimate a third order polynomial regression between the running variable and the bins on each side of the cut-off. }
	\end{figurenotes}

\end{figure}

\newpage

\begin{figure}[h]
	\begin{center}
	
  % include first image
        \stackunder{\includegraphics[width=.7\linewidth]{figures/3_rdplot_wages_main.png}}{ \\ \textbf{A. Post-election Period}}
        \\
        \stackunder{\includegraphics[width=.7\linewidth]{figures/3_rdplot_wages_main.png}}{ \\ \textbf{B. Pre-election Period}}
        \caption{Growth Rate of Wages and Governor's Vote Margin}\label{fig:rddplotwages}
	\end{center}
	\vspace{0.5em}
    \begin{figurenotes}
    {\footnotesize	
	\textit{Note. }The lines correspond to the estimates of the outcome variable  (defined in the y axis) on the a third order polynomial of the governor's vote margin. Each circle shows the mean of the observations that correspond to the specific half percentage point bin. }
	\end{figurenotes}

\end{figure}

\newpage

\begin{figure}[h]
    \begin{center}
    \begin{tikzpicture}[scale=0.65]

    % years' line
    
    \draw (0,1)(0,2);

    \draw [gray] (0,0)--(17,0);
    
    \draw (1,0)  node[above=3pt] {$ 1998 $};
    \draw (4,0)  node[above=3pt] {$ 2001 $};
    \draw (7,0)  node[above=3pt] {$ 2004 $};
    \draw (10,0) node[above=3pt] {$ 2007 $}; 
    \draw (13,0) node[above=3pt] {$ 2010 $};
    \draw (16,0) node[above=3pt] {$ 2013 $};
    
    \foreach \x in {0,1,2,3,4,5,6,7,8,9,10,11,12,13,14,15,16,17}
      \draw [gray] (\x,0) -- (\x,-0.3);;
      
    % Nuevo León
    \draw (-1,-1.5) node[right] {Nuevo León State};

    \draw [ChadGreen] (0,-2 )--(13,-2);
    \draw [blue]  (13,-2)--(17,-2);
    
    \foreach \x in {1,7,13}
      \node at (\x,-2) [rectangle, draw, fill=white!90, white] {$ $};
    \foreach \x in {1,4,7,10,13,16}
      \draw [dashed] (\x,-2) -- (\x,-2.5);
      
    \foreach \x in {1,4,10,13}
      \node at (\x,-2.5) [rectangle, draw, fill=white, ChadGreen, scale=0.5] {$ $};
    \foreach \x in {7,16}
      \node at (\x,-2.5) [rectangle, draw, fill=white, blue, scale=0.5] {$ $};

    % Puebla 
    
    \draw (-1,-3.5) node[right] {Puebla State};
    
    \draw [ChadGreen]  (0,-4 )--(6,-4);
    \draw [blue] (6,-4)--(17,-4);
    
    \foreach \x in {0,6,12}
      \node at (\x,-4) [rectangle, draw, fill=white!90, white] {$ $};
    \foreach \x in {0,3,6,9,12,15}
      \draw [dashed] (\x,-4) -- (\x,-4.5);
      
    \foreach \x in {0,3,6,15}
      \node at (\x,-4.5) [rectangle, draw, fill=white, ChadGreen, scale=0.5] {$ $};
    \foreach \x in {9,12}
      \node at (\x,-4.5) [rectangle, draw, fill=white, blue, scale=0.5]{$ $};
      
    % Legend 
    
    \draw [dashed] (3,-8) -- (3,-7) node[above=3pt] {municipal elections};
    \node at (3,-8) [rectangle, draw, fill=white, scale=0.5] {$ $};
    
    \draw [dashed] (10,-8) -- (10,-7) node[above=3pt] {State term limit};
    \draw (9,-7) -- (11,-7);
    %\node at (10,-7) [rectangle, draw, fill=white!90, white] {$ $};
    
    \node at (15,-7)   [text=blue] {PAN};
    \node at (15,-7.6) [text=ChadGreen] {PRI};

\end{tikzpicture}
\caption{Staggered Elections and Variation in Political Alignment}\label{fig:staggered}
    \end{center}
\vspace{0.5em}
\begin{figurenotes}
    \footnotesize

	\textit{Note. }This figure is an example of two states who have election cycles at different calendar years. The horizontal bar represent the governor's term limit (six years), while the space between squares is the mayoral term limit (three years). The vertical dashed represent election dates. The colors of the bar and squares represent the parties who won of each state and local election.
	\end{figurenotes}
\end{figure}

\end{document}