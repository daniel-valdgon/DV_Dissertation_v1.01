\documentclass[dv_diss_main.tex]{subfiles}

\begin{document}

\subsection{Ruling Out Public Employment and Infrastructure Construction}

\subsubsection{Public Employment}  One potential crowding-out mechanism is that higher public spending leads to an increase in public sector employment, which may deter private sector employment growth. To test this hypothesis, I construct a measure of total jobs and an aggregate wage bill for the public sector.

To measure public sector jobs at the municipality-year level, I combine administrative data from the institute of social security institute for state employees (ISSTE), and the institute of social security for private employees (IMSS).\footnote{ About 92\% of public workers are affiliated to ISSTE. I identify public workers affiliated with IMSS by taking advantage that IMSS data allows me to observe employment at the sector level. I classify a worker in IMSS as a public worker when it works in sector 94 or 99 defined by the 2-digit NAIC code.} It is important to clarify that this employment measure does not capture contractors hired by the public sector, as these individuals are counted as private-sector workers. I also use the public finance data to measure total spending on salaries and work benefits for public employees, which I define as the total wage bill of public sector employees, which represents about 30\% of total public spending. 

Table~\ref{tab:public} suggests that political alignment is not systematically associated with higher public employment. In particular, Columns 1 and 3 show positive and mildly statistically significant effects of alignment on public-sector employment and wage bills. However, these effects are unstable and change abruptly in columns 2 and 4, with the inclusion of the controls defined in equation \eqref{eq:didrdd}. The lack of coefficient stability between the estimates with and without controls is not a characteristic of our main results (Section \ref{MainResults}). This implies that the variation between public employment and alignment is not strongly correlated with the increase in the public sector.  Another aspect to highlight from the table is that the R-squared of the regression on public employment with controls is relatively high (0.9). This fact is consistent with the low turnover of the public sector in Mexico and may be why I do not find changes in public sector employment because of political alignment.

\subsubsection{Infrastructure Spending} 
An increase of infrastructure projects may be a potential mechanism behind the observed slowdown on local employment.
A recent literature review by \cite{ramey2020macroeconomic} concludes that the short-term effects of infrastructure spending on employment are either negative or zero. 
Two suggestive explanations for this negative effect are a \textit{disruption effect} and a \textit{delay effect}. The disruption effect refers to the fact that, during the construction phase, infrastructure projects can increase traffic or even reduce sales in specific areas like retail or tourism. 
 The delay effect suggests that agents may decide to delay any private investment until after an infrastructure project is built. They find it optimal to delay investment because the returns to private capital will be higher in the future once the stock of public capital increases. I test this hypothesis from several angles in table~\ref{tab:infra2}, \ref{tab:cons_jobs}, and \ref{tab:infrastructure}. 

Table~\ref{tab:infra2} confirms that the disproportional resources received by aligned municipalities led to an increase in public investment. In particular, column 2 of panel A suggests that political alignment increases the growth rate of infrastructure spending by 40 percentage points. This effect is relatively large compared to the growth rate of non-aligned municipalities (120 percent) and the point estimate of political alignment on total spending (12 percentage points).\footnote{See results in Table~\ref{tab:2revenues}. } This result could be interpreted in two ways; first, a mechanical effect of higher compliance with the earmarks' spending rules, which assigns spending to infrastructure projects.\footnote{Ramo-33 has two components FISM and FORTAMUN, the former was earmark to infrastructure projects, See section \ref{InstCont} for more details.}  Second, it has been argued by \cite{robinson2005white} that politicians may prefer to build infrastructure, even white elephants, to signal their power and obtain higher electoral returns. 

However, the increase in infrastructure spending is insufficient to argue that the rise in infrastructure projects explains the slowdown in private-sector jobs. As is stated by \cite{garin2019putting} and \cite{ramey2020macroeconomic}, one should observe either higher inputs used by infrastructure projects, i.e., construction jobs, or higher outputs that result from infrastructure projects, i.e., increases in the stock of public capital. 

Table~\ref{tab:cons_jobs} evaluates whether politically aligned municipalities experienced an increase in construction jobs. Columns 1 and 2 explore the effects of political alignment on the number of construction jobs using the social security records, while Columns 3 and 4 explore the results on wages.  The results in Panel A do not show a consistent result. The more conservative interpretation is that alignment does not have a statistically significant effect on jobs or wages in the construction sector. However, when looking at the results in Panel B, the point estimates for employment turn to be negative and statistically significant. The higher spending on infrastructure with missing construction jobs may be suggestive evidence that corruption is taking place. I will test this interpretation with data on corruption in the next subsection. However, the fact that construction jobs are not positive suggests that neither the disruption nor the delay effect explains my results.\footnote{Since about 85\% of the workers on infrastructure are informal, I complement this analysis using the subsample of municipalities present in the household surveys. Results suggest that political alignment increases the probability of working in the construction sector as informal workers. However, the estimates are not statistically significant (results available upon request).}

Finally, Table~\ref{tab:infrastructure} examines the effect of alignment on long-term (1995-2010) changes in distinct public infrastructure measures, the proportion of households with access to electricity, water, and sewerage.  Looking at the long term allows me to rule out that I cannot see results because of the expected delays of construction projects. The results suggest relatively small and not statistically significant infrastructure improvements in politically aligned places, which is in line with no higher construction jobs taking place.
\end{document}