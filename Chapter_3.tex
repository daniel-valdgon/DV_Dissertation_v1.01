\documentclass[dv_diss_main.tex]{subfiles}

\begin{document}


%\begin{document}
\section{Introduction}
\label{sec:into}

An one dollar increase in local public spending lead to such an increase in output, earnings and employment that suggest that local multipliers crowd-in private economic activity~\cite{chodorow2019geographic}. The consensus regarding the direction of fiscal policy\footnote{The two exceptions in this literature is \cite{Dupor2017} and \cite{cohen2011powerful}, the former finds a multiplier lower than one while the later finds that politically-motivated spending shocks lead to reductions in aggregate output en employment.} contrast with the relatively large heterogeneity across the estimates, which demands an explanation. In response to that, a nascent literature highlights the role of local economic characteristics on explaining the large heterogeneity observed across the estimates. This literature highlights how the economic cycle~\citep{cohen2011powerful, serrato2016estimating,buchheim2020job}, trade openness\citep{corbi2019regional}, households and firms liquidity constraints~\citep{Demyanyk2019,auerbach2020effects, bruckner2014local}, population demographics~\citep{basso2021young} and firm-size distribution \citep{Juarros2021} amplify the response of local economic activity to public spending. However, much less is known about the role of spending composition, what is actually purchased by the government, on the size of the fiscal multiplier.\footnote{The discussion on the spending composition has been mostly theoretical and focus on the comparison of government consumption vs government investment \citep{Boehm2020,ramey2020macroeconomic}. \cite{garin2019putting} and \cite{buchheim2017employment} are the only two empirical studies that focus on estimating the impact of local infrastructure investment on isolation. However, they do not test other types of spending simultaneously.} 

% This paper propose that the effect of the fiscal policy on the real economy not only depends on how much is spend but also in the composition of that spending. 

This paper ask whether the size of the cross sectional employment and earnings multipliers depends on the product that is being purchase by the government? To do so, we build a panel of military spending at the product-MSA level and estimate the effect of each public spending composition on MSA employment and earnings. We find that the the goods and services that are purchased matter to explain the differences in the fiscal multiplier. Products that are produced by more labor intensive technologies lead to higher cross sectional  employment and earnings multipliers. 
%behind the government purchase is directed matters to explain the heterogeneity of the fiscal multiplier. If that is the case, what sector specific characteristics may explain such heterogeneity? 

The scarce evidence on the how different government spending composition affects the size of the cross-sectional fiscal multiplier is explained by the lack of a granular data on public spending that can be implemented in a research design that exploits quasi-random variation on each of the different components of public spending. The bulk of research on cross sectional multipliers have use variation in public spending coming from either the American Recovery and Reinvestment Act (ARRA) \citep{chodorow2012does,wilson2012fiscal,conley2013american,dupor20162009} or changes in spending made by the Department of Defense (DoD)~\citep{Nakamura2014,dupor2017local,Demyanyk2019,Auerbach2020}. Both type of studies have not differentiated how does their estimate of the fiscal multiplier change with different types of spending by the lack of contract level data that allow to disentangle the type of good in which each dollar is spent.

To circumvent this problem, this paper harmonize a rich dataset based on U.S. Department of Defense contracts that allow us to track a wide variety of details about each public dollar spent by this institution during the period of 1966-2019. In particular we can distinguish many characteristics of each contract: i) name and unique identifier of the firms who receive the contract, ii) type of product (good or service) contracted (e.g. missiles, R\&D, office supplies, or cleaning services for military bases), iii) geographic location of the primary contractor and where the majority of the work has been performed, iv) duration of the contract, v) the industry classification by 4-digits NAICS of the firm who has received the contract. 

Our data set on public spending satisfies the two necessary conditions to answer our research question. First it allow us to identify, at the product level, the composition of government purchases that take place in a specific MSA at a specific time. We exploit this product level variation to compute aggregate spending measures at the MSA level for three spending categories: goods, services and R\&D.\footnote{The specific division of the spending categories is motivated by the guidelines of the procurement office of the Department of Defense. This categories are not only stable over time but also allow us to have a relatively stable trends in public spending at the local level. See examples of what is included in each category in the Annex~\ref{tab:cat_list}.} Second, it allow us to nest this product level variation into a well known cross-sectional research design, proposed originally by~\cite{Nakamura2014}, that exploits  the heterogeneous sensitivity of MSA’ military procurement to an increase in (aggregate) federal military spending.\footnote{Although this research design has become the gold standard in the subsequent in the literature of cross-sectional fiscal  multipliers \citep{Auerbach2019,Demyanyk2019,basso2021young,dupor2017local,Juarros2021} it is not free of identification threats that may bias our coefficient. We will describe them in depth in our identification section.}

We combine the spending dataset with data on employment and earnings from the Bureau of Labor statistics. These are the two measures that are available at the MSA level for the period study. To match the concept of the fiscal multiplier, one would like to estimate the impact of spending on GDP. However, MSA GDP is not available for the years before 2001, and as a response all the studies that focus on MSA or county level data study estimate local employment and earning multipliers~\citep{suarez2016estimating,Auerbach2019,dupor2017local,Demyanyk2019}. These studies have shown consistency between the estimates of output and earnings/employment multipliers. Albeit the multipliers found in MSA and county level research designs tend to be smaller than the ones found for state-level designs. 

To identify the effect of the different public spending categories on economic activity we implement an instrumental variables research design proposed by \cite{Nakamura2014}. This strategy exploits the heterogeneous sensitivity of MSA' military procurement contracts to changes in nation-wide military buildups and drawdowns between 1980 and 2019. This heterogeneous sensitivity measures the "comparative advantage" of a specific MSA to receive contracts from the DoD, and it is measure as the interaction of nationwide spending and the share of public spending received by the MSA during the period of 1966-1980. The identifying assumption is that conditional on time and MSA and time fixed effects, contemporaneous military buildups are not influence by persistent or contemporaneous economic shocks that may affect our outcomes of interest. 

%This is similar to a Bartik instrument, and it has been scrutinized recently regarding the endogeneity assumptions. One can assume that endogeneity comes from either the na
%The identifying assumption is that changes in national defense spending for a specific category are not responding to local ecoomic shocks of MSAs. is independent of local economic shocks that affect the are independent of past economic shocks that affected local shocks that affects our outcomes of interest
%shocks to the local economies of locations with existing military installations conditional on county and year fixed effects. 

We study the effects of different spending categories on economic activity in three steps. First, we test if our assembled data replicate previous estimates on the cross-sectional multiplier literature. Our benchmark specification focus on studying how the two years growth in economic activity is affected by a two year growth in the fiscal multiplier. We find that an one dollar increase in public spending increase local earnings by 0.51 dollars. Also we find that the increase of public spending by 1 percent of aggregate earnings increase employment by 0.28 percent. This estimates are close the previous MSA- level estimates found in the literature \citep{Auerbach2019, Demyanyk2019}, but as it was mentioned before, they are lower than the estimates obtained from state-level designs \citep{Nakamura2014}. The economic intuition of why state-level multipliers are larger than MSA-level multipliers is that the former, suffer from lower measurement error and it is less affected by the presence of geographical spillovers. 

%by have a reltively smaller biased \cite{} and \cite{} that MSA-level multipliers are lower than state-level multipliers because of measurement error in the geographic allocation of spending or because of the presence of spatial spillovers. It is well known, that measurement error   created a bias towards zero in our estimates. Moreover, the literature have identify that public spending shocks have positive spillovers in nearby localities, which also implies that spillover will lead to a downward biases.\footnote{ A positive geographic spillovers imply demand spillovers related to final consumption or input-output linkages are larger than negative spillovers related to reallocation of productions factors (e.g., immigration of workers from non-treated geographies)} Since both source of bias suggest that we may suffer from a downward bias one should consider our estimates as a lower bound of the effects of the employment an earnings multipliers. 

Second, we estimate the effect of each category of spending on employment and output. We find that the two-year employment and earnings multipliers are lower when government purchase goods compared to either purchase of services or R\&D activities. In particular, the employment multiplier of goods (i.e. 0.09) is about one fifth of the value of the employment multiplier for spending on R\&D (i.e. 0.57) and about one third of the value of for the employment multiplier on services (i.e. 0.33). The difference between goods and the other spending categories (services and R&D) are larger when analyzing earnings multipliers which may suggest that part of the demand shock translate into increases in factor prices.   

In the third part of the paper, we investigate which are the characteristics of the spending categories that can rationalize the differences on employment and earnings multipliers. We argue that a important factor that may explain this different estimates is that goods are mostly produce by the manufacturing sector, which is a capital intensive sector, while both the services and R\&D activities are produce with labor intensive technology. 
To test this hypothesis we assign to each product a specific labor intensity, measured as the ratio between the payroll and value added at the 2-digit NAICS level. Then we computed aggregate measures of spending at the category-labor intensity level for each MSA-year. The results confirm that government purchases of products made by economic sector with high labor intensity (a labor intensity above the median) lead to higher employment and earnings multipliers

\textbf{Related literature} This paper contributes to three strands of the literature. First, it contributes to the literature on the employment and earning multipliers at granular geographic aggregation like MSA or county level in US~\citep{suarez2016estimating,Demyanyk2019,Auerbach2019,Auerbach2020,Juarros2021}. Most of this literature have used data on military spending from 2001 onward. Our paper built a dataset that goes back to 1966 and therefore it serves the purpose of given external validity to the current estimates that are estimated in the context where local labor market suffer from two important shocks that have been documented by the literature: The great recession~\citep{yagan2019employment} and the import-competition shock due to the entrance of China to the WTO~\citep{autor2016importing}. 

Second, it contributes to the literature that study the heterogeneity of fiscal policy. A recent literature have focus on explaining how characteristics of the local economies where the stimulus take place matter for the size of the cross sectional fiscal multiplier.  This research suggest that the state of the local economy (e.g. recessions vs expansions) ~\citep{cohen2011powerful, serrato2016estimating,buchheim2020job}, the level of household and firm liquidity constraints~\citep{Demyanyk2019,auerbach2020effects, bruckner2014local}, the demographic distribution of the population~\citep{basso2021young}, the share of small and young firms~\citep{Juarros2021}, and the trade openness\citep{corbi2019regional} lead to important differences in the size of the employment and earnings multiplier. This paper shifts the attention from the characteristics of the local economy towards characteristics of the fiscal policy itself. In particular, this paper focus on the type of products (goods and services) that are demanded by the government's purchases. A notable exception, is the nascent literature that studies the size of infrastructure multipliers. The theoretical \citep{Boehm2020, ramey2020macroeconomic} and empirical studies \citep{buchheim2017employment,garin2019putting} in this literature suggest that the fiscal multipliers of government investment are different to fiscal multipliers of government consumption. However this literature never put government consumption and investment in a horse race and therefore can not estimate the difference between both type of fiscal multipliers. This is our particular contribution, namely: we put both estimates in a horse-race causal estimation.

Our study is closely related to a third strand of the literature that focus on how sector heterogeneity matters for aggregate outcomes. \cite{vom2022investment} and \cite{bouakez2020government} use a general equilibrium model with network effects to show that the economic sector where a demand/fiscal shock originates matters for its aggregate impact.\cite{vom2022investment} argues that this is explained by secoral heterogeneity on propensity to invest while \cite{bouakez2020government} suggest that network centrality in a input-output economy is the main amplification mechanism. Alonso, which is the closest paper to ours, use a calibrated model and reduced form estimates of the elasticity of consumption to the decline in jobs due to the main recession to argue that the sector where households consume is important for the aggregate consequences of any demand/fito show that labor intensity is a key factor that can amplify the response of economic activity to fiscal policy. Our main difference with respect to this set of  studies is that we study empirically the effect of the composition of government purchases on economic activity rather than doing it trough a quantitative general equilibrium model. We see our studies as complementary with the paper of Alonso, who reach to its conclusion about fiscal policy without using data on fiscal spending.  





\iffalse
In the fiscal multiplier literature, government spending is often thought as a homogeneous good \citep{}. Only recently, a few studies have explored some aspects of the heterogeneity in the fiscal policy. \cite{Cox2021} describe the heterogeneous characteristics of government spending. \cite{Auclert2018} and \cite{Hagedorn2019} study the importance of household heterogeneity on the transmission of fiscal policy. \cite{Dupor2021} investigate the effect of government spending on different types of consumption. Still, very little is known about the effect of heterogeneous government spending on the fiscal multiplier. This paper aims to fill this gap in the literature.

Fiscal Multiplier Literature: 

Heterogeneous Government Spending:

\fi



%%%%%%%%%%%%%%%%%%%%%%%%%%%%%%%%%%%%%%%%%%%%%%%%%%%%%%%%%%%%%%%%%%%%%%%%%%%%%%%%%%%%%%%%%%%%%%%%%%%%%%%%%%
%%%%%%%%%%%%%%%%%%%%%%%%%%%%%%%%%%%%%%%%%%% EMPIRICAL STRATEGY  %%%%%%%%%%%%%%%%%%%%%%%%%%%%%%%%%%%%%%%%%%
%%%%%%%%%%%%%%%%%%%%%%%%%%%%%%%%%%%%%%%%%%%%%%%%%%%%%%%%%%%%%%%%%%%%%%%%%%%%%%%%%%%%%%%%%%%%%%%%%%%%%%%%%%

\section{Empirical Strategy}
\label{sec:emp_strategy}

\subsection{Specification}
Our empirical strategy builds on the works of \cite{Nakamura2014}, \cite{Dupor2017}, and \cite{Auerbach2020}. We exploit the time and MSA variation in military spending to
quantify the fiscal multipliers. We estimate the following empirical specification:

\begin{equation}
    \frac{v_{l,t+k} - v_{l,t-1}}{v_{l,t-1}} = \beta^k \frac{G_{l,t+k}-G_{l,t-1}}{Y_{l,t-1}} + \alpha_l^k + \delta_{t+k} + \varepsilon_{l,t+k}
    \label{eq:fm_base}
\end{equation}
where $v$ is a generic variable of interest in location $l$ at horizon $t+k$ with $k = 0, 
\dots, 4$. We estimate each $\beta^k$ in a separate regression regression; $G$ is the total military procurement spending; $Y$ is the deflator for the normalization of military spending, which due to data availability is aggregate earnings; the set of locality fixed effects, $\alpha_l$, absorb for location-specific trends; the time fixed effects, $\delta_{t+k}$, account for any mechanical correlation between secular trends in military spending and local economic outcomes; and $\varepsilon_{l,t+k}$ is the error term. As it is likely that the error terms are correlated over time, we cluster standard errors at the location level.\footnote{ \cite{Auerbach2020} makes a case to cluster standard error at the state level. Our results (available upon request) remain unaffected by adopting their clustering decision.} The coefficient $\beta^k$ quantifies the the employment and earnings multipliers. Since we normalize total spending by total earnings, the earnings multipliers correspond to the dollar amount of aggregate earnings produced by an dollar increase in government spending. This interpretation change for the case of the employment multipliers because government spending is not normalize by employment. In the case of the employment multiplier the interpretation is the change in the growth rate of employment produced by a 1 percentage point increase in government spending, measured as percentage of aggregate earnings. 

We augment the specification in equation \eqref{eq:fm_base} by examining the heterogeneous effects of government spending by its components. In this respect, we estimate the specification \eqref{eq:fm_comp}:
\begin{equation}
    \frac{v_{l,t+k} - v_{l,t-1}}{v_{l,t-1}} = \beta_{g}^k \frac{G^{g}_{l,t+k}-G^{g}_{l,t-1}}{Y_{l,t-1}} + \beta_{s}^k \frac{G^{s}_{l,t+k}-G^{s}_{l,t-1}}{Y_{l,t-1}} + \beta_{rd}^k \frac{G^{rd}_{l,t+k}-G^{rd}_{l,t-1}}{Y_{l,t-1}} + \alpha_l^k + \delta_{t+k} + \varepsilon_{l,t+k}
    \label{eq:fm_comp}
\end{equation}
where $G^{g}$, $G^{s}$, and $G^{rd}$ are spending in goods, services, and R\&D activities, respectively. The set of coefficients $\{\beta_{g}^k$, $\beta_{s}^k$, $\beta_{rd}^k\}_{k=0}^{4}$ measure the effects of heterogeneous government spending on the local outcome variables. We estimate each triplet of coefficients indexed by $k$ in a separate regression.  This specification enables us to investigates the relative magnitudes of the fiscal multipliers and the difference in the horizon at which these effects take place. 

Finally, It is important to notice that the adding two-way fixed effects implies that the coefficient is estimated with both within MSA variation over time and within time variation across MSAs\cite{}. This implies that the interpretation is about the change in the two-way demeaned growth rate of employment or earnings due to a change in the two-way demeaned growth rate of spending.\footnote{As it has been suggested by the macroeconomics literature, we expect substantial heterogeneity in the estimation of the cross sectional multipliers. Therefore our estimation is safe from the unidentifiable problem when there the coefficient of interest is homogeneous across states or across time.}

\subsection{Instrumental variables design}

As highlighted in the previous literature, military spending is potentially endogenous due political or economic factors. The observed changes in military spending might respond to unobserved economic shocks that affect the outcome variables. For example, private firms in locations with higher military contracts could have exerted more effort trough lobbying to win those contracts. If this firms specific shocks affect the aggregate economic activity, we will have a bias in our estimates. 

To overcome this identification challenge due to the potential endogeneity of military spending, we follow the instrumental variable (IV) approach used by \cite{Nakamura2014} and \cite{Auerbach2020}, among others. This approach consists of constructing a predicted military spending using nation-wide changes in military procurement and a measure of historical comparative advantage that certain MSAs have in obtaining contracts from the DoD. This predicted military spending $Z_{l,t+h}$ is define by the following equation:

\begin{equation}
    Z_{l,t+h} = s_{l} \frac{G_{t+k}-G_{t-1}}{Y_{l,t-1}}
    \label{eq:fm_iv}
\end{equation}

where $s_{l}$ is the average share of spending in location $l$ and captures the comparative advantage that locality $l$ has in obtaining military outlays. For the specification that breakdown total spending into the three categories \eqref{eq:fm_comp}, we construct the instruments as it is indicated below:

\begin{equation}
    Z_{l,t+h}^g = s_{l}^g \frac{G^g_{t+k}-G^g_{t-1}}{Y_{l,t-1}};\;\;Z_{l,t+h}^s = s_{l}^s \frac{G^s_{t+k}-G^s_{t-1}}{Y_{l,t-1}};\;\;Z_{l,t+h}^{rd} = s_{l}^{rd} \frac{G^{rd}_{t+k}-G^{rd}_{t-1}}{Y_{l,t-1}}
    \label{eq:fm_iv2}
\end{equation}

As we investigate the effect of military spending at a higher disaggregation level than previous literature, the volatility of government spending by components in locality $l$ is higher than the volatility of the total spending. Hence, we construct the predetermined average shares in locality $l$ by using the first fifteen years of data, from 1966-1980. Finally, since our empirical specifications are in changes and we use locality fixed effects, the conditional variation that is provided by the instrument comes from national level changes in military spending, which is plausible exogenous to local shocks that affect the outcome variables.




%%%%%%%%%%%%%%%%%%%%%%%%%%%%%%%%%%%%%%%%%%%%%%%%%%%%%%%%%%%%%%%%%%%%%%%%%%%%%%%%%%%%%%%%%%%%%%%%%%%%%%%%%%
%%%%%%%%%%%%%%%%%%%%%%%%%%%%%%%%%%%%%%%%%%%%%%%% DATA %%%%%%%%%%%%%%%%%%%%%%%%%%%%%%%%%%%%%%%%%%%%%%%%%%%%
%%%%%%%%%%%%%%%%%%%%%%%%%%%%%%%%%%%%%%%%%%%%%%%%%%%%%%%%%%%%%%%%%%%%%%%%%%%%%%%%%%%%%%%%%%%%%%%%%%%%%%%%%%

\section{Data}
\label{sec:data}


\subsection{Military Spending}
\label{subsec:data_mil}

We assemble new data on military procurement contracts awarded by the U.S. Department of Defense (DoD). Our data have unique advantages compared to previous studies \citep{Nakamura2014, Dupor2017, Auerbach2020}. We collect and harmonize military procurement contracts data from two sources: National Archives and Records Administration (NARA) for the period $1966$-$2006$, and USASpending.gov for the period $2007$-$2019$.\footnote{The data from USASpending.gov are available from $2001$. We use the period $2001$-$2006$ to validate the quality of the data collected from NARA.} The data from both sources are based on DD-350 and DD-1057 military procurement forms that accounts for about $96\%$ of contracts awarded by the DoD.

The data contain detailed information including the contract identification, the dates of action and completion, the transaction value, the location where the contract is performed, and the Federal supply classification code. There are two main advantages of our data compared to previous works. First, we have geographically disaggregated data at city level available for a long period. These two dimensions, long historical data and rich cross-sectional variation help us to better quantify the causal effect of government spending on economic outcomes. Second, the collected federal product classification enables us to allocate for each contract the amount of spending allocated to goods, services, and R\&D. That allows us to study the heterogeneous fiscal multipliers by components of government spending.

We construct and aggregate the military spending in the following way. We define the year in which a contract is approved as the year of the signature date that is the date when a contract is either awarded or modified.\footnote{The government fiscal year has been defined from October $1^{st}$ to September $30^{th}$ since $1976$. The mismatch between the fiscal year of the government and the calendar year could cause a time inconsistency between the the military spending and the other economic variables. Thus, we use the calendar year as reference year.} The contract completion date corresponds to the delivery date of the requested tasks. The contract dollar value is reported in nominal term. For comparability over time, we convert the nominal transaction value into real values by using the Consumer Price Index from the US Bureau of Labor Statistics.

The data also include modifications to existing contracts. Some modifications consist in downward revisions to contract amounts reported as negative entries.\footnote{For most years, the contract value is reported as an alphanumeric code. The last digit identifies whether the contract is an obligation or a de-obligation. We use the contract dictionaries to decode the alphanumeric strings into numeric values.} We follow \cite{Auerbach2020} and consider contracts with obligations and de-obligations with magnitudes within $0.5\%$ of each other to be null and void. 


\iffalse
We follow two approaches to allocate the military spending for a contract across years. The first approach consists of assigning the entire value of the contract to the year in which the contract has been signed. The second approach follows \cite{Auerbach2020} and it smooths the allocation of the contract value over the duration of the contract, computed as the period between the signature date and the completion date.\footnote{As in this second approach, we need to calculate the period passed between the signature and the completion dates, we remove contracts with missing completion dates or with completion dates before the signature dates.} As the empirical analysis is carried at the annual frequency, we then aggregate the value of a contract by the years covered between the signature and the completion years.
\fi 

The collected data also include the Federal supply classification code is an alphanumeric code and it proxies for the type of product that the contract is requested to deliver. The product code consists of 4-digits, and there are over $1500$ 4-digit products. As argued by \cite{Draca2013}, although some product codes are added or deleted over time, the classification is consistently defined over the years. That makes possible to compare product codes over time. We classify contracts based on the first digit of the code that corresponds to macro-category of the type of product requested by a contract. If the first digit is \textit{A}, then the spending is on ``Research and Development.'' If the first digit is any other letter different from \textit{A}, then the spending is on ``Services.'' If the first digit is a number, then the spending is on ``Goods.'' Appendix \ref{tab:cat_list} in \ref{sec:app_empres} reports the major product codes included in each of the three categories.

Finally, our empirical strategy presented in the previous section exploits the geographic variation in military outlays. As standard in the literature, the geographic allocation is based on the location of the firm that performs the tasks of a contracts. The detailed location information permits us to geolocate contracts in narrow geographic areas. The available information differs between our two sources of data, NARA and USASpending. On the one hand, in the USASpending we know the city and the zip code of the performing firm.\footnote{Following  \cite{Demyanyk2019}, if we know that a contract has been performed in the US, but we do not know the exact location where it was performed, we assign the location of the contract recipient as the performance location. Notice that, differently from \cite{Demyanyk2019}, we adjust a marginal share of contracts. That's the case because we locate contracts not only using the postal code, but also the city. There are contracts for which information about the postal code is missing, but the information about the city is not missing.} We use these two pieces of information to identify the county in which a firm is operating. On the other hand, in NARA, the county in which a firm is performing the contract tasks is reported directly. We then use the spatial crosswalks provided by the National Bureau of Economic Research (NBER) to aggregate the county-level military contracts into Core-Based Statistical Areas (CBSAs) that are the geographic aggregation used in our analysis. As $97\%$ of the government military spending is concentrated in Metropolitan Statistical Areas (MSAs), we restrict our analysis only to these geographic agglomerations.

In \ref{sec:app_data}, we test the comparability of our data with the ones from previous studies. Overall, these tests validate the high-quality and comparability of our data with respect to data previously used in the literature.


\subsection{Economic Outcomes}
\label{subsec:data_eco}

Due to data limitations, gross domestic product by MSA is reported only from the beginning of the 2000s. As our analysis employs historical data, we use to quantify the effect of local spending shocks on economic activities two measures. The first measure is the employee income (pre-tax earnings) that includes salaries and wages, bonuses, stock options, profits, and some fringe benefits. The second measure is the employment that consists in the headcount of employed workers. These data are collected annually from the Bureau of Economic Analysis at MSA-level.\footnote{Both employment and employee earnings are derived from the Bureau of Labor Statistics’ Quarterly Census of Employment and Wages (QCEW).} We collect  employee earnings in nominal terms, and, for consistency with the other monetary variables, we convert them in real terms by using the Consumer Price Index from the US Bureau of Labor Statistics.


\iffalse
\subsection{Innovation Activities}
\label{subsec:data_inn}

We collect patent data from PatentsView that contains the universe of granted patents from the US Patent and Trademark Office (USPTO) starting from 1976 until 2019. These data contain patent-level information including application and grant dates, assigned technology class, the type of patent, the claims, the name of the assignees, the latitude and longitude of their addresses, and the citations’ network consisting of the number of citations made to and received from other patents. 

We restrict our analysis to utility patents that cover the creation of a new or improved product, process, or machine. Utility patents are also commonly known as ``patents for invention,’’ and they account for about $98\%$ of the universe of patents granted by the USPTO. We also restrict to patents with application year starting from 1976.

In the patent data, we do not observe the exact date in which an innovation occurs. As common in the literature, we identify the year when an innovation occurs as the application year of a patent, which is the year when the provisional application is considered complete by the USPTO, and a filing date is set. A patent inventor is a person who who conceives an invention or contributes to the claims of a patentable invention. Thus, the choice of the application year as innovation year seems reasonable.

As well-discussed in the patent literature, for example \cite{Hall2001} or \cite{Lerner2017}, patent data suffer from three major issues that affect the over-time comparison of patent statistics: 1) the changes in the propensity to cite; 2) the lifespan of a patent; and 3) the truncation bias. We address both issues and adjust the microdata by following different strategies developed in the literature. 

The propensity to cite bias is generated by the changing patterns of patenting and citing over time. Starting from the late eighties, we have seen a dramatic acceleration in patenting and citing activity in the US. This increase in the use of patents is view as a response to the increase in the patent protection provided by the legislator, rather than an endogenous rise in the amount of innovation. An over-time comparison of patent activities without a proper adjustment to correct these trends could generate misleading conclusions. Similarly, in addition to changes in the propensity to cite, also a shift to the left of the citation-lag distribution, implying that citations are coming sooner than they used to in the previous decades, could lead to the same misreading of the results. We address these issues by implementing the ``quasi-structural’’ approach developed by \cite{Hall2001}. In \ref{sec:app_data}, we extensively described the methodological approach we follow.

The second critical time effect is linked to the lifespan of a patent. Older patents have longer time to accumulate citations than more recent patents. Thus, if we do not account for this issue, we could get misleading results. As common in the literature, we address this problem, once the previous adjustments have been made, by measuring the quality-adjusted innovation rates within a fixed window of five years after the application year. In this way, we make the citing activities comparable between patents that have different lifespans.

Finally, the truncation bias also mechanically affects the the number of citations that a patent receives from other patents. The truncation bias arises from the fact that patent records are only released at the grant dates when the review process by the USPTO is completed. As a result, the truncation bias causes that the patents in the last years of our sample are mechanically less cited, independently of their innovativeness. As reported by the USPTO,\footnote{``As of 12/31/2012, utility patent data, as distributed by year of application, are approximately 95\% complete for utility patent applications filed in 2004, 89\% complete for applications filed in 2005, 80\% complete for applications filed in 2006, 67\% complete for applications filed in 2007, 49\% complete for applications filed in 2008, 36\% complete for applications filed in 2009, and 19\% complete for applications filed in 2010.’’} the review process takes essentially $8$ years to be fully completed. Therefore, we drop patents with application years later than $2011$. 

We associate a patent with a MSA by using the latitude and longitude of the address of the patent inventors. We geolocate the latitude and longitude into MSAs by using the 2010 TIGER/Line Shapefile constructed by the Us Census Bureau. When patents have multiple inventors whose addresses are in different MSAs, we equally split those patents across MSAs according to the number of inventors. 

We exploit the rich information in the patent data to construct measures of private innovation outcomes. As we want to investigate the effects of government spending on private innovation outcomes, we exclude all patents for which the DoD has some economic interests. Our main outcome variable is the number of patents that measures the number of raw patents that have filed the application at the USPTO. As robustness checks, we also use the number of patents weighted by the number of adjusted forward citations within a 5-year windows after the application year,\footnote{Forward citations are citations received by a patent.}. This measure, in addition to the quantity, captures the innovation quality. 
\fi 










\subsection{Sample Definition}
\label{subsec:data_sam}

We apply some additional filters to the data aggregated by MSA to avoid the estimate to be driven by outliers. First, we exclude MSAs with incomplete histories in any of the variables described above (military spending, earnings, and employment). We remove MSAs with an average population over the period of analysis smaller than 50,000 inhabitants. We also exclude MSAs in which at least in one period the ratio of military spending to earnings is greater than $1.5$. We also drop MSAs with earnings or employment growth rates between two consecutive periods either greater than $1$ or smaller than $-0.5$. To avoid to have a zero instrument for any categories of spending, we remove MSAs that have zero spending in any of the three types of spending over the period $1966-1980$. Finally, as in previous studies, \cite{Auerbach2020} and \cite{Demyanyk2019}, the analysis is at locality aggregated level rather than per capita.








%%%%%%%%%%%%%%%%%%%%%%%%%%%%%%%%%%%%%%%%%%%%%%%%%%%%%%%%%%%%%%%%%%%%%%%%%%%%%%%%%%%%%%%%%%%%%%%%%%%%%%%%%%
%%%%%%%%%%%%%%%%%%%%%%%%%%%%%%%%%%%%%%%%%%% EMPIRICAL RESULTS  %%%%%%%%%%%%%%%%%%%%%%%%%%%%%%%%%%%%%%%%%%%
%%%%%%%%%%%%%%%%%%%%%%%%%%%%%%%%%%%%%%%%%%%%%%%%%%%%%%%%%%%%%%%%%%%%%%%%%%%%%%%%%%%%%%%%%%%%%%%%%%%%%%%%%%

\section{Descriptive Statistics}
\label{sec:des_stats}

As the data are part of the novelty of our study, we briefly discuss their main features. Figure \ref{fig:share_comp} reports the time series of military spending computed by using the full universe of military procurement contracts over the period $1966-2019$.\footnote{The sample consists of more than $20$ millions contracts.} 

Panel A of Figure \ref{fig:share_comp} shows the evolution of aggregate military spending deflated by the CPI. We observe three sharp rises in spending. The first increase is in the 1980s as a consequence of the Reagan military buildup, the second is in the early 2000s due to the Afghani and Iraqi wars, and the last is in the recent years due to the escalations in military buildups with Russia and China.

Panel B of Figure \ref{fig:share_comp} reports the shares of military spending allocated to each of the three spending categories: goods, services, and R\&D. There are two main takeaways. First, the largest share of government spending is directed to the purchase of goods, followed by the purchase of services, and finally, by the investment in research and development activities. As reported in the first row of Table \ref{tab:desstats_contracts}, over the period $1966$-$2019$ the average share of spending in goods, services, and R\&D are $54\%$, $31\%$, and $15\%$, respectively.

Second, starting in the 1980s we observe a reallocation of government spending from goods to services. While at the beginning of the 1980s the spending in goods was about $60\%$ of the value of the military procurement contracts compared to $20\%$ in services, at the end of the 1990s the two shares were both around $40\%$. The share of spending in R\&D has remained more constant around $0.2$ until the beginning of the 2000s. From the 2000s onward, we document a drop in R\&D spending of about $50\%$, from $0.16$ to $0.08$, in favor of spending in goods.

Due to the filters described in the previous section, the empirical analysis includes $296$ MSAs. We exclude all procurement contracts awarded to micropolitan statistical areas and rural counties. The $296$ MSAs included in the sample account for a large share of the military procurement spending. Our sample includes $90\%$ of both the total number of contracts and the aggregate spending. These shares are greater than $90\%$ for spending in goods and R\&D. Not surprisingly, due to a less degree of tradeability, services are less geographically concentrated. Indeed, our sample accounts for about $80\%$, instead of $90\%$, of the aggregate spending in services.\footnote{Figure \ref{fig:share_comp_sample} in \ref{sec:app_empres} replicates Figure \ref{fig:share_comp} only for the contracts awarded to the MSAs in our sample. The patterns are substantially consistent with the ones described above.} Finally, per capital employee earnings in the MSAs from our sample are a $20\%$ greater than the average earnings in the US.

Let us now explore the heterogeneity across military contracts awarded to MSAs included in the sample for the empirical analysis.\footnote{Statistics computed for the full universe of procurement contracts provide the same insights reported here for the restricted sample.} Table \ref{tab:desstats_contracts} reports some basic descriptive statistics on the contract characteristics by category of spending. $90\%$ of procurement contracts are signed by the government to purchase goods from the private sector. Only $8.5\%$ of contracts is awarded to provide services, and even a smaller share, about $1.5\%$, requires R\&D activities. 

The results about the shares of contracts joint and spending imply, as showed in the third row of Table \ref{tab:desstats_contracts}, that On the one hand, the average outlay in R\&D component per contract is four times the average spending across all contracts. On the other hand, the average spending in goods per contract is smaller than the average spending over all contracts. 

The last two rows of Table \ref{tab:desstats_contracts} explore the distributional characteristics of spending within categories. The distribution of contracts for the purchase of goods has a significantly fatter right-tail than the distributions for spending in services or R\&D. In the case of spending for goods, the contract at the top decile of the distribution of outlays is almost $40$ times greater than the median contract. The $90\%$-to-$50\%$-percentile ratios are significantly smaller for spending in services and R\&D, with the contract at the top decile of the distribution to be $12$ times greater than the median contract for services and $8$ greater for R\&D. 

The last row of Table \ref{tab:desstats_contracts} shows the share of spending allocated to contracts in the top decile. As one can notice, the top $10\%$ of contracts awarded to purchase goods accounts for $99\%$ of the total spending for goods. This result implies that although the majority of contracts have a component of spending for goods, only relatively few matter in size. The share of spending in services and R\&D allocated to the top decile is around $80\%$ implying a more equal allocation of spending across contracts.  

These results provide some insights on the government procurement process. R\&D spending is awarded through few but large contracts and the dispersion across these contracts is relatively small. The purchase of goods occurs through a large number of contracts of which only few of them account for a sizeable monetary value. Finally, spending in services is in between the two previous cases.

Before turning our attention to the regression results, as our identification strategy exploits the cross-sectional variation in spending, it is worth to explore the geographic heterogeneity in the allocation of outlays across the MSAs in our sample. First, military spending is unequally distributed across MSAs. Figure \ref{fig:map_spend} shows the quartile to which a MSA belongs based on the average value of the military spending that it has receive over the period $1966-2019$. There are two main results to highlight. As showed in Panel A, most of the MSAs in the top quartile are located along the two coastal regions, and the Midwest. 

The visual inspection of the remaining three panels of Figure \ref{fig:map_spend} suggests that there are differences in the geographic allocation among the three types of spending. These differences are particularly marked in the Midwest. While most MSAs in the Midwest are in the top two quartiles of spending in goods, only few of these MSAs are ranked as high in the distribution of spending in services. These results are consistent with the geographic economic structure of the US. The Midwest has the highest concentration of production occupations with a average employment share in production jobs almost $50\%$ higher than the US average. Only $14\%$ of MSAs are in the top quartile in all categories of spending. These shares remain similar if we compare pairs of types of spending.\footnote{$14\%$ of MSAs are in the top quartile in goods and services spending. $16\%$ of MSAs are in the top quartile in R\&D and service spending. $19\%$ of MSAs are in the top quartile in R\&D and good spending.} The differences in the production structure between regions also reflect in the government allocation of types of spending.

Table \ref{tab:desstats_cbsas} reports a set of descriptive statistics on the distribution of military spending across the MSAs in the sample. The figures emphasize that the spending is unequally distributed across MSAs. The $90\%$-to-$50\%$-percentile ratio is $22$ for good spending, $18$ for service spending, and even $77$ for R\&D spending. The second row highlights between $55\%$ and $65\%$ of the spending in each category is awarded to the top decile of the distribution of spending by MSA.\footnote{The top $10\%$ MSA receivers of R\&D spending receive $55\%$. In these MSAs is also located the primary address of the inventors of $55\%$ of granted patents by the USPTO.}


To sum up, the results provide evidence that although the DoD signs a large number of contracts, the largest share of government procurement spending is captured by a handful of contracts. Furthermore, these contracts are not equally distributed across the MSAs, but about $20-30$ MSAs receive over $60\%$ of the entire spending. These findings imply that the distribution of spending at both contract and MSA level is significantly skewed to the right. 

Our findings are complementary to \cite{Cox2021}. They show that government spending is granular and concentrated among a few firms. Their analysis is based on the contract recipients. The granularity in recipients does not necessary imply a granularity in firms actually performing the tasks of a contract. It could happen contract recipients allocate several tasks of a contract to different performers evenly located in the national territory. If that were the case, we would not observe a geographic concentration in the spending based on the place of performance. Our results show the opposite suggesting, in addition to a granularity in contract recipients, also a granularity in contract performers.\footnote{We cannot test directly this granularity in contract performers because the data do not report the name of the firm that performs a contract. We only observe the place of performance.} Finally, similarly to \cite{Cox2021}, we also show a substantial variation in the range of contract values. Our results also emphasize that there exist significant differences in the distributional features of contract size across categories of spending.



\section{Main Result}
\label{sec:emp_fm}

Before talking about the estimation results, it is worth to highlight that our study provides estimates for the local employment and earnings multipliers. An important caveat in the literature that study a local multiplier, is that its estimates cannot be easily translated into a national multiplier for two reasons: First, because there existence of spillover effects across localities. Second, the parametrization of macroeconomic models that map the local to the national fiscal multiplier generate a broad range of estimates. In the following of the paper, we will use the term local fiscal multiplier and fiscal multiplier interchangeably. Also we want to highlight that, as all other studies who study the impacts of fiscal policy at MSA level, we do not have accurate measures of ouput for most for our period of study 1980-2019. Therefore, following \cte{} we report employment and earnings multipliers rather than output multipliers. This outcomes respond to data constraints, namely, the granularity of employment data publish by the Bureau of Labor Statistics is the only one that match the granularity of our spending data. 

Figure \ref{fig:fm_earnings} plots the estimates the effect of military spending shocks on employee earnings at different horizons after the shock occurs. The shaded area represents the $90\%$ confidence interval. Part A of Table \ref{tab:fm_main} in \ref{sec:app_empres} reports the exact point estimates.

Panel A shows the impact of the total military spending on earnings using the estimation approach in equation \eqref{eq:fm_base}. The estimates suggest that an increase in military spending by a percent of local earnings causes earnings to increase by $0.21$ percent on impact. As the time passed, the effect of the military spending on earnings become larger, moving from $0.2\%$ on impact to about $0.6\%$ four periods after the shock. Our estimates are comparable in size and significance to the ones reported by \cite{Auerbach2020}. 

Turning now to the decomposition of military spending by categories, we document a significant heterogeneity in the effects. The estimates in each of the remaining panels of Figure \ref{fig:fm_earnings} refers to one of the terms in equation \eqref{eq:fm_comp}. Panel B shows the effect of shocks in goods spending. The estimates are close to zero in magnitude and they are non-significant at any horizon after the occurrence of the shock. Panel C presents the earnings effect of shocks in services spending. The effect of one percent increase in spending in services on earnings is of $0.35\%$ and it is significant at the $1\%$ significance level. Employee earnings continue to increase as time after the shock passes, reaching an effect of $0.8\%$ 3-4 periods after the shock. Finally, Panel D tracks the impulse-response function after the occurrence of shock to the spending in R\&D. The estimates show that, on impact, earnings are only slightly affected, but after a few periods the earnings significantly increase up to around $3\%$ per each $1\%$ increase in the spending.\footnote{We test whether the correlations between the endogenous regressors and the instruments are small. We compute the first-stage F-statistic, and we find the instruments do not suffer from the weak-instrument problems as the F-statistic is well above 10 in all specifications.}

On the one hand, these results suggest that the on impact increase in earnings after a shock to spending in services is due to an increase demand. On the other hand, although spending in R\&D generates small short-term gains in earnings, the medium-term increase is significantly larger than the effect generated by spending in services. One plausible explanation for the lagged large effect of spending in R\&D on earnings is R\&D investment produces new innovations that, in turn, improve productivity. The lagged effect is consistent with the idea that innovation needs time to be developed and used in production.\footnote{A vast literature has documented a productivity paradox that consists in an initial slowdown in productivity after the adoption of a new technology. New innovations need a few periods to become productive and benefit the production process. See, among the others, the works of \cite{Huggett2001} and \cite{Acemoglu2014}.}

We consider employment as a second measure of economic activities. Figure \ref{fig:fm_employment} plots the estimates the effect of military spending shocks on employment at different horizons. Part B of Table \ref{tab:fm_main} in \ref{sec:app_empres} reports the exact point estimates. The results for the employment multipliers are qualitatively identical to the ones presented on the earnings multipliers. An increase of a percent in local government spending increases the local employment by $0.13\%$ on impact and by $0.32\%$ four periods after the shock.

One concern with our estimates is that the results capture short-term movements in population between localities due to government spending. We address this concern by re-estimating our main specifications with per capita earnings rather than aggregated at MSA-level. The results for the earnings per capita are reported in Table \ref{tab:fm_percapita} in \ref{sec:app_empres}. The per capita estimates confirm the previous findings of a positive and lagged effect of spending in R\&D on employee earnings, and a positive effect of service spending since the shock occurs.

Our approach has excluded localities with incomplete military spending histories. It still could be the case that some MSAs have zero entries in any of the spending category. One concern is that the zero entries play a quantitatively important role in the estimation of the multipliers. We address this concern by further restricting the sample and dropping all MSAs with zero entries. The results for both earnings and employment multipliers are reported in Table \ref{tab:fm_restricted} in \ref{sec:app_empres}. Although the sample size significantly decreases to $113$ MSAs, the findings remain unchanged, implying that zero entries do not play an important role in our estimation.

\iffalse
Another valid concern is the timing of actual spending. In the data, we observe the start and end date of the contract. In the regressions presented in sub-section \ref{subsec:emp_fm}, we use the year in which a contract has been signed to allocate the spending, and not the years in which the disbursements actually occur. To alleviate this source of spending mismeasurement, we follow \cite{Auerbach2020} and we construct a flow spending measure for each contract by allocating the value of a contract equally over its duration. The results for both the earnings and employment multipliers are reported in Table \ref{tab:fm_smooth} in \ref{sec:app_empres}. ???????????????
\fi

To sum up the previous results, spending in services and R\&D generate increases in economic outcomes. Spending on services has a medium-size effect since the time the shock occurs. Spending on R\&D produces a larger increase in the economic activities, but this effect is delayed due to technology creation and adoption. Surprisingly, spending in goods has no effect on the multipliers at any horizon and in any of the specifications we tested.




%%%%%%%%%%%%%%%%%%%%%%%%%%%%%%%%%%%%%%%%%%%%%%%%%%%%%%%%%%%%%%%%%%%%%%%%%%%%%%%%%%%%%%%%%%%%%%%%%%%%%%%%%%
%%%%%%%%%%%%%%%%%%%%%%%%%%%%%%%%%%%%%%%%%%%%%%%%%%% MODEL %%%%%%%%%%%%%%%%%%%%%%%%%%%%%%%%%%%%%%%%%%%%%%%%
%%%%%%%%%%%%%%%%%%%%%%%%%%%%%%%%%%%%%%%%%%%%%%%%%%%%%%%%%%%%%%%%%%%%%%%%%%%%%%%%%%%%%%%%%%%%%%%%%%%%%%%%%%


%%%%%%%%%%%%%%%%%%%%%%%%%%%%%%%%%%%%%%%%%%%%%%%%%%%%%%%%%%%%%%%%%%%%%%%%%%%%%%%%%%%%%%%%%%%%%%%%%%%%%%%%%%
%%%%%%%%%%%%%%%%%%%%%%%%%%%%%%%%%%%%%%%%%% LABOR INTENSITY %%%%%%%%%%%%%%%%%%%%%%%%%%%%%%%%%%%%%%%%%%%%%%%
%%%%%%%%%%%%%%%%%%%%%%%%%%%%%%%%%%%%%%%%%%%%%%%%%%%%%%%%%%%%%%%%%%%%%%%%%%%%%%%%%%%%%%%%%%%%%%%%%%%%%%%%%%


\section{Mechanism: The role of Labor Intensity}
\label{sec:lab_int}

In this section, we present two pieces of evidence that support the idea that differences in the fiscal multipliers by type of spending are due to the different intensity of labor usage in the production of the products to which these types of spending are directed.

The first step of our analysis consists in assessing the labor intensity from each contract. Our dataset does not contain any information on the amount of inputs used in the production. thus, we use an alternative strategy to quantify the labor intensity. We collect annual data from the BEA on value added and employees' compensation by industry.\footnote{The data contain $81$ NAICS codes starting from $1997$. Although valued added and its components are available from the BEA since $1988$, the industrial aggregation for years before $1997$ is not comparable with the one for the years after $1997$.} We then compute a measure of labor intensity as the contribution of employees to the value added. Finally, we assign the constructed measure of labor intensity to each contract based on the industry to which the contractor belongs. We remove from the sample contracts not linked to any industry. As we match contracts that amount for more than $98\%$ of the total value of military spending from $1997$, this restriction does not seem to affect our analysis. Table \ref{tab:indlabint_list} reports the classification of the available industries separated between low- and high-labor intensity. Our classification matches the common sense. Indeed, industries as healthcare, education, hospitality and food service are classified as high labor-intensive, while manufacturing and retail as low labor-intensive.

Figure \ref{fig:shlabint_comp} reports the evolution of the share of spending allocated to labor-intensive industries by type. We define an industry as labor-intensive if the average labor intensity in that industry is in the top half of the distribution of these averages. About three-quarter of the total military spending is directed to industries with high labor intensity. The share of spending allocated to labor-intensive industries significantly varies across types of spending. On average, $78\%$ of the spending in services goes to industries that rely more on employees. This share is about $8$ percentage points greater than the share allocated from spending in goods. Finally, over $91\%$ of spending in R\&D is directed to industries with higher labor intensity. These results present a first evidence on the importance of labor intensity in determining the difference in the fiscal multipliers across categories of spending. Indeed, the local fiscal multipliers are higher for types of spending whose larger share is allocated to more labor-intensive industries.

Intuitively, our main test to assess the role of labor intensity in determining the differences in the local fiscal multipliers across categories of spending consists in running the specification \ref{eq:fm_comp} considering separately the components of spending for each category that goes to high and low labor-intensive industries. Specifically, we run the following regression: 
\begin{equation}
\begin{split}
    \frac{v_{l,t+k} - v_{l,t-1}}{v_{l,t-1}}  = & \gamma_{g,H}^k \frac{G^{g,H}_{l,t+k}-G^{g,H}_{l,t-1}}{Y_{l,t-1}} + \gamma_{s,H}^k \frac{G^{s,H}_{l,t+k}-G^{s,H}_{l,t-1}}{Y_{l,t-1}} + \gamma_{g,L}^k \frac{G^{g,L}_{l,t+k}-G^{g,L}_{l,t-1}}{Y_{l,t-1}} + \gamma_{s,L}^k \frac{G^{s,L}_{l,t+k}-G^{s,L}_{l,t-1}}{Y_{l,t-1}} \\[0.1in] 
    & + \gamma_{rd}^k \frac{G^{rd}_{l,t+k}-G^{rd}_{l,t-1}}{Y_{l,t-1}} + \alpha_l^k + \delta_{t+k} + \varepsilon_{l,t+k}
\end{split}
\label{eq:fm_comp_labint}
\end{equation}
where $G^{g,H}$ represents the spending in goods produced by high labor-intensive industries, and $G^{g,L}$ is spending in goods produced by low labor-intensive industries. Similarly,  $G^{s,H}$ and $G^{s,L}$ are the spending in services produced by high and low labor-intensive industries, respectively. In the case of spending in R\&D, $G^{rd}$, in the benchmark specification, we do not split it between high and low labor-intensive industries. This choice is motivated by the fact that, as showed in Figure \ref{fig:shlabint_comp}, over $90\%$ of the R\&D spending is directed to high-intensity industries. Including separately R\&D spending in high and low-intensity industries generates serious estimation issues because a large share of MSAs has several years of zero spending in R\&D directed to low labor-intensive industries.\footnote{Although not reported in the paper, as a robustness check, we estimate the model by splitting the R\&D spending between high and low-intensity industries. The results for good and services spending are robust, but the estimation of coefficients for R\&D spending directed to low labor-intensive industries is highly imprecise.} We instrument each regressor with instruments as defined in equation \eqref{eq:fm_iv} for each pair of type of spending and labor intensity.\footnote{As data on labor intensity by industry are available only from $1997$, we cannot construct the instruments for the period $1966-1980$ as above. Therefore, we construct the instruments using the entire available period $1997-2015$.}

Table \ref{tab:fm_labint} reports the impulse response functions estimated from specification \eqref{eq:fm_comp_labint}. Panel A reports the local fiscal multipliers split between government fiscal shocks directed to high and low labor-intensive industries. The positive effects on earnings come exclusively from government shocks directed to industries with high labor intensity. Indeed, the estimates for shocks to high labor-intensive industries are statistically significant at any time horizons and larger in size than the estimates from spending in low intensity industries.

We turn now to Panel B that explores the effects for the different categories of spending by labor-intensity usage. The first two rows of Table \ref{tab:fm_labint} show positive and statistically significant effects on earnings for spending in either services or goods in response to a government shock directed to industries that intensively use labor. In terms of magnitude, the estimates highlight that the effect of a shock to spending in services in high labor-intensive industries is about three times the impact of a shock to spending in goods in high labor-intensive industries. The third and fourth rows highlight that independently on the type of spending, in either goods or services, a fiscal government shock directed to low labor-intensive industries generate local fiscal multipliers that are not statistically different from zero at any time horizons after the shock. In terms of size, as for spending directed to high labor-intensive industries, the multipliers are larger for spending in services rather than in goods. Finally, the spending in R\&D generates a sizeable multiplier effect, but smaller than the ones previously showed in Figure \ref{fig:fm_earnings}.

These results provide a strong support that the differences in the multipliers across categories of spending are due to the amplification of the government shock through the intensity of labor usage in the production of the different tasks required by the types of spending.









%%%%%%%%%%%%%%%%%%%%%%%%%%%%%%%%%%%%%%%%%%%%%%%%%%%%%%%%%%%%%%%%%%%%%%%%%%%%%%%%%%%%%%%%%%%%%%%%%%%%%%%%%%
%%%%%%%%%%%%%%%%%%%%%%%%%%%%%%%%%%%%%%%%%%%%%%%%%%% ALTERNATIVE MECHANISM %%%%%%%%%%%%%%%%%%%%%%%%%%%%%%%%
%%%%%%%%%%%%%%%%%%%%%%%%%%%%%%%%%%%%%%%%%%%%%%%%%%%%%%%%%%%%%%%%%%%%%%%%%%%%%%%%%%%%%%%%%%%%%%%%%%%%%%%%%%

\section{Ruling Out Alternative Mechanisms}
\label{sec:emp_mec}

This section explores alternative mechanisms, proposed in the literature, that could explain the differences in the fiscal multipliers by category of spending. Overall, our empirical analysis highlights that none of these alternative mechanisms generates these differences.

\subsection{Sector tradeability}\label{subsec:outflow}

Physical goods are more tradeable than services.\footnote{The average service industry is less tradable than the
average manufacturing industry. However service tradeability has been growing over time \citep{gervais2019tradability}.} If the production of goods occurs in the neighboring locations, then one could observe no effects of good spending on the local fiscal multiplier. If that were the case, differences in the fiscal multipliers would reflect geographic spillovers rather than the actual nature of the spending, and these spillovers would be more sizable for more tradeable products as physical goods.

This explanation could be relevant in our context due to the well-documented issues in correctly allocating government spending to localities. Our data only contain records on prime contracts and they do not reflect the amount of subcontracting for basic and intermediate materials and components. A contract is assigned to its place of performance defined as the place where the product is assembled or processed. If the intermediate steps of the production  are done by sub-contractors outside the location of interest, we would geographically misallocate part of the spending. In addition to that, the definition of the place of performance slightly varies across categories of products.\footnote{The location of the majority of manufacturing contracts reflect the location of the plant where the product is finally assembled or processed. The location of construction contracts corresponds to the location were the construction is performed. The location for contracts involving purchases from wholesale or other distribution firms reflects the location of the contractor's place of business. Finally, for service contracts, the location is the place where the service is performed, with the exception of transportation and communication services that report the location of the contracting firm.} The measurement errors in allocating contracts to MSAs could affect our results, and these effects could be heterogeneous across categories of spending.

We implement two test to quantify the importance of the geographic allocation of contracts in driving our previous results. The literature has extensively argued that the geographic misallocation of contracts becomes less important as one moves to larger geographic aggregation. Specifically, it becomes a minor issue at state-level. \cite{Isard1962} argues that geographic disaggregation at state-level does not contain significant measurement errors. \cite{Nakamura2014} use shipment data to the government from defense industries, reported by the U.S. Census Bureau from $1963$ to $1983$, and verify that, on average, the relationship between the prime contracts allocated to a state and the shipments from that state is one-for-one. These two studies imply that on average, all contracts that are allocated to a state are also performed in that state.

Our first test consists in comparing the estimates from the regressions using MSA-level data with the ones using state-level data. If the concern with the geographic allocation of spending to MSAs is not a main driver of our estimates, one should expect that the state-level analysis would lead to the same conclusions previously discussed. The estimates reported in Table \ref{tab:fm_state} of \ref{sec:app_empres} are qualitatively comparable to the ones presented in section \ref{sec:emp_fm}. The sizes of the effects at any horizons are much larger by using the state aggregation. This result is in line with the findings of \cite{Demyanyk2019} and \cite{Auerbach2020}. They argue the discrepancy between MSA-level and state-level estimates can be attributed to within-state subcontracting. In smaller geographic areas, it is more likely that part of the spending to spill into or from other geographic areas. Thus, the potential measurement error in subcontracting outside a MSA attenuates the estimates of the fiscal multipliers, implying that our MSA-level estimates can be considered as lower bounds. Even if we consider state-level estimates, we do not find any significant effect of government spending on goods on the fiscal multiplier, while the effects for service and R\&D spending remain significant at least at the $10\%$ level.\footnote{The significance of our estimates is lower in the state-level regressions than in the MSA-level analysis. That's the result of loss in power in the state-level regressions due to less cross-sectional variation in the military spending.} Once we account for the effect that subcontracting might have on our estimates, the main findings from the previous section remain unaltered.

Likely, MSAs that are along the state borders have strong economic interactions with other MSAs outside the state. In these cases, government spending allocated to locality $l$ could be used for production in some neighboring locations outside the state borders. Our previous test would not capture these cross-state ``outflows''. Our second test, instead, captures these interactions. We implement the following specification to investigate whether military spending shocks in location $l$ have some positive effect in the neighboring locations:
\begin{equation}
    \frac{\tilde{Y}_{l,t+k} - \tilde{Y}_{l,t-1}}{\tilde{Y}_{l,t-1}} = \tilde{\beta}_{g}^k \frac{G^{g}_{l,t+k}-G^{g}_{l,t-1}}{\tilde{Y}_{l,t-1}} + \tilde{\beta}_{s}^k \frac{G^{s}_{l,t+k}-G^{s}_{l,t-1}}{\tilde{Y}_{l,t-1}} + \tilde{\beta}_{rd}^k \frac{G^{rd}_{l,t+k}-G^{rd}_{l,t-1}}{\tilde{Y}_{l,t-1}} + \alpha_l^k + \delta_{t+k} + \varepsilon_{l,t+k}
    \label{eq:fm_spill}
\end{equation}
where $\tilde{Y}_{l,t+k}$ is the total earnings for the neighboring locations of $l$. We define neighboring locations to $l$ MSAs whose center is located within a $100$ miles radius distance from the center of $l$.\footnote{While \cite{Auerbach2020} consider one neighbor location within the distance with a similar size to location $l$, we include all neighboring MSAs within the distance.} Similarly, the instrument for each spending category is defined as it is indicated by equation \ref{eq:fm_iv} but keeping in mind that now spending is normalized by $\tilde{Y}_{l,t-1}$.

\begin{equation}
    Z_{l,t+h}^g = s_{l}^g \frac{G^g_{t+k}-G^g_{t-1}}{\tilde{Y}_{l,t-1}};\;\;Z_{l,t+h}^s = s_{l}^s \frac{G^s_{t+k}-G^s_{t-1}}{\tilde{Y}_{l,t-1}};\;\;Z_{l,t+h}^{rd} = s_{l}^{rd} \frac{G^{rd}_{t+k}-G^{rd}_{t-1}}{\tilde{Y}_{l,t-1}}
    \label{eq:fm_iv2}
\end{equation}

\noindent Results are showed in Figure \ref{fig:earn_spill}. Table \ref{tab:earn_spill} in \ref{sec:app_empres} reports the point estimates.\footnote{$12$ MSAs are excluded because they do not have any neighboring locations within $100$ miles. We also tested, without reporting in the paper, the effect of reducing or increasing the distance on the results reported in the paper. The size of the effects become bigger as distance increases. Nevertheless, the main takeaways remain unchanged.}

Panel A shows that, on average the ``outflow'' effects are small and statistically non-significant. The remaining three panels present the results for each category of spending. Panel B shows the effect on neighboring locations of fiscal shocks in good spending. The finding highlights there are no spillover effect of a shock in locality $l$ on the neighboring localities. This result is important and it reinforces the conclusion that the zero fiscal multiplier effect of good spending cannot be explained by spillover effects on the neighboring locations. The two bottom panels show opposite stories. While Panel C shows that a spending shock in services has some positive and significant ``outflow'' effects on neighboring locations, Panel D highlights that R\&D spending has some small but negative spillover effects on the surrounded MSAs.
 
To sum up, these results point out that tradeability of the different type of products cannot explain the heterogeneous reactions of earnings to different types of government spending. Specifically, even accounting for these factors, we find that spending in goods does not generate any positive significant change in earnings and employment. These results imply that the heterogeneous reactions of outcome variables to different types of spending does not depend of geographic aggregation or misallocation of contracts.


\subsection{Crowding-out of Private Consumption}
\label{subsec:crowd}

Government spending might crowd out private consumption when the increase in government demand is followed by an increase in prices rather than an increase in total production.\footnote{There is no conclusive evidence whether government spending crowds-in or crowds-out private consumption. While \cite{Bailey1971} and \cite{Barro1981} suggest that there is a crowding out effect. Other studies have found the opposite, a crowding in effect. For example, \cite{Perotti2005}, \cite{Canzoneri2002}, and \cite{Mountford2009}, argue the opposite, i.e. government spending crowds in private spending.} A limited response of output could be explain by adjustment cost from the side of the production factors. Since capital tend to have higher adjustment costs than labor,\ref{} one may expect that capital intensive products are less responsive to government spending shocks than labor intensive products.Therefore, the heterogeneity in capital/labor intensity can be the reasons behind the different response that we observe for different types of spending. This may be the case because the manufacturing sector, which mostly produce goods, is more capital intensive than the service sector.  

%prIt could be that the manufacturing industries, who are capital intensive, have lower margin of adjustment, than service sectors who are labor intensive. This would imply that a demand shock in goods will be less  producers will lead to higher inflationary pressures than a demand shock in services, and as a consequences will reduce the consumption to adjust their production and therefore do not respond to this increase in demand like eto In a non-perfectly competitive environment, prices would not adjust enough to absorb the excess in demand causing a substitution of private consumption with government consumption. \textcolor{red}{REFERENCE} As capital is less flexible than labor and capital adjustments to absorb excess in demand need more time, binding feasibility constraints are more likely to be observed for capital-intensive firms that produce goods. Thus, if spending in goods crowd out more private consumption than spending in services or R\&D, then one could observe a zero fiscal multiplier effect for spending in goods. 

To test this channel, we investigate whether different types of government spending has different effects on private consumption expenditure.\footnote{\cite{Dupor2021} investigate the effect of government spending on different categories of consumer spending. Our paper studies the opposite hypothesis with respect to \cite{Dupor2021}.} Due to the lack of private consumption data, that is representative at the MSA level, we carry out this test at state-level.\footnote{State-level data are collected from the Bureau of Economic Analysis for the period $1998-2019$. The data include expenditure in durable and non-durable goods, and services. The figures are deflated by using the national CPI.} We implement the specification in equation \eqref{eq:fm_comp} by replacing $v_{l,t}$ with the private consumption expenditure by state $l$. Our estimates are plotted in Figure \ref{fig:crowd_privcons}. The point estimates are reported in Table \ref{tab:crowd_privcons} in \ref{sec:app_empres}. The results clearly highlight the absence of private consumption crowding out effects. All estimates, for any type of spending and at any horizon, are close to zero and non statistically significant at any standard level. Our results suggest that government spending has neither a crowding in nor a crowding out effect on private consumption expenditure.

Overall, our results point out that as the crowding out effects are small, they cannot explain the differences in multipliers across categories of spending. Specifically, we do not document any significant crowding out effect due to spending in goods that might be the offsetting force that leads to a zero effect in the good spending fiscal multiplier. 

\subsection{Firm dynamics and entry costs}\label{subsec:bus_dyn}

The effects of government spending on economic activity may be explained by public spending reducing affecting firm behavior (investment and growth) because a stronger aggregate demand may reduce the uncertainty about future profits or ease credit constraints. If that is the case, the increase in economic growth may be explained by either higher firm entry\footnote{\cite{Lewis2017} find that net firm entry rises after an expansion in the U.S. government spending.} or a stronger expansion of incumbent firms.\footnote{On one hand, some studies show that incumbent firms expand after winning a contract in US~\citep{Juarros2021} and Brazil~\citep{Ferraz2015, Lee2021}. On the other hand, \cite{Atanassov2018} show that government spending crowds out the R\&D effort of private firms.} Which are two canonical mechanisms to explain economic growth in economies with firm-level heterogeneity~\citep{Acemoglu2018}.

This may explain the heterogeneous impact of fiscal policy if the firm entry or expansion have a larger fixed cost in the manufacturing sector compared to the services sector. Therefore the mass of firms who are at the margin of entering or not to the market is higher in the services sector. 

We test this conjecture with two exercises. First, we study the impact of government spending by category on the net entry rate of firms. We collect data at MSA-level on establishments entry and exit rates from the Business Dynamics Statistics Datasets. We ran the specification in equation \eqref{eq:fm_comp} by replacing the outcome variable $v_{l,t}$ with the establishment net entry rate in MSA $l$ at time $t$. Results are reported in Table \ref{tab:entry_rate}. 

Panel A shows changes in establishment entry rate after a local spending shock. The estimates suggest local fiscal shocks have marginal effects on the creation of new establishments. We only document a positive and significant effect on the creation of new establishment one period after a one percent increase in good spending. A standard endogenous growth model would predict, as a consequence of the increase establishment entry due to the government spending in goods, a rise in output instead of a zero fiscal multiplier effect. Panel B show establishment exit rates after a local fiscal stimulus. We report a strong negative effect on establishment exit rates after a shock in service spending. An endogenous growth model would predict that spending in services to lower the economy-wide growth rate and decrease output. Estimates from Figure \ref{fig:fm_earnings} show the opposite pattern. 

Second, we explore the effect of government procurement contracts on the innovation activities of existing firms. We measure innovation activities by the number of granted patents to inventors in specific localities. We collect patent data from PatentsView at the end of 2019. The PatentsView database contains the universe of granted patents from the US Patent and Trademark Office (USPTO) starting from 1976 until 2019. These data contain patent-level information including application dates, the type of patent, the name of the inventors, and the latitude and longitude of their addresses. We restrict our analysis to utility patents that cover the creation of a new or improved product, process, or machine. Utility patents are also commonly known as ``patents for invention,’’ and they account for about $98\%$ of the universe of patents granted by the USPTO. We also restrict to patents with application year starting from 1976. In the patent data, we do not observe the exact date in which an innovation occurs. As common in the literature, we identify the year when an innovation occurs as the application year, which is the year when the provisional application is considered complete by the USPTO, and a filing date is set. We use the latitude and longitude of the address of an inventor to geolocate the patents ans assign them to MSAs.

To study the effect of government spending on private innovation, we impose some additional filters to the sample. First, we remove all MSAs with incomplete histories in granted patents and citations, and with patent growth rates between two consecutive periods greater or smaller than $\pm 150\%$. Second, the truncation bias arises from the fact that patent records are only released at the grant dates when the review process by the USPTO is completed. As a result, the truncation bias causes that the number of patents in the last years of the sample are mechanically fewer. The review process takes essentially $8$ years to be fully completed.\footnote{The USPTO reports:  ``\textit{As of 12/31/2012, utility patent data, as distributed by year of application, are approximately 95\% complete for utility patent applications filed in 2004, 89\% complete for applications filed in 2005, 80\% complete for applications filed in 2006, 67\% complete for applications filed in 2007, 49\% complete for applications filed in 2008, 36\% complete for applications filed in 2009, and 19\% complete for applications filed in 2010.}’’} Thus, to alleviate the truncation bias, we end the sample period in $2011$. 

Table \ref{tab:patent} reports the estimates from the specification in equation \eqref{eq:fm_comp} where the dependent variable $v_{l,t}$ measures the number of patents granted to inventors resident in location $l$ at time $t$.\footnote{The number of patents only includes patents for which the DoD has no economic interest. We restrict only to private patents to better proxy for private innovation efforts. Results remain unchanged if we include the patents for which the DoD has economic interests.} Our estimates suggest a significant crowding out effect of private innovation due to an increase in government R\&D spending. Other types of spending do not cause any significant changes in the amount of private innovation. One possible explanation for the crowding-out effect of government spending in R\&D on private innovation is the scarcity of researchers. If the local labor supply of researchers is quite inelastic in the short-run, then a demand increase of R\&D activities generates an excess in demand for research activities leading to a switch from private innovation projects to government funded projects. According to an endogenous growth model, these estimates would suggest, differently from what we have showed in sub-section \ref{subsec:emp_fm}, a decline in economy-wide growth rate and a drop in output.

The previous tests explore the impact on business dynamism and innovation, two main mechanisms suggested by a standard endogenous growth models, explain the results we observe. The empirical estimates are not consistent with this mechanism explaining the effects we observe on total employment growth. If anything we observe a puzzling crowding out effect of government spending on R\&D on local innovation. This fact needs to be investigated further. 


\begin{comment}
\subsection{Marginal Propensity to Consume}
\label{subsec:mpc}

Government spending is not equally distributed across the US. Figure \ref{fig:map_spend} also emphasizes that the distribution of government spending among locations varies by type of spending. Locality-specific characteristics might matter in the transmission of a fiscal shock to the economy. One potential explanation for the heterogeneous fiscal multipliers by category of spending is marginal propensities to consume (MPCs) are different across locations. Specifically, if spending in physical goods were directed to locations with lower MPCs, then one could observe a low or even zero fiscal multiplier.

To test this mechanism, we compute the MPCs at state-level using the Panel Study of Income Dynamics (PSID) from $1999$ to $2019$. Our estimation is based on the PSID Core Sample after applying a few restrictions. First, we remove households with missing information on race, education, or state of residence. We also drop households that experience income growth greater than $500\%$ or drops greater than $80\%$. We discard households with income below $100$ USD and with top-coded income or consumption. To estimate the MPCs we need a minimum of three periods, thus we drop households that appear fewer than three consecutive times. We restrict the sample to households with the head between 25 and 55 years old. We follow \cite{Blundell2014} in constructing the consumption measure. Our consumption measure accounts for food at home and food away from home, food stamps, utilities, gasoline, car maintenance, public transportation, childcare, health expenditures, and education. Our definition of household income the labor earnings of a household after-taxes plus government transfers.

With these data in our hands, we compute the MPCs for each state $s$ following the methodology proposed by \cite{Blundell2008} and \cite{Kaplan2010}. We first regress log income and log consumption on the interactions between the year dummies and education, race, and employment status. We also include dummies that control for family structure such as family size, number of dependent kids who live inside or outside the household, dummies for the cohort to which the head belongs, and the taxable income earned from other members of the household different from the head and its spouse. We then use the residuals from the first stage to construct a first-difference for the residuals in log consumption $\Delta c_{it}$ and for the residuals in log  income $\Delta y_{it}$. We assume the income process to be an error component model $y_{it} = \eta_{it} + \Delta \varepsilon_{it}$ with $\eta_{it}$ being a permanent shock, and  $\varepsilon_{it}$ a transitory shock. Finally, the MPC that corresponds to the estimator of the transmission coefficient of transitory income shocks to consumption, is computed as:
\begin{equation*}
    \widehat{MPC}_{s} = \dfrac{cov(\Delta c_{it}, \Delta y_{i,t+1})}{cov(\Delta y_{it}, \Delta y_{i,t+1})}
\end{equation*}
with the $\widehat{MPC}$ consistently estimated if households have no foresight, or no advance information, about future shocks. The estimator is implemented by an IV approach in which $\Delta c_{it}$ is regressed on $\Delta y_{it}$ instrumented by $\Delta y_{it+1}$. To get more precise estimates at state-level, we restrict our analysis to states that have at least $100$ distinct households used in the estimation.\footnote{We also exclude three states for which MPCs are estimated to be non-positive.} 

Figure \ref{fig:scatter_spendmpc} in \ref{sec:app_empres} shows the scatter plot between the state MPCs and the average state government spending by category. The four panels highlight a negative relationship between MPC and government spending implying high-MPC states in which the fiscal policy should have a larger multiplier effect receive on average less spending. The results also show that there is no significant difference between the three categories of spending. If MPC was a major driver of the differences in fiscal multipliers across categories of spending, we would have likely observed different patterns across spending categories. These scatter plots suggest that's not the case.

To further verify the previous point, we complement our investigation by rerunning the specification \eqref{eq:fm_comp} for two sub-samples: the states with high MPCs and the states with low MPCs.\footnote{We define the $14$ high-MPC states as the states with MPCs above the average of our sample of states. Similarly, we denote the $13$ low-MPC states as the states with MPCs below the average.} Results are reported in Figure \ref{fig:mpc}. The blue line reports the estimates for states with MPCs below the average, while the red line describes states with MPCs above the average. As the estimation only includes $27$ states, the estimates become more imprecise, particularly the estimates for R\&D and services spending at a longer horizon, than the ones reported from the previous estimations. Nevertheless, the results provide some important insights.


First, the local fiscal multipliers for spending in goods are non-significant at any horizon for both high- and low-MPC states. This implies the zero fiscal multiplier cannot be explained by the geographic allocation of spending in goods to states with low MPCs. That's the case because the estimates for both sub-groups of states generate the same patterns of a zero multiplier effect. Second, while we estimate the local fiscal multiplier due to a spending in services to be significantly positive and large only in state with high-MPC, the spending in R\&D generates a positive and significant effect only in states with low-MPC at late horizons. These results is consistent with the intuition that states with high MPC have a greater share of hand-to-mouth households. On the one hand, as a income shock occurs in the form of a government spending in services, hand-to-mouth households immediately react by significantly increasing their consumption. On the other hand, spending in R\&D mostly aims to produce new technologies and improve productivity. As a consequence, the multiplier effects come later in time and it is more likely to advantage households with lower propensity to consume and more willing to wait for the realization of the economic benefits.

Overall, geographic differences in the allocation of types of spending and in the propensity to consume cannot explain the differences in the multiplier effects among the three categories of spending.

\end{comment}

%%%%%%%%%%%%%%%%%%%%%%%%%%%%%%%%%%%%%%%%%%%%%%%%%%%%%%%%%%%%%%%%%%%%%%%%%%%%%%%%%%%%%%%%%%%%%%%%%%%%%%%%%%
%%%%%%%%%%%%%%%%%%%%%%%%%%%%%%%%%%%%%%%%%%%%%% CONCLUSIONS  %%%%%%%%%%%%%%%%%%%%%%%%%%%%%%%%%%%%%%%%%%%%%%
%%%%%%%%%%%%%%%%%%%%%%%%%%%%%%%%%%%%%%%%%%%%%%%%%%%%%%%%%%%%%%%%%%%%%%%%%%%%%%%%%%%%%%%%%%%%%%%%%%%%%%%%%%

\section{Conclusions}
\label{sec:conclusion}

The Great Recession renewed the interest on the effectiveness of fiscal policy as a countercyclical policy tool. Most of the studies estimate cross-sectional fiscal multipliers that conclude that fiscal spending crowds in local economic activity. A new theoretical work suggest that local economic characteristics and the specific features of the fiscal policy matter for the size of the fiscal multiplier. While an nascent empirical literature shows the amplification role that local economic characteristics () has on the size of fiscal multipliers, much less is known about the role of the specific features of the fiscal spending package. The contribution of this paper is to show how the amplification role of the composition of government purchases affect the size of the fiscal multiplier. 

In particular we show that purchases of services (R&D and non R&D) have an amplification effect compared to the purchase of goods. We find that the two-year  employment and earnings multipliers are lower when government purchase goods compared to either purchase of services or R\&D activities. In particular, the employment multiplier of goods (i.e. 0.09) is about one fifth of the value of the employment multiplier for spending on R\&D (i.e. 0.57) and about one third of the value of for the employment multiplier on services (i.e. 0.33). The difference between goods and the other spending categories (services and R&D) are larger when analyzing earnings multipliers which may suggest that part of the demand shock translate into increases in factor prices.   

We find that the difference in the response of economic activity to the type of spending is associated with the intensity of labor used to produce goods compared to the one used to compare services. We rule out that the higher presence of geographical spillovers in the production of goods is the factor driving our results. Also we rule out that the results are explained by higher elasticity of firm entry in the service sector compared to the sector that produce goods because, which can be rationalize if sectors who produce goods have higher entry cost than sectors that produce services. Still other important channels like input-output linkages, demand-induced productivity and consumption remain to be tested. 


%by higher firm entry of small and young firms in the  presence of trade linkags in goods The main identification threat of cross-sectional multipliers is that they do not account for geographical spillovers effects, which may be much more important when we consider spending on goods and R\&D. \cite{} use aggregate military spending data and find that geographical spillover are positive and decay rapidly with distance. A positive spillover implies that our local multipliers are biased towards zero and therefore our estimates should be considered as a lower bound. We rule that heterogeneity in the size of the spillover is the main factor driving our estimates. Still a deeper analysis on how spillovers differ by type of spending is needed and it is part of our research agenda. 

%Overall the A related question is how the effectiveness of public investments in creating jobs compares with the job creation of other major tools of fiscal policy, like direct transfers to households or tax cuts. Although Parker et al. (2013) show that government transfers have a high marginal propensity to consume, there is little evidence on how this translates into employment gains in the short run. However, given that job creation is a major policy objective, it is important for policy makers to know which of their tools are most suitable for achieving it. By evaluating the effectiveness of one specific policy, countercyclical investments, this paper takes a first step towards answering this question. More research is needed to inform policy makers about the employment effects of other policy tools at their disposal.



\end{document}