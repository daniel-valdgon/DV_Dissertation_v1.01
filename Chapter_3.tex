\documentclass[dv_diss_main.tex]{subfiles}

\begin{document}

\section{Introduction}
\label{sec:into}


Policymakers and academics know that a one-dollar increase in local public spending has positive effects on output, earnings and total employment~\cite{chodorow2019geographic}. However, the consensus that fiscal policy crowds-in private economic activity\footnote{The two exceptions in this literature are \cite{Dupor2017} and \cite{cohen2011powerful}, the former finds a multiplier lower than one while the latter finds that politically-motivated spending shocks shrinks private economic activity.}
contrasts with a relatively large heterogeneity across the different point estimates. Understanding the sources of this heterogeneity is key for the optimal design of a fiscal stimulus package. A growing literature study the role of local economic characteristics on the subjacent heterogeneity of the estimates. This literature highlights how the economic cycle~\citep{cohen2011powerful, serrato2016estimating,buchheim2020job}, trade openness~\citep{corbi2019regional}, households and firms liquidity constraints~\citep{Demyanyk2019,auerbach2020effects, bruckner2014local}, population demographics~\citep{basso2021young} and firm-size distribution \citep{Juarros2021} amplify the response of local economic activity to public spending. However, much less is known about the role of spending composition, understood as the basket of products purchased by the government when implementing a fiscal stimulus package, on the size of the fiscal multiplier.\footnote{The research on the effects of spending composition has been mostly focus around the size of infrastructure multipliers~\citep{Boehm2020,ramey2020macroeconomic}. \cite{garin2019putting}, \cite{buchheim2017employment} and \cite{Valderrama} are empirical studies that estimate the impact of fiscal policies that mostly increase infrastructure spending. None of these studies test other types of spending simultaneously.} 


This paper studies whether the size of the local employment and earnings multipliers depends on the products acquired through government purchases? To answer this question, we build a panel dataset on military spending at the product-MSA level and estimate the effect of the public spending composition on employment and earnings. We find that the content of the government's purchases has an important role in explaining the heterogeneity in the estimates of fiscal multipliers. Employment and earnings multipliers are larger when public spending increases the demand for products produced by labor-intensive industries. 
 

The scarce research on this area may be explained by the lack of granular data on public spending that can be used in a quasi-experimental research design. The bulk of research on local fiscal multipliers has used variation in public spending coming from either the American Recovery and Reinvestment Act (ARRA) \citep{chodorow2012does,wilson2012fiscal,conley2013american,dupor20162009} or from the government purchases made by the Department of Defense (DoD)~\citep{Nakamura2014,dupor2017local,Demyanyk2019,Auerbach2020}. These studies estimate the local fiscal multipliers using aggregate spending, and do not delve into how their estimates are affected by the composition of the government spending by the lack of harmonized data at the product level. 

This paper harmonizes a rich contract-level public spending dataset that satisfies the two necessary conditions to answer our research question. First, it allows us to identify, at the product level, the composition of government purchases that take place in a specific MSA at a specific time.\footnote{This dataset allows us to track various details about each public dollar spent by the department of defense from 1966 to 2015 such as: i) identifiers and 4-digits NAICS code of the firm who received the contract, ii) type of product contracted (e.g., missiles, R\&D, office supplies, or cleaning services for military bases), iii) geographic location of the primary contractor or where the majority of the work has been performed, iv) duration and v) amount of the contract.}
We exploit this product level variation to compute aggregate spending measures at the MSA level for three aggregate spending categories that can be followed over time: goods, services, and R\&D.\footnote{The specific division of the spending categories is motivated by the guidelines of the procurement office of the Department of Defense. See examples of the products included in each category in the Annex~\ref{tab:cat_list}.} Second, it allows us to nest this product level variation into a well-known cross-sectional research design, proposed originally by~\cite{Nakamura2014}, that we explain in detail below.\footnote{Although this research design has become the gold standard in the literature of cross-sectional fiscal multipliers \citep{Auerbach2019, Demyanyk2019,basso2021young,dupor2017local, Juarros2021} it is not free of identification threats that may bias our coefficient. We describe those in-depth in our Section \ref{sec:emp_strategy}.}

To estimate the effect of the composition of public spending on local employment and earnings,\footnote{To match the concept of the fiscal multiplier, one would like to estimate the impact of spending on GDP. However, MSA GDP is not available for the years before 2001, and as a consequence, all the studies that focus on MSA or county-level data estimate local employment and earning multipliers~\citep{suarez2016estimating, Auerbach2019, Demyanyk2019}.} we implement an instrumental variables research design proposed by \cite{Nakamura2014}. This strategy exploits the heterogeneous sensitivity of MSA's defense spending to nationwide military buildups and drawdowns between 1980 and 2019. Where the MSA's sensitivity is a function of the historical-comparative advantage of that MSA in attracting military procurement contracts (from 1966 to 1980).

The identification assumption is that, conditional on time and MSA fixed effects, the MSA military-spending measured by our instrument is not associated with unobserved local economic or political factors that may affect our outcomes of interest.  In other words, this instrument is purge from variatio in MSA spending that comes from endogenous variation such as the federal government's interest to use defense spending as a countercyclical policy or as a resource to favor politically important areas.

We study the effects of different spending categories on economic activity in three steps. First, we test if our data replicates the estimates of previous studies. Since we cover a longer time period, this exercise works as a external validity test. This test is particularly important because most of the literature's estimates come from a context where local labor markets were strongly affected by the Great Recession~\citep{yagan2019employment} and by the import-competition shock due to the entrance of China to the WTO~\citep{autor2016importing}. In our benchmark specification, which estimates a two-year local fiscal multiplier,\footnote{
Two-year fiscal multipliers are the benchmark estimates in the literature for three reasons: first, are the policy-relevant time-frame for countercyclical policy packages. Second, it reduces the measurement error that comes from differences between the fiscal year, the unit of measurement of public spending, and the calendar year, the unit of measurement of economic activity. Third, it is affected, to a lesser extent, by any bias that results from not adopting a fully dynamic specification.
} 
we find that a one-dollar increase in public spending increases local earnings by 0.51 dollars. Also, we find that the increase of public spending by 1 percent of aggregate earnings increases local employment by 0.28 percent. Our estimates of the earnings multiplier are slightly larger than those found in the literature, while our employment multipliers lie in the same range as previous estimates \citep{Auerbach2019, Demyanyk2019, suarez2016estimating}. Also, similar to the literature on state-level multipliers \citep{Nakamura2014,dupor2017local} we find employment and earnings multipliers that are higher than one.\footnote{MSA-level estimates tend to be smaller than state-level estimates because the presence of geographical spillovers is higher at the MSA level. Also, the more granular the unit of observation (MSA vs. State), the higher the measurement error in total spending.}

Second, we estimate the effect of each type of spending on employment and earnings. We find that local multipliers are lower when the government purchases goods compared to a scenario in which the government purchases services or funds R\&D activities. In particular, the two-year employment multiplier of goods (0.09) is about one-fifth of the value of the employment multiplier for spending on R\&D (0.57) and about one-third of the value of the employment multiplier on services (0.33). The difference between goods and the other spending categories (services and R&D) is larger for the case of earnings multipliers than employment multipliers, which suggests that part of the demand shock translates into wage increases.   

In the third part of the paper, we investigate which characteristics of the spending categories (goods, services, and R\&D) can rationalize the results we obtain in employment and earnings multipliers. We find suggestive evidence that the difference in labor-intensity across industries could be an important factor driving our results. To test this hypothesis, we assign a specific measure of labor intensity to each contract, and divide further each spending category into a labor-intensive and a non-labor intensive sub-components (for example, goods labor-intensive vs. goods non-labor intensive).\footnote{ 
We measure labor intensity at the 2-digit NAICS level as the ratio between the payroll and value-added. We assign this measure to each contract for which we observe the NAICS industry of the contractor.} Then, we estimate the earnings and employment multipliers for the new spending sub-components. The results confirm that government purchases of products made by labor-intensive industries lead to higher employment and earnings multipliers.

\textbf{Related literature.} This paper contributes to three strands of the literature. First, it contributes to the literature that estimates MSA-level employment and earnings multipliers in US~\citep{suarez2016estimating, Demyanyk2019, Auerbach2019, Auerbach2020, Juarros2021}. Due to data limitations, most of this literature has used data on military spending from 2001 onward. We contribute to this literature by building a dataset that goes back to 1966, which allows us to test the external validity of what is known about MSA-level multipliers.\footnote{\citep{Cox2021} build a contract level dataset similar to ours using data from USAspending.gov. The main difference with our dataset is that the authors measure all type of procurement contracts from 2001 onward, while our data measure only the contracts from the DoD from 1966 to 2015.}  

Second, our study contributes to the literature that asks about the amplification mechanisms of fiscal policy. This literature has mostly focused on explaining how characteristics of the local economies where the stimulus takes place matter for the size of the local fiscal multiplier. This research suggests that the economic cycle (e.g., recessions vs. expansions) ~\citep{cohen2011powerful, serrato2016estimating,buchheim2020job}, the level of household and firm liquidity constraints~\citep{Hagedorn2019, Auclert2018, Demyanyk2019,auerbach2020effects, bruckner2014local}, the demographic distribution of the population~\citep{basso2021young}, the share of small and young firms~\citep{Juarros2021}, and the trade openness~\citep{corbi2019regional} are important factors behind the size of the local fiscal multiplier. A notable exception to this literature, is the studies that focus on the size of infrastructure multipliers \citep{Boehm2020, buchheim2017employment,garin2019putting,ramey2020macroeconomic}. Similar to the infrastructure multipliers literature, our paper shifts the attention from the characteristics of the local economy to the characteristics of the fiscal policy itself. Additionally to this literature, we focus on difference in the composition of government spending within the broad category of government  consumption. 

Our third contribution is to the literature that focuses on how sector heterogeneity matters for the aggregate economic impact of demand shocks. \cite{vom2022investment}, and \cite{bouakez2020government} use a general equilibrium model with network effects to show that the economic sector where a demand/fiscal shock originates matters for its aggregate impact.\cite{vom2022investment} argues that this is explained by sectoral heterogeneity in the propensity to invest. In contrast, \cite{bouakez2020government} suggests that network centrality in an input-output economy is the primary amplification mechanism. \cite{Alonso2017} is the closest paper to our study. This study uses a GE model and causal estimates of the elasticity of consumption to job losses to argue that the composition of consumption matters for the impact of shocks on aggregate economic activity. Our contribution to this literature is that we directly measure the composition of fiscal policy, and therefore we can implement a horse race exercise that estimates fiscal multipliers at the product level.   






%%%%%%%%%%%%%%%%%%%%%%%%%%%%%%%%%%%%%%%%%%%%%%%%%%%%%%%%%%%%%%%%%%%%%%%%%%%%%%%%%%%%%%%%%%%%%%%%%%%%%%%%%%
%%%%%%%%%%%%%%%%%%%%%%%%%%%%%%%%%%%%%%%%%%% EMPIRICAL STRATEGY  %%%%%%%%%%%%%%%%%%%%%%%%%%%%%%%%%%%%%%%%%%
%%%%%%%%%%%%%%%%%%%%%%%%%%%%%%%%%%%%%%%%%%%%%%%%%%%%%%%%%%%%%%%%%%%%%%%%%%%%%%%%%%%%%%%%%%%%%%%%%%%%%%%%%%

\vspace{0.15in}


\section{Empirical Strategy}
\label{sec:emp_strategy}

\subsection{Specification}

Our empirical strategy builds on the works of \cite{Nakamura2014}, \cite{Dupor2017}, and \cite{Auerbach2020}. We exploit variation in defense spending across time and localities (MSA and State) to
estimate local employment and earnings multipliers. The benchmark specification for the period $k$ is defined as follows:
\begin{equation}
    \frac{v_{l,t+k} - v_{l,t-1}}{v_{l,t-1}} = \beta^k \frac{G_{l,t+k}-G_{l,t-1}}{Y_{l,t-1}} + \alpha_l^k + \delta_{t+k} + \varepsilon_{l,t+k}
    \label{eq:fm_base}
\end{equation}
\noindent where $v$ is a outcome of interest (employment or earnings) in location $l$ at horizon $t+k$ with $k =\{0, 
\dots, 4\}$. Our endogenous variable is $G_{l,t+k}-G_{l,t-1}$, it measures the change in military spending, which is normalized by aggregate earnings---$Y_{t-1}$. The specification includes a set of locality fixed effects---$\alpha_l$, to control for locality-specific trends; also, it includes time fixed effects---$\delta_{t+k}$, to account for any mechanical correlation between secular trends in military spending and unobserved macroeconomic shock. Since error terms---$\varepsilon_{l,t+k}$, are correlated over time, we cluster standard errors at the locality level.\footnote{ \cite{Auerbach2020} makes a case to cluster standard error at the state level. Our results (available upon request) remain unaffected by adopting their clustering decision.} 

The coefficient $\beta^k$ quantifies the employment and earnings multipliers in a window of $k$ years. Since we normalize total spending by total earnings, the earnings multipliers correspond to the dollar amount of aggregate earnings produced by an dollar increase in government spending. This interpretation change for the case of the employment multipliers because government spending is not normalized by employment. In the case of the employment multiplier the interpretation is the change in the growth rate of employment produced by a 1 percentage point increase in government spending measured in earnings percentage points.\footnote{\cite{chodorow2019geographic} shows that there is a mapping between the employment and the output multiplier and both estimates are strongly associated.}


In equation \eqref{eq:fm_comp}, we augment the standard specification to examine the heterogeneous effects of each government spending component:
\begin{equation}
\begin{split}
    \frac{v_{l,t+k} - v_{l,t-1}}{v_{l,t-1}} = & 
    \beta_{g}^k \frac{G^{g}_{l,t+k}-G^{g}_{l,t-1}}{Y_{l,t-1}} 
    + \beta_{s}^k \frac{G^{s}_{l,t+k}-G^{s}_{l,t-1}}{Y_{l,t-1}} 
    + \beta_{rd}^k \frac{G^{rd}_{l,t+k}-G^{rd}_{l,t-1}}{Y_{l,t-1}} 
    \\ & + \alpha_l^k + \delta_{t+k} + \varepsilon_{l,t+k}
    \label{eq:fm_comp}
\end{split}
\end{equation}
\noindent where $g$, $s$ and $rd$ are three different types of military spending, such that $\sum_{i=g,s,rd}G^{i}_{l,t}=G_{l,t}$. 

It is important to mention that using a two-way fixed-effect model implies that the coefficient of interest is estimated with both within locality variation over time, and within time variation across localities \citep{kropko2020interpretation}.\footnote{This implies that the interpretation of our estimate is how employment/earnings growth rate deviate from either its MSA-mean or its year-mean when spending deviate from its MSA-mean or its year-mean, see \cite{kropko2020interpretation} }^{,}\footnote{Also, \cite{kropko2020interpretation} suggest that the estimates could not be identified when either the within-case slopes or the within within-time slopes have low variance. This concern does not apply to our case since the puzzle of the fiscal multiplier literature is the substantial heterogeneity of the empirical estimates.}^{,}\footnote{
 \cite{kropko2020interpretation} suggests that two-way fixed effects models produce the most efficient and unbiased estimate when the population equation is similar to that specification. The tradition in macroeconomic theory is that both MSA characteristics and macroeconomic shocks determine the long-run path of the economy. We opted for a two-way fixed effect model because there is no clear argument of why one may think that local economic growth is driven by only aggregate macroeconomic shocks or only MSA level characteristics. Moreover, our estimates remain unaffected when we do not control for MSA fixed effects, and are severely affected if we do not control for time-fixed effects.}

\subsection{Instrumental Variables Design}

As highlighted in the previous literature, military spending is potentially endogenous due to political or economic factors. The observed military spending might respond to unobserved economic shocks if, for example, the federal government may use military procurement as a countercyclical policy tool. Also, political characteristics like firms' lobbying or local elections may affect the amount of DoD's spending received. 

To obtain causal estimates of the fiscal multipliers, we follow the instrumental variable (IV) research design proposed by \cite{Nakamura2014}. The instrument is obtained by combining the nationwide changes in military procurement with a measure of historical-comparative advantage that certain MSAs have in obtaining contracts from the DoD. This instrument, $Z_{l,t+h}$ , is defined by the following equation:
\begin{equation}
    Z_{l,t+h} = s_{l} \frac{G_{t+k}-G_{t-1}}{Y_{l,t-1}}
    \label{eq:fm_iv}
\end{equation}
\noindent where $s_{l}$ is the average share of spending in location $l$ and captures the comparative advantage that locality $l$ has in obtaining military outlays. The instrument for equation~\eqref{eq:fm_comp} are defined below:
\begin{equation}
    Z_{l,t+h}^g = s_{l}^g \frac{G^g_{t+k}-G^g_{t-1}}{Y_{l,t-1}};\;\;Z_{l,t+h}^s = s_{l}^s \frac{G^s_{t+k}-G^s_{t-1}}{Y_{l,t-1}};\;\;Z_{l,t+h}^{rd} = s_{l}^{rd} \frac{G^{rd}_{t+k}-G^{rd}_{t-1}}{Y_{l,t-1}}
    \label{eq:fm_iv2}
\end{equation}
\noindent as we investigate the effect of military spending at a higher disaggregation level than in previous literature, the volatility of government spending by components in locality $l$ is higher than the volatility of the total spending. Hence, we construct the predetermined average shares in locality $l$ by using the first fifteen years of data, from 1966 to 1980. Finally, since our empirical specifications are in changes and we use locality fixed effects, the conditional variation that is provided by the instrument comes from national-level changes in military spending, which is plausibly exogenous to local shocks that affect the outcome variables.

\subsection{Identification Assumptions and Threats}

Our identification strategy suffers from three main identification threats. The first is related to measurement error because of the prevalence of outsourcing in military procurement. A contract is assigned to its place of performance, defined as where the product is assembled or processed. Suppose sub-contractors outside the location of interest do the intermediate steps of the production. In that case, we would geographically misallocate the part of the outsourced military spending. This measurement error will create an attenuation bias in our estimates of the local fiscal multiplier, and therefore we should interpret them as a lower bound of the national fiscal multiplier. Moreover, this measurement error does not affect the relative size of the product-spending multipliers if it is not systematically correlated with a specific spending category. 

The second identification threat is the potential presence of geographical spillovers (GS). This would imply a violation of the SUTVA assumption.\footnote{This is particularly relevant when one thinks about the spillovers of R\&D spending. In the medium run, once the patent of the technology that is funded by public spending is implemented, non-treated MSAs are also affected.} In other words, it may be that defense spending in a specific MSA affects neighboring MSAs who did not receive spending at all. The bias that spillovers create in our estimates depends on the sign of this GS. Spillovers can be positive if public spending increases the demand for final-consumption goods or intermediate goods via input-output linkages. Also, spillover can be negative if the increase in spending affects factor prices, and as a consequence, affects the allocation of production factors across MSAs. For example, when spending attracts workers from non-treated geographies).\footnote{ \cite{chodorow2019geographic} shows that limited factor mobility ameliorates the aforementioned negative spillovers.} Reduced form estimates from \cite{Auerbach2019} suggest that geographical spillovers are positive. Therefore, we can safely assume that our estimates would suffer from a downward bias.

The third is related to the emerging literature that criticizes shift-share instruments\citep{goldsmith2020bartik,borusyak2022quasi,adao2019shift}. One can interpret this instrument as a shift-share instrumental variable (SSIV). In particular, an SSIV with several shifts  (e.g., as many as $G_{t+k}-G_{t-1}$ differences can be computed in the data), and one single share for each MSA (e.g., the time-invariant measure of comparative advantage).

\cite{borusyak2022quasi} shows that SSIV requires that either the shares~\citep{goldsmith2020bartik} or the shifts~\citep{borusyak2022quasi} are not correlated with unobserved characteristics that may affect MSA level outcomes. We argue that the shifts are a reasonable source of quasi-random variation because military buildups respond to international geopolitical events rather than to unobserved factors such as automation, trade competition, or national fiscal policy that could have a heterogeneous impact across MSAs. 

Assuming exogeneity of the shifts assures the validity of our instrument, even in the case that MSAs with a high comparative advantage in attracting DoD are in a different economic trend than MSAs with a low comparative advantage. Still, it is worth mentioning that our research design purges our instrument from the potential correlation between the shares and the unobserved local economic trends. Our outcome and instrument are both measured in changes. Since we include a locality fixed effect, we absorb any secular trend at the MSA level. This implies that the variation provided by our instrument consists of deviations in spending from its long-term trends, which is plausibly exogenous to local economic characteristics.


%%%%%%%%%%%%%%%%%%%%%%%%%%%%%%%%%%%%%%%%%%%%%%%%%%%%%%%%%%%%%%%%%%%%%%%%%%%%%%%%%%%%%%%%%%%%%%%%%%%%%%%%%%
%%%%%%%%%%%%%%%%%%%%%%%%%%%%%%%%%%%%%%%%%%%%%%%% DATA %%%%%%%%%%%%%%%%%%%%%%%%%%%%%%%%%%%%%%%%%%%%%%%%%%%%
%%%%%%%%%%%%%%%%%%%%%%%%%%%%%%%%%%%%%%%%%%%%%%%%%%%%%%%%%%%%%%%%%%%%%%%%%%%%%%%%%%%%%%%%%%%%%%%%%%%%%%%%%%

\section{Data}
\label{sec:data}

\subsection{Military Spending}
\label{subsec:data_mil}

We assemble new data on military procurement contracts awarded by the U.S. Department of Defense (DoD). Our data have unique advantages compared to previous studies \citep{Nakamura2014, Dupor2017, Auerbach2020}. We collect and harmonize military procurement contracts data from two sources: National Archives and Records Administration (NARA) for the period $1966$-$2006$, and USASpending.gov for the period $2007$-$2019$.\footnote{The data from USASpending.gov are available from $2001$. We use the period $2001$-$2006$ to validate the quality of the data collected from NARA.} The data from both sources are based on DD-350 and DD-1057 military procurement forms that accounts for about $96\%$ of contracts awarded by the DoD.

The data contain detailed information including the contract identification, the dates of action and completion, the transaction value, the location where the contract is performed, and the Federal supply classification code. There are two main advantages of our data compared to previous works. First, we have geographically disaggregated data at city level available for a long period. These two dimensions, long historical data and rich cross-sectional variation help us to better quantify the causal effect of government spending on economic outcomes. Second, the collected federal product classification enables us to allocate for each contract the amount of spending allocated to goods, services, and R\&D. That allows us to study the heterogeneous fiscal multipliers by components of government spending.

We construct and aggregate the military spending in the following way. We define the year in which a contract is approved as the year of the signature date that is the date when a contract is either awarded or modified.\footnote{The government fiscal year has been defined from October $1^{st}$ to September $30^{th}$ since $1976$. The mismatch between the fiscal year of the government and the calendar year could cause a time inconsistency between the the military spending and the other economic variables. Thus, we use the calendar year as reference year.} The contract completion date corresponds to the delivery date of the requested tasks. The contract dollar value is reported in nominal term. For comparability over time, we convert the nominal transaction value into real values by using the Consumer Price Index from the US Bureau of Labor Statistics.

The data also include modifications to existing contracts. Some modifications consist in downward revisions to contract amounts reported as negative entries.\footnote{For most years, the contract value is reported as an alphanumeric code. The last digit identifies whether the contract is an obligation or a de-obligation. We use the contract dictionaries to decode the alphanumeric strings into numeric values.} We follow \cite{Auerbach2020} and consider contracts with obligations and de-obligations with magnitudes within $0.5\%$ of each other to be null and void. 


\iffalse
We follow two approaches to allocate the military spending for a contract across years. The first approach consists of assigning the entire value of the contract to the year in which the contract has been signed. The second approach follows \cite{Auerbach2020} and it smooths the allocation of the contract value over the duration of the contract, computed as the period between the signature date and the completion date.\footnote{As in this second approach, we need to calculate the period passed between the signature and the completion dates, we remove contracts with missing completion dates or with completion dates before the signature dates.} As the empirical analysis is carried at the annual frequency, we then aggregate the value of a contract by the years covered between the signature and the completion years.
\fi 

The collected data also include the Federal supply classification code is an alphanumeric code and it proxies for the type of product that the contract is requested to deliver. The product code consists of 4-digits, and there are over $1500$ 4-digit products. As argued by \cite{Draca2013}, although some product codes are added or deleted over time, the classification is consistently defined over the years. That makes possible to compare product codes over time. We classify contracts based on the first digit of the code that corresponds to macro-category of the type of product requested by a contract. If the first digit is \textit{A}, then the spending is on ``Research and Development.'' If the first digit is any other letter different from \textit{A}, then the spending is on ``Services.'' If the first digit is a number, then the spending is on ``Goods.'' \ref{tab:cat_list} reports the major product codes included in each of the three categories.

Finally, our empirical strategy presented in the previous section exploits the geographic variation in military outlays. As standard in the literature, the geographic allocation is based on the location of the firm that performs the tasks of a contracts. The detailed location information permits us to geolocate contracts in narrow geographic areas. The available information differs between our two sources of data, NARA and USASpending. On the one hand, in the USASpending we know the city and the zip code of the performing firm.\footnote{Following  \cite{Demyanyk2019}, if we know that a contract has been performed in the US, but we do not know the exact location where it was performed, we assign the location of the contract recipient as the performance location. Notice that, differently from \cite{Demyanyk2019}, we adjust a marginal share of contracts. That's the case because we locate contracts not only using the postal code, but also the city. There are contracts for which information about the postal code is missing, but the information about the city is not missing.} We use these two pieces of information to identify the county in which a firm is operating. On the other hand, in NARA, the county in which a firm is performing the contract tasks is reported directly. We then use the spatial crosswalks provided by the National Bureau of Economic Research (NBER) to aggregate the county-level military contracts into Core-Based Statistical Areas (CBSAs) that are the geographic aggregation used in our analysis. As $97\%$ of the government military spending is concentrated in Metropolitan Statistical Areas (MSAs), we restrict our analysis only to these geographic agglomerations.

In \ref{sec:app_data}, we test the comparability of our data with the ones from previous studies. Overall, these tests validate the high-quality and comparability of our data with respect to data previously used in the literature.

\subsection{Economic Outcomes}
\label{subsec:data_eco}


Due to data limitations, gross domestic product by MSA is reported only from the beginning of the 2000s. As our analysis employs historical data, we use to quantify the effect of local spending shocks on economic activities two measures. The first measure is the employee income (pre-tax earnings) that includes salaries and wages, bonuses, stock options, profits, and some fringe benefits. The second measure is the employment that consists in the headcount of employed workers. These data are collected annually from the Bureau of Economic Analysis at MSA-level.\footnote{Both employment and employee earnings are derived from the Bureau of Labor Statistics’ Quarterly Census of Employment and Wages (QCEW).} We collect  employee earnings in nominal terms, and, for consistency with the other monetary variables, we convert them in real terms by using the Consumer Price Index from the US Bureau of Labor Statistics.



\subsection{Sample Definition}
\label{subsec:data_sam}


We apply some additional filters to the data aggregated by MSA to avoid the estimate to be driven by outliers. First, we exclude MSAs with incomplete histories in any of the variables described above (military spending, earnings, and employment). We remove MSAs with an average population over the period of analysis smaller than 50,000 inhabitants. We also exclude MSAs in which at least in one period the ratio of military spending to earnings is greater than $1.5$. We also drop MSAs with earnings or employment growth rates between two consecutive periods either greater than $1$ or smaller than $-0.5$. To avoid to have a zero instrument for any categories of spending, we remove MSAs that have zero spending in any of the three types of spending over the period $1966-1980$. Finally, as in previous studies, \cite{Auerbach2020} and \cite{Demyanyk2019}, the analysis is at locality aggregated level rather than per capita.








%%%%%%%%%%%%%%%%%%%%%%%%%%%%%%%%%%%%%%%%%%%%%%%%%%%%%%%%%%%%%%%%%%%%%%%%%%%%%%%%%%%%%%%%%%%%%%%%%%%%%%%%%%
%%%%%%%%%%%%%%%%%%%%%%%%%%%%%%%%%%%%%%%%%%% EMPIRICAL RESULTS  %%%%%%%%%%%%%%%%%%%%%%%%%%%%%%%%%%%%%%%%%%%
%%%%%%%%%%%%%%%%%%%%%%%%%%%%%%%%%%%%%%%%%%%%%%%%%%%%%%%%%%%%%%%%%%%%%%%%%%%%%%%%%%%%%%%%%%%%%%%%%%%%%%%%%%


\section{Descriptive Statistics}
\label{sec:des_stats}


As the data are part of the novelty of our study, we briefly discuss their main features. Figure \ref{fig:share_comp} reports the time series of military spending computed by using the full universe of military procurement contracts over the period $1966-2019$.\footnote{The sample consists of more than $20$ millions contracts.} 

Panel A of Figure \ref{fig:share_comp} shows the evolution of aggregate military spending deflated by the CPI. We observe three sharp rises in spending. The first increase is in the 1980s as a consequence of the Reagan military buildup, the second is in the early 2000s due to the Afghani and Iraqi wars, and the last is in the recent years due to the escalations in military buildups with Russia and China.

Panel B of Figure \ref{fig:share_comp} reports the shares of military spending allocated to each of the three spending categories: goods, services, and R\&D. There are two main takeaways. First, the largest share of government spending is directed to the purchase of goods, followed by the purchase of services, and finally, by the investment in research and development activities. As reported in the first row of Table \ref{tab:desstats_contracts}, over the period $1966$-$2019$ the average share of spending in goods, services, and R\&D are $54\%$, $31\%$, and $15\%$, respectively.

Second, starting in the 1980s we observe a reallocation of government spending from goods to services. While at the beginning of the 1980s the spending in goods was about $60\%$ of the value of the military procurement contracts compared to $20\%$ in services, at the end of the 1990s the two shares were both around $40\%$. The share of spending in R\&D has remained more constant around $0.2$ until the beginning of the 2000s. From the 2000s onward, we document a drop in R\&D spending of about $50\%$, from $0.16$ to $0.08$, in favor of spending in goods.

Due to the filters described in the previous section, the empirical analysis includes $296$ MSAs. We exclude all procurement contracts awarded to micropolitan statistical areas and rural counties. The $296$ MSAs included in the sample account for a large share of the military procurement spending. Our sample includes $90\%$ of both the total number of contracts and the aggregate spending. These shares are greater than $90\%$ for spending in goods and R\&D. Not surprisingly, due to a less degree of tradeability, services are less geographically concentrated. Indeed, our sample accounts for about $80\%$, instead of $90\%$, of the aggregate spending in services.\footnote{Figure \ref{fig:share_comp_sample} in \ref{sec:app_empres} replicates Figure \ref{fig:share_comp} only for the contracts awarded to the MSAs in our sample. The patterns are substantially consistent with the ones described above.} Finally, per capital employee earnings in the MSAs from our sample are a $20\%$ greater than the average earnings in the US.

Let us now explore the heterogeneity across military contracts awarded to MSAs included in the sample for the empirical analysis.\footnote{Statistics computed for the full universe of procurement contracts provide the same insights reported here for the restricted sample.} Table \ref{tab:desstats_contracts} reports some basic descriptive statistics on the contract characteristics by category of spending. 

The table suggest that $90\%$ of procurement contracts are signed by the government to purchase goods from the private sector. Only $8.5\%$ of contracts is awarded to provide services, and even a smaller share, about $1.5\%$, requires R\&D activities. The third row of Table \ref{tab:desstats_contracts} show that the average outlay in R\&D component per contract is four times the average spending across all contracts; and that the average spending in goods per contract is smaller than the average spending over all contracts. 

The last two rows of Table \ref{tab:desstats_contracts} explore the distributional characteristics of spending within categories. The distribution of contracts for the purchase of goods has a significantly fatter right-tail than the distributions for spending in services or R\&D. In the case of spending for goods, the contract at the top decile of the distribution of outlays is almost $40$ times greater than the median contract. The $90\%$-to-$50\%$-percentile ratios are significantly smaller for spending in services and R\&D, with the contract at the top decile of the distribution to be $12$ times greater than the median contract for services and $8$ greater for R\&D. 

The last row of Table \ref{tab:desstats_contracts} shows the share of spending allocated to contracts in the top decile. As one can notice, the top $10\%$ of contracts awarded to purchase goods accounts for $99\%$ of the total spending for goods. This result implies that although the majority of contracts have a component of spending for goods, only relatively few matter in size. The share of spending in services and R\&D allocated to the top decile is around $80\%$ implying a more equal allocation of spending across contracts.  

These results provide some insights on the government procurement process. R\&D spending is awarded through few but large contracts and the dispersion across these contracts is relatively small. The purchase of goods occurs through a large number of contracts of which only few of them account for a sizeable monetary value. Finally, spending in services is in between the two previous cases.

Before turning our attention to the regression results, as our identification strategy exploits the cross-sectional variation in spending, it is worth to explore the geographic heterogeneity in the allocation of outlays across the MSAs in our sample. First, military spending is unequally distributed across MSAs. Figure \ref{fig:map_spend} shows the quartile to which a MSA belongs based on the average value of the military spending that it has receive over the period $1966-2019$. There are two main results to highlight. As showed in Panel A, most of the MSAs in the top quartile are located along the two coastal regions, and the Midwest. 

The visual inspection of the remaining three panels of Figure \ref{fig:map_spend} suggests that there are differences in the geographic allocation among the three types of spending. These differences are particularly marked in the Midwest. While most MSAs in the Midwest are in the top two quartiles of spending in goods, only few of these MSAs are ranked as high in the distribution of spending in services. These results are consistent with the geographic economic structure of the US. The Midwest has the highest concentration of production occupations with a average employment share in production jobs almost $50\%$ higher than the US average. Only $14\%$ of MSAs are in the top quartile in all categories of spending. These shares remain similar if we compare pairs of types of spending.\footnote{$14\%$ of MSAs are in the top quartile in goods and services spending. $16\%$ of MSAs are in the top quartile in R\&D and service spending. $19\%$ of MSAs are in the top quartile in R\&D and good spending.} The differences in the production structure between regions also reflect in the government allocation of types of spending.

Table \ref{tab:desstats_cbsas} reports a set of descriptive statistics on the distribution of military spending across the MSAs in the sample. The figures emphasize that the spending is unequally distributed across MSAs. The $90\%$-to-$50\%$-percentile ratio is $22$ for good spending, $18$ for service spending, and even $77$ for R\&D spending. The second row highlights between $55\%$ and $65\%$ of the spending in each category is awarded to the top decile of the distribution of spending by MSA.\footnote{The top $10\%$ MSA receivers of R\&D spending receive $55\%$. In these MSAs is also located the primary address of the inventors of $55\%$ of granted patents by the USPTO.}


To sum up, the results provide evidence that although the DoD signs a large number of contracts, the largest share of government procurement spending is captured by a handful of contracts. Furthermore, these contracts are not equally distributed across the MSAs, but about $20-30$ MSAs receive over $60\%$ of the entire spending. These findings imply that the distribution of spending at both contract and MSA level is significantly skewed to the right. 

Our findings are complementary to \cite{Cox2021}. They show that government spending is granular and concentrated among a few firms. Their analysis is based on the contract recipients. The granularity in recipients does not necessary imply a granularity in firms actually performing the tasks of a contract. It could happen that contract recipients allocate several tasks of a contract to different performers evenly located in the national territory. If that were the case, we would not observe a geographic concentration in the spending based on the place of performance. Our results show the opposite suggesting, in addition to a granularity in contract recipients, also a granularity in contract performers.\footnote{We cannot test directly this granularity in contract performers because the data do not report the name of the firm that performs a contract. We only observe the place of performance.} Finally, similarly to \cite{Cox2021}, we also show a substantial variation in the range of contract values. Our results also emphasize that there exist significant differences in the distributional features of contract size across categories of spending.

\section{Main Results}
\label{sec:emp_fm}


This section shows the main results of the paper. That is, the earnings and employment multiplier are higher for defense spending that demand R\&D and services compared to defense spending that demands manufacturing goods. Also we show the sensibility of our estimates to other specifications.


Before talking about the results, it is worth to highlight that our study provides estimates for the local employment and earnings multipliers. An important caveat in the literature of local fiscal multipliers, is that this estimates cannot be easily translated into a national fiscal multiplier for two reasons: First, the existence of spillover effects across localities. Second, the parametrization of macroeconomic models that map the local to the national fiscal multiplier generate a broad range of estimates. In the following of the paper, we will use the term local fiscal multiplier and fiscal multiplier interchangeably. 

Also we want to highlight that, as all other studies who study the impacts of fiscal policy at MSA level, we report earnings and employment multipliers rather than output multipliers. This decision respond to lack of output data at MSA level for our period of study.


\subsection{Results}

The first set of results correspond to the earnings and employment multipliers of aggregate military spending. Figure \ref{fig:fm_earnings} plots the  effect of military spending shocks on earnings at different horizons after the shock occurs. The shaded area represents the $90\%$ confidence interval. Panel A of Table \ref{tab:fm_main} reports the point estimates of aggregate spending using the specification presented in equation \eqref{eq:fm_base}. The estimates suggest that an increase in one-dollar of military spending increase local earnings by $0.21$ dollars. As the time passed, the effect of the military spending on earnings become larger, moving from $0.2$, on impact, to about $0.6$ four periods after the shock. Our estimates are slightly larger than the ones reported by the literature \citep{suarez2016estimating, Auerbach2019,Demyanyk2019}.

The second set of results correspond to the decomposition of military spending by categories. The estimates in each of the remaining panels of Figure \ref{fig:fm_earnings} refers to one of the terms in equation \eqref{eq:fm_comp}. The results suggest a substantial heterogeneity in the estimates of earnings multiplier by spending category. Panel B shows the effects of goods' spending. It shows that the local multipliers of good's spending are close to zero in magnitude and not statistically significant at any horizon after the occurrence of the shock. Panel C presents the earnings effect of shocks in services spending. The effect of one-dollar increase in service's spending is of $0.35$ dollars in total earnings. Aggregate earnings continue to increase as time after the shock passes, reaching an effect of $0.8$ 3-4 periods after the spending shock. Finally, Panel D tracks the impulse-response of R\&D spending. The estimates show that, on impact, earnings are only slightly affected, but after a few periods the earnings multiplier significantly increase up to around $3$ dollars per each $1$ dollar increase in R\&D spending.\footnote{We test whether the correlations between the endogenous regressors and the instruments are small. We compute the first-stage F-statistic, and we find the instruments do not suffer from the weak-instrument problems as the F-statistic is well above 10 in all specifications.} Finally, Table \ref{tab:fm_main} shows that the coefficients for each type of spending are statistically different from each other. The p-values of the null hypothesis that all types of spending have similar effects o the aggregate economic activity suggest a clear rejection. This highlights the main result of this paper. The assumption made by the previous literature, that all type of spending had equal impact on economic activity, was wrong. 

These results may suggest that spending shocks that demand services have more immediate response than spending shocks that demand R\&D. There are several plausible explanations for the lagged large effect of R\&D spending on local economic activity: First, R\&D spending may lead slow gains in firm productivity\footnote{The lagged productivity effect is consistent with the idea that innovation needs time to be developed and used in production, a phenomena termed as the productivity paradox \citep{Huggett2001,Acemoglu2014}.}. Second, it may be that the innovations funded by the R\&D spending lock-in future government purchases of goods that incorporate the novel technology or future government contracts that fund other R\&D projects. We explore this hypothesis in the Section~\ref{sec:emp_mec}.


We consider employment as a second measure of local economic activity. Figure \ref{fig:fm_employment} plots the estimates the effect of military spending shocks on employment at different horizons. Panel C and D of Table \ref{tab:fm_main} reports the point estimates. The results for the employment multipliers are qualitatively identical to the ones presented on the earnings multipliers. 

\subsection{Robustness}

One concern with our estimates is that the results capture short-term movements in population between localities due to government spending. We address this concern by re-estimating our main specifications with per capita earnings rather than aggregated at MSA-level. The results for this specification are reported in Table \ref{tab:fm_percapita} in \ref{sec:app_empres} in \ref{sec:app_empres}. The per capita estimates confirm the previous findings.

Our approach has excluded localities with incomplete military spending histories. It still could be the case that some MSAs have zero entries in any of the spending category. One concern is that the zero entries play a quantitatively important role in the estimation of the multipliers. We address this concern by further restricting the sample and dropping all MSAs with zero entries. The results for both earnings and employment multipliers are reported in Table \ref{tab:fm_restricted} in \ref{sec:app_empres}. Although the sample size significantly decreases to $113$ MSAs, the findings remain unchanged, implying that zero entries do not play an important role in our estimation.

An additional concern is that the specification ignores the dynamics of spending and employment for the years in between the period of study. For example, when the change in spending is measure as the difference in spending between the year $t+k$ and the year $t-1$, one ignores how spending change in the years $t\in(t-1,t+k)$. To circumvent this problem, we follow \cite{ramey2018government} and estimate a specification where the outcome and spending variables are defined as the cumulative change between $t-1$ and $t+k$. In other words, our dependent variable is defined as $\sum_{j}^{k}(\frac{v_{l,t+j}-v_{l,t-1}}{v_{l,t-1}})$, where $v$ is either employment or earnings. Similarly, our spending variable, (aggregate or spending by category) is defined as $\sum_{j}^{k}(\frac{G_{l,t+j}-G_{l,t-1}}{Y_{l,t-1}})$. Results are presented in table \ref{tab:fm_main_cum} in \ref{sec:app_empres}. We can see that all coefficients are quantitative similar to the ones presented in our main specification. 

\iffalse
Another valid concern is the timing of actual spending. In the data, we observe the start and end date of the contract. In the regressions presented in sub-section \ref{subsec:emp_fm}, we use the year in which a contract has been signed to allocate the spending, and not the years in which the disbursements actually occur. To alleviate this source of spending mismeasurement, we follow \cite{Auerbach2020} and we construct a flow spending measure for each contract by allocating the value of a contract equally over its duration. The results for both the earnings and employment multipliers are reported in Table \ref{tab:fm_smooth} in \ref{sec:app_empres}. ???????????????
\fi

To sum up the previous results, spending in services and R\&D increases economic outcomes. Spending on services has a medium-size effect immediately after the shock occurs. On the contrary, spending on R\&D produces a larger increase in economic activities, but this effect takes place several years after the spending took place. Surprisingly, spending in goods does not affect the multipliers at any horizon and in any of the specifications we tested.


%%%%%%%%%%%%%%%%%%%%%%%%%%%%%%%%%%%%%%%%%%%%%%%%%%%%%%%%%%%%%%%%%%%%%%%%%%%%%%%%%%%%%%%%%%%%%%%%%%%%%%%%%%
%%%%%%%%%%%%%%%%%%%%%%%%%%%%%%%%%%%%%%%%%% LABOR INTENSITY %%%%%%%%%%%%%%%%%%%%%%%%%%%%%%%%%%%%%%%%%%%%%%%
%%%%%%%%%%%%%%%%%%%%%%%%%%%%%%%%%%%%%%%%%%%%%%%%%%%%%%%%%%%%%%%%%%%%%%%%%%%%%%%%%%%%%%%%%%%%%%%%%%%%%%%%%%


\section{Mechanism: The Role of Labor Intensity}\label{sec:lab_int}


In this section, we present two pieces of evidence that support the idea that differences in the fiscal multipliers by type of spending are due to the different intensity of labor usage in the production of the products to which these types of spending are directed.


Two primary intuitions help to understand why the product labor-intensity amplifies the response of economic activity to fiscal policy: First, when public spending is directed to labor-intensive industries, it leads to labor-biased demand shocks that increase employment and wages. That is a direct effect that is particularly important during periods of slack in the economy. Second, local private consumption is more responsive to increases in employment and wages than to increases in investment and the returns to capital. This indirect effect is explained by the fact that labor rather than capital is disproportionately concentrated among hand-to-mouth workers. 

The first step to empirically test the role of of labor intensity is to assign a labor-intensity measure at the contract level. Our dataset does not contain any information on the amount of inputs used in the production. thus, we use an alternative strategy to quantify the labor intensity. We collect annual data from the BEA on value added and employees' compensation by industry.\footnote{The data contain $81$ NAICS codes starting from $1997$. Although valued added and its components are available from the BEA since $1988$, the industrial aggregation for years before $1997$ is not comparable with the one for the years after $1997$.} We then compute a measure of labor intensity as the contribution of employees to the value added. Finally, we assign the constructed measure of labor intensity to each contract based on the industry to which the contractor belongs. We remove from the sample contracts not linked to any industry. As we match contracts that amount for more than $98\%$ of the total value of military spending from $1997$, this restriction does not seem to affect our analysis. Table \ref{tab:indlabint_list} reports the classification of the available industries separated between low- and high-labor intensity. Our classification matches the common sense. Indeed, industries as healthcare, education, hospitality and food service are classified as high labor-intensive, while manufacturing and retail as low labor-intensive.

Figure \ref{fig:shlabint_comp} reports the evolution of the share of spending allocated to labor-intensive industries by type. We define an industry as labor-intensive if the average labor intensity in that industry is in the top half of the distribution of these averages. About three-quarter of the total military spending is directed to industries with high labor intensity. The share of spending allocated to labor-intensive industries significantly varies across types of spending. On average, $78\%$ of the spending in services goes to industries that rely more on employees. This share is about $8$ percentage points greater than the share allocated from spending in goods. Finally, over $91\%$ of spending in R\&D is directed to industries with higher labor intensity. These results present a first evidence on the importance of labor intensity in determining the difference in the fiscal multipliers across categories of spending. Indeed, the local fiscal multipliers are higher for types of spending whose larger share is allocated to more labor-intensive industries.

Intuitively, our main test to assess the role of labor intensity in determining the differences in the local fiscal multipliers across categories of spending consists in running the specification \ref{eq:fm_comp} considering separately the components of spending for each category that goes to high and low labor-intensive industries. Specifically, we run the following regression: 
\begin{equation}
\begin{split}
    \frac{v_{l,t+k} - v_{l,t-1}}{v_{l,t-1}}  = & \gamma_{g,H}^k \frac{G^{g,H}_{l,t+k}-G^{g,H}_{l,t-1}}{Y_{l,t-1}} + 
     \gamma_{s,H}^k  \frac{G^{s,H}_{l,t+k}-G^{s,H}_{l,t-1}}{Y_{l,t-1}} \\[0.1in] 
    & + \gamma_{g,L}^k \frac{G^{g,L}_{l,t+k}-G^{g,L}_{l,t-1}}{Y_{l,t-1}}
     + \gamma_{s,L}^k \frac{G^{s,L}_{l,t+k}-G^{s,L}_{l,t-1}}{Y_{l,t-1}}  
    \\[0.1in] 
    & + \gamma_{rd}^k \frac{G^{rd}_{l,t+k}-G^{rd}_{l,t-1}}{Y_{l,t-1}} + \alpha_l^k + \delta_{t+k} + \varepsilon_{l,t+k}
\end{split}
\label{eq:fm_comp_labint}
\end{equation}
where $G^{g,H}$ represents the spending in goods produced by high labor-intensive industries, and $G^{g,L}$ is spending in goods produced by low labor-intensive industries. Similarly,  $G^{s,H}$ and $G^{s,L}$ are the spending in services produced by high and low labor-intensive industries, respectively. In the case of spending in R\&D, $G^{rd}$, in the benchmark specification, we do not split it between high and low labor-intensive industries. This choice is motivated by the fact that, as showed in Figure \ref{fig:shlabint_comp}, over $90\%$ of the R\&D spending is directed to high-intensity industries. Including separately R\&D spending in high and low-intensity industries generates serious estimation issues because a large share of MSAs has several years of zero spending in R\&D directed to low labor-intensive industries.\footnote{Although not reported in the paper, as a robustness check, we estimate the model by splitting the R\&D spending between high and low-intensity industries. The results for good and services spending are robust, but the estimation of coefficients for R\&D spending directed to low labor-intensive industries is highly imprecise.} We instrument each regressor with instruments as defined in equation \eqref{eq:fm_iv} for each pair of type of spending and labor intensity.\footnote{As data on labor intensity by industry are available only from $1997$, we cannot construct the instruments for the period $1966-1980$ as above. Therefore, we construct the instruments using the entire available period $1997-2015$.}

Table \ref{tab:fm_labint} reports the impulse response functions estimated from specification \eqref{eq:fm_comp_labint}. Panel A reports the local fiscal multipliers split between government fiscal shocks directed to high and low labor-intensive industries. The positive effects on earnings come exclusively from government shocks directed to industries with high labor intensity. Indeed, the estimates for shocks to high labor-intensive industries are statistically significant at any time horizons and larger in size than the estimates from spending in low intensity industries.

We turn now to Panel B that explores the effects for the different categories of spending by labor-intensity usage. The first two rows of Table \ref{tab:fm_labint} show positive and statistically significant effects on earnings for spending in either services or goods in response to a government shock directed to industries that intensively use labor. In terms of magnitude, the estimates highlight that the effect of a shock to spending in services in high labor-intensive industries is about three times the impact of a shock to spending in goods in high labor-intensive industries. This fact, suggest that labor intensity may only explain part of the difference that we find between spending in goods and spending in services. We test some alternative explanations in the next section. 

Finally, the third and fourth rows of Table~\ref{tab:fm_labint} highlight that independently on the type of spending, in either goods or services, a fiscal government shock directed to low labor-intensive industries generate local fiscal multipliers that are not statistically different from zero at any time horizons after the shock. In terms of size, as for spending directed to high labor-intensive industries, the multipliers are larger for spending in services rather than in goods. Finally, the spending in R\&D generates a sizeable multiplier effect, but smaller than the ones previously showed in Figure \ref{fig:fm_earnings}.

These results provide support that the differences in the multipliers across categories of spending are partially explained by the differences in the production factor requirement that each type of product has. In particular, products that are produced by labor-intensity technologies have higher multipliers than products produced by non- labor intensive technologies. 









%%%%%%%%%%%%%%%%%%%%%%%%%%%%%%%%%%%%%%%%%%%%%%%%%%%%%%%%%%%%%%%%%%%%%%%%%%%%%%%%%%%%%%%%%%%%%%%%%%%%%%%%%%
%%%%%%%%%%%%%%%%%%%%%%%%%%%%%%%%%%%%%%%%%%%%%%%%%%% ALTERNATIVE MECHANISM %%%%%%%%%%%%%%%%%%%%%%%%%%%%%%%%
%%%%%%%%%%%%%%%%%%%%%%%%%%%%%%%%%%%%%%%%%%%%%%%%%%%%%%%%%%%%%%%%%%%%%%%%%%%%%%%%%%%%%%%%%%%%%%%%%%%%%%%%%%


\section{Ruling out Alternative Mechanisms}
\label{sec:emp_mec}


This section explores alternative mechanisms, proposed in the literature, that could explain the differences in the fiscal multipliers by category of spending. Overall, our empirical analysis highlights that none of these alternative mechanisms generates these differences.

\subsection{Sector Tradeability}\label{subsec:outflow}


Physical goods are more tradeable than services.\footnote{The average service industry is less tradeable than the average manufacturing industry. However service tradeability has been growing over time \citep{gervais2019tradability}.} If the production of goods occurs in the neighboring locations, then one could observe no effects of good spending on the local fiscal multiplier. If that were the case, differences in the fiscal multipliers would reflect geographic spillovers rather than the actual nature of the spending, and these spillovers would be more sizable for more tradeable products as physical goods.

This explanation could be relevant in our context due to the well-documented issues in correctly allocating government spending to localities. Our data only contain records on prime contracts and they do not reflect the amount of subcontracting for basic and intermediate materials and components. A contract is assigned to its place of performance defined as the place where the product is assembled or processed. If the intermediate steps of the production  are done by sub-contractors outside the location of interest, we would geographically misallocate part of the spending. In addition to that, the definition of the place of performance slightly varies across categories of products.\footnote{The location of the majority of manufacturing contracts reflect the location of the plant where the product is finally assembled or processed. The location of construction contracts corresponds to the location were the construction is performed. The location for contracts involving purchases from wholesale or other distribution firms reflects the location of the contractor's place of business. Finally, for service contracts, the location is the place where the service is performed, with the exception of transportation and communication services that report the location of the contracting firm.} The measurement errors in allocating contracts to MSAs could affect our results, and these effects could be heterogeneous across categories of spending.

We implement two test to quantify the importance of the geographic allocation of contracts in driving our previous results. The literature has extensively argued that the geographic misallocation of contracts becomes less important as one moves to larger geographic aggregation. Specifically, it becomes a minor issue at state-level. \cite{Isard1962} argues that geographic disaggregation at state-level does not contain significant measurement errors. \cite{Nakamura2014} use shipment data to the government from defense industries, reported by the U.S. Census Bureau from $1963$ to $1983$, and verify that, on average, the relationship between the prime contracts allocated to a state and the shipments from that state is one-for-one. These two studies imply that on average, all contracts that are allocated to a state are also performed in that state.

Our first test consists in comparing the estimates from the regressions using MSA-level data with the ones using state-level data. If the concern with the geographic allocation of spending to MSAs is not a main driver of our estimates, one should expect that the state-level analysis would lead to the same conclusions previously discussed. The estimates reported in Table \ref{tab:fm_state} of \ref{sec:app_empres} are qualitatively comparable to the ones presented in section \ref{sec:emp_fm}. The sizes of the effects at any horizons are much larger by using the state aggregation. This result is in line with the findings of \cite{Demyanyk2019} and \cite{Auerbach2020}. They argue the discrepancy between MSA-level and state-level estimates can be attributed to within-state subcontracting. In smaller geographic areas, it is more likely that part of the spending to spill into or from other geographic areas. Thus, the potential measurement error in subcontracting outside a MSA attenuates the estimates of the fiscal multipliers, implying that our MSA-level estimates can be considered as lower bounds. Even if we consider state-level estimates, we do not find any significant effect of government spending on goods on the fiscal multiplier, while the effects for service and R\&D spending remain significant at least at the $10\%$ level.\footnote{The significance of our estimates is lower in the state-level regressions than in the MSA-level analysis. That's the result of loss in power in the state-level regressions due to less cross-sectional variation in the military spending.} Once we account for the effect that subcontracting might have on our estimates, the main findings from the previous section remain unaltered.

Likely, MSAs that are along the state borders have strong economic interactions with other MSAs outside the state. In these cases, government spending allocated to locality $l$ could be used for production in some neighboring locations outside the state borders. Our previous test would not capture these cross-state ``outflows''. Our second test, instead, captures these interactions. We implement the following specification to investigate whether military spending shocks in location $l$ have some positive effect in the neighboring locations:
\begin{equation}
\begin{split}
    \frac{\tilde{Y}_{l,t+k} - \tilde{Y}_{l,t-1}}{\tilde{Y}_{l,t-1}} = & \tilde{\beta}_{g}^k \frac{G^{g}_{l,t+k}-G^{g}_{l,t-1}}{\tilde{Y}_{l,t-1}} + \tilde{\beta}_{s}^k \frac{G^{s}_{l,t+k}-G^{s}_{l,t-1}}{\tilde{Y}_{l,t-1}} + \tilde{\beta}_{rd}^k \frac{G^{rd}_{l,t+k}-G^{rd}_{l,t-1}}{\tilde{Y}_{l,t-1}} 
    \\ & + \alpha_l^k + \delta_{t+k} + \varepsilon_{l,t+k}
    \label{eq:fm_spill}
\end{split}
\end{equation}
\noindent where $\tilde{Y}_{l,t+k}$ is the total earnings for the neighboring locations of $l$, which are defined as those MSAs whose center is located within a $100$ miles radius distance from the center of $l$.\footnote{While \cite{Auerbach2020} defines neighboring locations as those MSAs within the same state.%The equal size within bandwith exerise is quite different
} Similarly, the instrument for each spending category is defined as it is indicated below by equation \ref{eq:fm_iv}. Keep in mind that now spending is normalized by $\tilde{Y}_{l,t-1}$. 
\begin{equation}
    Z_{l,t+h}^g = s_{l}^g \frac{G^g_{t+k}-G^g_{t-1}}{\tilde{Y}_{l,t-1}};\;\;Z_{l,t+h}^s = s_{l}^s \frac{G^s_{t+k}-G^s_{t-1}}{\tilde{Y}_{l,t-1}};\;\;Z_{l,t+h}^{rd} = s_{l}^{rd} \frac{G^{rd}_{t+k}-G^{rd}_{t-1}}{\tilde{Y}_{l,t-1}}
    \label{eq:fmspill_iv}.
\end{equation}
Results are showed in Figure \ref{fig:earn_spill}. Table \ref{tab:earn_spill} in \ref{sec:app_empres} reports the point estimates.\footnote{$12$ MSAs are excluded because they do not have any neighboring locations within $100$ miles. We also tested, without reporting in the paper, the effect of reducing or increasing the distance on the results reported in the paper. The size of the effects become bigger as distance increases. Nevertheless, the main takeaways remain unchanged.} Panel A shows that, on average the ``outflow'' effects are small and statistically non-significant. The remaining three panels present the results for each category of spending. Panel B shows the effect on neighboring locations of fiscal shocks in good spending. The finding highlights there are no spillover effect of a shock in locality $l$ on the neighboring localities. This result is important and it reinforces the conclusion that the zero fiscal multiplier effect of good spending cannot be explained by spillover effects on the neighboring locations. The two bottom panels show opposite stories. While Panel C shows that a spending shock in services has some positive and significant ``outflow'' effects on neighboring locations, Panel D highlights that R\&D spending has some small but negative spillover effects on the surrounded MSAs.
 
To sum up, these results point out that tradeability of the different type of products cannot explain the heterogeneous reactions of earnings to different types of government spending. Specifically, even accounting for these factors, we find that spending in goods does not generate any positive significant change in earnings and employment. These results imply that the heterogeneous reactions of outcome variables to different types of spending does not depend of geographic aggregation or misallocation of contracts.


\subsection{Crowding-out of Private Consumption}
\label{subsec:crowd}


Government spending might crowd out private consumption when the increase in government demand is followed by an increase in prices rather than an increase in total production.\footnote{There is no conclusive evidence whether government spending crowds-in or crowds-out private consumption. There is advocates for both sides, namely, a crowding-out effect \citep{Bailey1971, Barro1981} and a crowding-in effect \citep{Perotti2005, Canzoneri2002, Mountford2009}.} The increase in prices may be explained by the presence of adjustment cost. In the presence of adjustment cost, firms can not rapidly respond to positive demand shocks and as a consequence the demand shocks translate into price increases. Since capital face larger adjustment cost than labor,\footnote{Standard macroeconomic textbook assumes that adjustment cost are larger for capital than labor and therefore in the short run only capital can be adjusted.} one may expect that demand shocks to capital-intensive have lower impacts on total production than demand shocks to labor-intensive products.   

To test this channel, we investigate whether different types of government spending has different effects on private consumption expenditure.\footnote{\cite{Dupor2021} investigate the effect of government spending on different categories of consumer spending. Our paper studies the opposite hypothesis with respect to \cite{Dupor2021}.} Due to the lack of private consumption data, that is representative at the MSA level, we carry out this test at state-level.\footnote{State-level data are collected from the Bureau of Economic Analysis for the period $1998-2019$. The data include expenditure in durable and non-durable goods, and services. The figures are deflated by using the national CPI.} We implement the specification in equation \eqref{eq:fm_comp} by replacing $v_{l,t}$ with the private consumption expenditure by state $l$. Our estimates are plotted in Figure \ref{fig:crowd_privcons}. The point estimates are reported in Table \ref{tab:crowd_privcons} in \ref{sec:app_empres}. The results clearly highlight the absence of private consumption crowding out effects. All estimates, for any type of spending and at any horizon, are close to zero and non statistically significant at any standard level. Our results suggest that government spending has neither a crowding in nor a crowding out effect on private consumption expenditure.

Overall, our results point out that as the crowding out effects are small, they cannot explain the differences in multipliers across categories of spending. Specifically, we do not document any significant crowding out effect due to spending in goods that might be the offsetting force that leads to a zero effect in the good spending fiscal multiplier. 

\subsection{Firm Dynamics and Entry Costs}\label{subsec:bus_dyn}


Government spending may affect firm entry and investment decisions. Increases in government spending may reduce the uncertainty about future profits, and ease credit constraints. If that is the case, the increase in economic growth may be explained by either higher firm entry\footnote{\cite{Lewis2017} find that net firm entry rises after an expansion in the U.S. government spending.} or a stronger expansion of incumbent firms.\footnote{On one hand, some studies show that incumbent firms expand after winning a contract in US~\citep{Juarros2021} and Brazil~\citep{Ferraz2015, Lee2021}. On the other hand, \cite{Atanassov2018} show that government spending crowds out the R\&D effort of private firms.} Which are two canonical mechanisms to explain economic growth in economies with firm-level heterogeneity~\citep{Acemoglu2018}.

\subsubsection{Firm Entry}
The heterogeneity on fiscal multipliers between goods and services may be explained by difference in entry cost between firms who produced goods vs firms who produce services. If firms who produce goods (e.g. manufacturing industries) have a larger entry cost than firms who produce services, we should observe that the mass of firms who are at the margin of entering or not to the market is higher in the services sector. 

We test this conjecture with the following exercise. First, we study the impact of government spending by category on the net entry rate of firms. We collect data at MSA-level on establishments entry and exit rates from the Business Dynamics Statistics Datasets. We ran the specification in equation \eqref{eq:fm_comp} by replacing the outcome variable $v_{l,t}$ with the establishment net entry rate in MSA $l$ at time $t$. Results are reported in Table \ref{tab:entry_rate}. 

Panel A shows changes in establishment entry rate after a local spending shock. The estimates suggest local fiscal shocks have marginal effects on the creation of new establishments. We only document a positive and significant effect on the creation of new establishment one period after a one percent increase in good spending. A standard endogenous growth model would predict, as a consequence of the increase establishment entry due to the government spending in goods, a rise in output instead of a zero fiscal multiplier effect. Panel B show establishment exit rates after a local fiscal stimulus. We report a strong negative effect on establishment exit rates after a shock in service spending. An endogenous growth model would predict that spending in services to lower the economy-wide growth rate and decrease output. Estimates from Figure \ref{fig:fm_earnings} show the opposite pattern. 

\subsubsection{Firm Innovation}

Also we test whether government procurement contracts increase innovation of incumbent  firms. We measure innovation activities by the number of granted patents to inventors in a specific locality and time (See details on the data construction in Appendix \ref{sec:app_patentdata}). Our estimates on how public spending affect the number of granted patents are presented in Table \ref{tab:patent}.\footnote{The number of patents only includes patents for which the DoD has no economic interest. We restrict only to private patents to better proxy for private innovation efforts. Results remain unchanged if we include the patents for which the DoD has economic interests.} The reported results suggest a significant crowding out effect of private innovation due to an increase in government R\&D spending. Other types of spending do not cause any significant changes in the amount of private innovation. One possible explanation for the crowding-out effect of government spending in R\&D on private innovation is the scarcity of researchers. If the local labor supply of researchers is quite inelastic in the short-run, then a demand increase of R\&D activities generates an excess in demand for research activities leading to a switch from private innovation projects to government funded projects. According to an endogenous growth model, these estimates would suggest, differently from what we have showed in section \ref{sec:emp_fm}, a decline in economy-wide growth rate and a drop in output.

The previous tests explore the impact on business dynamism and innovation, two main mechanisms suggested by a standard endogenous growth models, explain the results we observe. The empirical estimates are not consistent with this mechanism explaining the effects we observe on total employment growth. If anything we observe a puzzling crowding out effect of government spending on R\&D on local innovation. This fact needs to be investigated further. 


%%%%%%%%%%%%%%%%%%%%%%%%%%%%%%%%%%%%%%%%%%%%%%%%%%%%%%%%%%%%%%%%%%%%%%%%%%%%%%%%%%%%%%%%%%%%%%%%%%%%%%%%%%
%%%%%%%%%%%%%%%%%%%%%%%%%%%%%%%%%%%%%%%%%%%%%% CONCLUSIONS  %%%%%%%%%%%%%%%%%%%%%%%%%%%%%%%%%%%%%%%%%%%%%%
%%%%%%%%%%%%%%%%%%%%%%%%%%%%%%%%%%%%%%%%%%%%%%%%%%%%%%%%%%%%%%%%%%%%%%%%%%%%%%%%%%%%%%%%%%%%%%%%%%%%%%%%%%

\section{Conclusions}
\label{sec:conclusion}



The Great Recession renewed the interest on the effectiveness of fiscal policy as a countercyclical policy tool. Most of the studies that estimate local fiscal multipliers conclude that fiscal spending crowds in local economic activity. However, there is substantial heterogeneity in this estimates. Theory suggest that local economic characteristics and the composition of government purchases matter for the size of the fiscal multiplier. While an nascent empirical literature shows the amplification role of local economic characteristics, much less is known about the role of the composition of the government stimulus. This paper aims to contribute in that direction.

In particular we show that purchases of services (R&D and non R&D) have an amplification effect compared to the purchase of goods. We find that the two-year employment and earnings multipliers are lower when government purchase goods compared to a scenario in which government hire services or fund R\&D activities.    

We find that the difference in the response of economic activity to the type of spending is associated with the intensity of labor used to produce the products demanded by the government. We rule out that the higher presence of geographical spillovers in the production of goods is the factor driving our results. Also we rule out that the results are explained by government demand inducing higher firm entry, which may be heterogeneous across sectors because of the difference in the sunk cost. Still other important channels like input-output linkages, investment and consumption elasticities remain to be tested. 

Overall, the results suggest that there is room for governments to redesign their fiscal stimulus packages in order to obtain more bang for the buck. Reallocating dollars from goods towards services or R\&D could be optimal if policymakers' metric is anchored to aggregate employment and earnings. 


\end{document}