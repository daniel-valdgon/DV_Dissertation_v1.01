\documentclass[dv_diss_main.tex]{subfiles}

\begin{document}
\section{Institutional Context}\label{InstCont}

This section describes the functioning of public finances in Mexico, focusing on the creation of new intergovernmental transfers in 1998. This expansion of transfers, known as Ramo-33, led to an unexpected and sizable increase in local public spending. Also, I explain how the institutional design of the newly enacted transfers gave state governors disproportional power and, therefore, left room for political favoritism.  

\subsection{The Expansion of Earmarked Transfers}

Mexico has a revenue-sharing system in which the federal government collects most of the taxes and later redistributes them across states and municipalities.
Thus, the fiscal capacity of sub-national governments is limited severely by the number of intergovernmental transfers they receive. Therefore, any policy that affects intergovernmental transfers to a great extent can affect the total of local public spending substantially. 

This paper takes advantage of the expansion of intergovernmental transfer that followed the creation of Ramo-33 in 1998. It focuses on the two subcomponents of Ramo-33 that allocated funds to municipalities: FORTAMUN-DF and FISM. I do not include the rest of the subcomponents of Ramo-33 because those transfers are managed by state governments; therefore, I cannot observe the municipalities where they are allocated.  From now on, I refer to these two subcomponents as Ramo-33.\footnote{Both sub-components use explicit allocation formulas: FORTAMUN-DF is allocated based on population, and FISM is distributed according to a formula that uses a multidimensional deprivation index that considers access to electricity, poverty, education, and sewerage among others. The formulas are updated every time new information from the most recent population census is released.  See guidelines of these sub-components here  \url{https://www.coneval.org.mx/Informes/Evaluacion/Estrategicas/Ramo_33_PDF_02032011}}
These two funds are desirable for identification purposes for two reasons:
First, they provide unexpected increases in local public spending. As Figure~\ref{fig:unexpected}  shows, there is a weak correlation between the growth rate of Ramo-33 municipal transfers and the growth rate of public spending before 1998. Second, these transfers are economically significant. Figure~\ref{fig:iceconsizer33}  shows that, between 1998 and 2006, local spending as a share of municipal GDP increased by 20 percentage points. 
These relatively large magnitudes resulted from the redistributive nature of Ramo-33. One of the core objectives of the transfers was to allocate disproportional public resources to less developed municipalities, which explains the significant increases for the average municipality. 

Additionally, it is crucial to highlight that a substantial fraction of Ramo-33 is earmarked to infrastructure projects. As a result, it can be observed in Figure~\ref{fig:icbudgsize33} that the expansion of funds also led to an increase in public investment. Specifically, FISM is earmarked for a broad set of infrastructure projects: from social infrastructure (e.g., health and school facilities) to core economic infrastructure (e.g., electrification, construction of dams, sewerage, and municipal roads). In comparison, FORTAMUN-DF is earmarked for either infrastructure (e.g., maintenance of urban infrastructure) or non-infrastructure projects (public security, debt payments, and acquisition of goods to strengthen the productivity of public workers). Overall, more than two-thirds of the transfers are earmarked to infrastructure projects.  

To summarize, Ramo-33 implied an unexpected and economically substantial increase in local public spending. This spending shock is explained by a sizable growth of intergovernmental transfers to local economies and by a shift in the allocation of resources towards less developed areas, where the economic size of this spending increase is expected to be larger.



\subsection{The Political Economy of Earmarked Transfers}

One of the main motivations behind the creation of Ramo-33 was to protect the intergovernmental transfers from the political discretion of the Federal government. 
For that purpose, these intergovernmental transfers were designed as entitlement programs, meaning: the transfers were to be received by municipalities every year according to a fixed formula define by predetermined municipal characteristics. 
Also, the law that enacted these transfers assigned state governments the role of allocating these transfers at municipal level.  
This restriction aimed to prevent municipalities, with low bargaining power regarding the federal's government, from receiving systematically fewer transfers.

However, these safeguards created numerous related distortions by transferring disproportional power from the federal to the state government. 
As a result, anecdotal evidence suggests that state governments did not strictly follow the guidelines defined by the allocation formula when distributing the resources of Ramo-33. 
Some governors publicly stated that they should be authorized to skip the allocation defined by law and replace it with theirs.\footnote{See \cite{diaz2004descentralizacion}, \cite{trillo2007transferences}, \citep{langston2010governors}.}

Figure~\ref{fig:pol} shows two pieces of evidence to substantiate the claim of manipulation of intergovernmental transfers. First, it shows that a large proportion of municipalities reported not receiving intergovernmental transfers between 1998 and 2002. This result is odd since the statutory allocation mandated a positive amount of transfers for all municipalities. Second, it shows the dispersion in the growth rate of intergovernmental transfers. Given that these transfers worked as entitlement programs that allocated resources across municipalities based on a fixed formula, one should expect a coefficient of variation close to zero.\footnote{
Several reasons can explain a positive dispersion on yearly growth rates, namely:  measurement error, political business cycle, or a change in formula's inputs, which takes place every time a new population census is released (1992, 2002, 2012).}
However, Figure~\ref{fig:pol} shows not only a positive coefficient of variation but that it increases after the enactment of Ramo-33 until 2005 as well. I find the substantial decline after 2005 is correlated with a relatively large increase in fines imposed by the ASF (Autonomous watchdog agency) to municipal and state governments for malfeasance and waste of public resources. 

To summarize, the evidence points towards significant discretion in the allocation of intergovernmental transfers across municipalities. Since state governors were in charge of distributing these funds, distinct political factors they care about, such as political competition, turnout, and partisan favoritism, could explain their allocation. 

%This paper focuses on the role of partisan favoritism in the allocation of transfer at local level. Where partisan favoritism is defined as the tendency of central governments to favor local constituencies ruled by their same political party. This political factor is convenient for two reasons: first, it connects our research question with one of the most prevalent political distortion found in democracies; second, it provides a transparent identification strategy based on close elections which will be explained in detail in the following section. 



\end{document}

%\subsection{Political alignment and infrastructure spending shocks}
%Since the paper takes advantage of variation in political alignment, we require enough variation of alignment across other party characteristics that the political economy literature finds relevant to explain the quality and quantity of public goods. Notably, we want a mayor aligned with a state governor's party not necessary to align with the President's party, represents only a particular political ideology, nor is always an incumbent party.

%During our period of study (1998-2005), Mexico abandoned the famous one-party rule system to become a democracy with solid parties and high political competition.\footnote{Mexico is a federal republic with a stable multiparty system with three strong parties: PRI, PAN, and PRD. Currently, Mexico has about ten registered parties. The political parties with the highest vote share during my period of study are PAN (Partido Acción Nacional) and PRI (Partido Revolucionario Institutcional). PRD (Partido de la Revolucion Democratica) also has an important vote share in the states among southern states. After 2014, an additional party, MORENA (Movimiento Regeneración Nacional), became an important political actor.}

%This political competition is favorable for our research design in the sense it provides a large set of close elections, which is the sub-sample of the data that I exploit for a causal identification of the parameters of interest. 

%Regarding the electoral cycle, we will have changes in alignment that can happen every three years; therefore, our main estimates will be able to speak about the short-term effects of partisan alignment. 
%President, governors, and mayors are elected through a plurality rule for a six-year term with no re-election possibilities (re-election is allowed after 2014). 
%Still, we consider that parties, instead of politicians, care about remaining in office.
%As figure ~ref{} shows, Presidential and governors' (state) elections are held at different calendar years, while mayoral (municipality) elections are held every three years, with some marked exceptions; since state governors are elected every six years, the municipal-mayors are elected either held separately or concurrently with state elections.
%\footnote{Only the mayoral elections of Mexico State-1994 and Veracruz-2000 and Coahuila-2005 experienced for particular reasons a 4-year term}

%The period of study provides enough variation to separately control by party ideology, presidential and incumbency status. 
%As Figure~\ref{fig:pol} shows, the period of study provides rich variation in the partisan ideology of the governors. About 60 percent of aligned municipalities are from the left-wing (PRI and PRD), while the rest belong to the right-wing party (PAN). This variation also guarantees we can easily separate the effect of being aligned with the President's party. Finally, the high political competition that characterizes the studied will provide us with additional variation in incumbency status within the variable of political alignment.

%Overall, the creation of the two sub-programs of Ramo 33 that this paper focuses on, implied an unexpected increase in infrastructure spending at the local level, which was not correlated with the previous allocation of public investment. Although the law contemplated allocating resources based on an objective formula, politics play a role in defining the intensity of transfers that each municipality received. Particularly, anecdotal and academic literature has identified state-governors as critical players in the allocation of resources \citep{diaz2004descentralizacion,langston2010governors,trillo2007transferences}. This paper exploits the fact that political alignment was an important determinant in allocating the municipal transfers from Ramo-33. Particularly, it implements a research design that exploits quasi-experimental variation of political alignment to estimate the evolution of local employment in municipalities that received a disproportional amount of transfers due to being politically aligned with state governors. 

%National public spending also increase during this period by XX. 

%\textcolor{blue}{fact here about hte size of FISM and PROGRESA in terms of BUDGET for the Periods Studied and its GROWTH. Figure of total ammount of each program at national level for the period studied ideally with the line of PRONASOL}

%\footnote{The infrastructure projects comteis infrastructure projects  Despite being earmarked-transfers aimed to fund a wide range of pubic goods from health and education infrastructure to roads and utilities.} 

%This paper focuses on these funds (FISM and FORTAMUN-DF) because they were easier to divert by state politicians since they aimed to fund non-programmatic spending, contrary to the health and education components of Ramo 33, which were restricted by the pre-existent distribution of public hospitals and schools.

%Local revenues multiplied as a consequence of the creation of FORTAMUN-DF and FISM. In less than three years, between 1998 and 2000, this new revenue source went from accounting to nothing to making up 40\% of local spending. The average local spending grew six-fold across municipalities with a wide variation. This considerable variation could be due to differences in the needs of public infrastructure across municipalities or to political factors. This paper aims to measure the role of a particular political factor, partisan alignment, in explaining that variance.

%There are three reasons why despite the existence of the formula, there may be some discretion in the assignment of the transfers: First, the fiscal watchdog agency that monitors these resources only started to implement audits in 2000 and to impose severe fines in 2004. Two, the mayors and voters were not fully aware of how the formula worked to make state governors accountable. Three, there was a disagreement about how the transfer should be distributed and some delay in implementing the census information from the side of the governors. These three factors lead to a significatively considerable variation in the allocation of FORTAMUN and FISM over time within a state. As Table 1 and 2 show, the effect of alignment on intergovermental transfer from Ramo 33 is positive before 2004 and almost zero afterwards. 



% studied by \cite{garin2019putting}
% \cite{cappelen2003impact}. 

%Figure~\ref{fig:iceconsizer33} and ~\ref{fig:icbudgsize33} shows that Ramo-33 meant a sizable increase in local infrastructure spending. Panel A of Figure 1 shows the average economic size of Ramo-33 and public investment as a percentage of the local GDP in 1998. In 1998, the first year of the program, the transfers from Ramo-33 amount to 10 percent of local GDP, while the public investment accounted for 17 percent. By 2004, only six years after the policy took place, the economic size of Ramo 33 tripled, from 10 to 30 percent, while investment almost doubled from 17 to 27 percent of 1998's local economy size. 
%The fact that there is not a one-to-one change between the size of Ramo-33 and infrastructure spending could be explained by several factors: the cost of allocating and supervising the infrastructure projects, the lack of a counterfactual of what would have happened to infrastructure in the absence of Ramo-33 (it shows a declining trend, so the effect of Ramo-33 could be higher than the difference of any year to 1998), and third that may be there is not after all a flypaper effect (money, i.e., infrastructure spending, do not sticks where it hits), which is consistent with recent literature challenging that old paradox \citep{suarez2016estimating,helm2021dynamic}.
