\documentclass[dv_diss_main.tex]{subfiles}


\begin{document}
\section{Main Results}

This section presents our main estimates. First, it explores the extent to which political alignment affects the allocation of intergovernmental transfers and total public spending. Second, it shows the effects of political alignment on employment and wages for the universe of formal private-sector workers. Third, it uses coarser measures of total employment (both formal and informal) to evaluate whether the results on formal employment affect total employment or are explained by shifts between the formal and the informal sector. Unless otherwise indicated, I use equation (2) and equation (3) to obtain all results presented in this section.

\subsection{Public Revenues and Spending}

\subsubsection{Transfers} 
The first-order question is whether being politically aligned during the period of expansion of earmarked transfers benefits municipalities. I use two measures to answer this question. The first measure is the probability of receiving transfers (extensive margin), and the second is the three-year growth rate of transfers per capita (intensive margin). The first measure aims to identify if state governments punish non-aligned municipalities by holding up intergovernmental transfers. The second measure tests whether there is a difference in the total amount of transfers received.

Table~\ref{tab:1trans} shows that politically aligned municipalities receive a higher amount of transfer, but alignment does not significantly increase the probability of receiving transfers. In particular, columns 1 through 4 show that political alignment increased the growth rate of earmarked transfers between 29 and 65 percentage points depending on the specification. Column 2 is my preferred specification since it uses the narrower bandwidth and includes the set of controls. This column indicates that the growth rate of earmarked transfers was 42 percentage points higher in aligned municipalities compared to their non-aligned counterparts. To put this into context, the average growth rate of earmarked transfers, for the non-aligned municipalities is 138 percent, which implies that political alignment increased the growth rate of intergovernmental transfers by one-third (= 42/138). These relatively large growth rates in the control groups are explained by the context of the study period, during which earmarked transfers expanded from near zero (as a fraction of local revenue) to accounting for almost 30\% of total revenues. 
Columns 5 to 8 of Table~\ref{tab:1trans} show that political alignment does not consistently increase the likelihood of receiving earmarked transfers during the years of the mayor's term. 
This null result could be explained by the fact that 96 percent of municipalities in the control group (non-aligned municipalities) report receiving transfers leaving little room for discretion between aligned and misaligned municipalities. 

Figure~\ref{fig:rddplotearmarks}  plots the growth rate of earmarked transfers around the alignment threshold for the post-election and pre-election periods. The discontinuity for the post-election period is evident and implies that aligned municipalities have higher growth rates than their non-aligned counterparts (Panel A). Moreover, the fact that there are no discontinuities in the observed pre-election growth rates at the alignment threshold validates the identification assumption. This test is analogous to what is referred to in the literature of difference and difference as the parallel trends test. 


%\subsubsection{Alternative State and Federal Policies}
\subsubsection{Other Transfers, Taxes, and Debt} Other revenue sources may offset the effect of earmarked transfers on total public revenues. Governors may compensate non-aligned municipalities with another type of public resources, leading to a misleading conclusion when studying the net effect of political alignment   \cite{kramon2013benefits}. 
Another mechanism that could offset the effect of earmarked transfer on total spending is the response of taxes and debt.
Local governments may change their optimal decisions regarding taxes and debt as a result of higher intergovernmental transfers.  
For example, a well-established theoretical result suggests that governments should reduce their taxes after a fiscal windfall.\footnote{
This prediction has motivated extensive empirical literature around the flypaper effect, which has found mixed results on this prediction. See \cite{inman2008flypaper} for a review} The impact of fiscal windfalls on debt is less clear. On the one hand, higher transfers imply higher collateral for local governments; on the other hand, the fiscal windfalls imply lower borrowing needs.\footnote{Also, political alignment could directly affect debt if central politicians can influence the access to credit. \cite{de2020political} shows that in Mexico, political alignment with the president between 2009 and 2013 explains higher access to debt. It is important to validate is his findings also apply to our context, which not only focuses on a different period but a different measure of political alignment.}  

Table~\ref{tab:2revenues} uses the detailed categories of revenues collected in the public finance data to test if any of the mechanisms mentioned above amplify or offset the effects of political alignment on local public resources. 

For ease of comparison, column 1 again presents the effect of political alignment on the growth rate of total earmarked transfers. Column 2 shows the impact on revenue sharing transfers, which are the main revenue source (representing 53\% of total revenues) and are also administrated by state governors. I do not find robust evidence of revenue transfers changing at the alignment threshold; results are statistically significant with an 11 percentage point bandwidth but disappear when using a 5 percentage point bandwidth. Although this coefficient estimate is non-statistically significant, it is economically significant, and I cannot rule out that it plays a role in increasing overall spending. 

Column 3 of Table~\ref{tab:2revenues} reports the effect of alignment on taxes, which represent about 5.9 to 6.2 percent of total revenues. I find that the growth rate of taxes for aligned municipalities is higher but not in a statistically significant manner. Moreover, the point estimates in both Panel A and B suggest that the effect of alignment on taxes would be positive.  Column 4 of Table~\ref{tab:2revenues} shows the result for debt. During the study period, debt constitutes a relatively small fraction of local revenues; in our sample, it represents at most 3.3 percent of total revenues. The point estimates for the effect of political alignment on debts are positive but neither statistically significant nor consistent across different bandwidths.
To summarize, alignment-induced transfers do not seem to significantly affect local governments' decisions regarding taxes and debt. This evidence is in line with the idea of a flypaper effect; grants increase total spending, which has been validated by \cite{bracco2015intergovernmental} using data from Italy and a similar research design as this paper. 

This implies that, if anything, political alignment brings more rather than fewer resources to local economies. Yet, I have not entirely ruled out that other public resources that I can not observe with the public finance data are not responding to political alignment. I will go back to this in the robustness section, where I explore the effect of political alignment on the allocation of other nationwide programs unrelated to intergovernmental transfers. 
%Progresa and Seguro Popular. 

\subsubsection{Public Spending}

The second main result is presented in the last column of Table~\ref{tab:2revenues}. Total public spending increase in politically aligned municipalities. This is consistent with the fact that I did not find crowding-out effects from other sources of revenue  (transfers, taxes, or debt). Both Panel A and B illustrate a consistent story; political alignment increases the total public spending growth rate by 10 to 12 percentage points. Since the growth rate of spending for the control was about 56 percent, this suggests that the effect of alignment on total resources is about 7 percent (1.07=164/156). This estimate can be interpreted as a net effect of partisan alignment after any compensation and behavioral effects induced by the increase in intergovernmental transfers have been netted out. Panel A of Figure~\ref{fig:rddplottotspending} confirms the discontinuity, while Panel B reassures our identification assumption. 

\subsection{Employment and Wages}

This subsection explores whether employment and wages evolve differently in politically aligned municipalities, which, as it was explained above, receive a disproportional amount of governmental resources.  To do so, I use data on total jobs and the aggregate wage bill for the universe of formal sector jobs recorded by the Mexican Institute of Social Security.
\footnote{ Formal workers represent 40\% of the jobs and account for about 70\% of the output. I interpret the result here as relevant for the formal sector and do not extrapolate its conclusion to the informal sector. In the next subsection, I study the effects on aggregate employment using coarser sources of information like household surveys or economic censuses} 
I compute two outcome measures with this data that can be observed at the municipal-year level and disaggregated by sector and firm size. The first is the absolute number of formal jobs, and the second is a measure of wages that I compute as a total wage bill divided by the number of jobs.

Columns 1 to 4 of Table~\ref{tab:3femp} show the impact of political alignment on the growth rate of private employment. Overall, the results show that aligned municipalities have a slower growth rate than their non-aligned counterparts.  The growth rate of formal employment is between 9.5 to 12.1 percentage points lower in politically aligned municipalities. This effect is robust to the choice of bandwidth and is not sensitive to the inclusion of different controls. 

To provide an interpretation of the coefficient, I look at the sample mean for non-aligned municipalities presented in the table and the plots of the outcome variation against the running variable. Since the mean growth rate of private formal employment in non-aligned municipalities is between 7 and 9.1 percent, the coefficient estimate suggests that the negative coefficient should be interpreted more as a slowdown in job creation in aligned places compared to non-aligned municipalities. Both Panel A and B of Figure~\ref{fig:rddplotformalemp} provide the same interpretation. The intersection of each slope with the alignment threshold, when the vote margin is equal to zero, could be interpreted as the conditional growth rates at the threshold. Panel A indicates that employment grew by more than 10 percent in non-aligned municipalities while slightly above zero for aligned municipalities. Panel B shows that before the election, both aligned and non-aligned municipalities were experiencing private employment growth at the threshold.

Columns 5 to 8 of Table~\ref{tab:3femp} show that political alignment did not affect average wages. The point estimates are not statistically significant and relatively small (less than 0.1 percent) compared with the average wage growth of non-aligned municipalities (between 7 to 8 percent). For the sake of completeness, Figure~\ref{fig:rddplotwages} plots the growth rate of wages for both the post and pre-election periods, and as expected, I do not observe discontinuities in wage growth either before or after alignment takes place. 

\end{document}


%%%%%%%%%%%%% Estimates 
%Aggregate employment 


%s respond into political alignment during the particular times when Mexico is well known for its large informality rates, which have been blamed as one of the main reasons for the reluctant economic growth (Hanson 2010). The formal sector employs about 40\% of the workforce and produces about 70\% of the total output. Formal employment is not sparse across all municipalities; only about 1300 out of 2496 municipalities register at least one private-formal employee during our period of study (1997-2006). Our estimates consider only municipalities where I see at least 100 private-formal employees on average for the period for which information is available. This allows having less noisy estimates of the effect of alignment on formal employment growth. 




%It is interesting to explore if the coefficients I observe on transfers follows a logic outlined by the literature on political economy. Particularly, I estimate if the effect of political-induced transfers is more prevalent in state governments who face high political competition\cite{Curto } and whether the effect of alignment on transfers follows the electoral cycle\cite{Brollo, Chile, CC}, i.e. whether the effects of alignment on transfers are higher the year before the election takes place. To answer this questions, Table~\ref{} shows the heterogeneity of the effect of political alignment along these two dimensions. Column (1) and (2) present the benchmark results with no heterogeneity. Column (3) and (4) show the heterogeneity of the effects of alignment along the political competition dimension, \footnote{To obtain municipal level measures of political competition I measure the historical average vote margin between the winner and the runner up before the close election of reference take place, those municipalities with a historical average above the median of our subset of close elections are labelled as places with relatively high political competition.} while column (5) to (8) presents the effects of alignment on the electoral cycle of mayors and governors.\footnote{To measure the local political business cycle I use an indicator variable for the last year of the mayors term (three year term limit) while the state business cycle is measure by an indicator variable for the last two years of the governor’s term (six year term limit). Notice that since I am asking heterogeneity along this dimension, I estimate equation~\ref{eq:dd} without the cohort-time dummy, only cohort dummy is used in the specification.} According to column (3) and (4) of Table~\ref{} the effect of political alignment is XX times stronger in states that have lower political competition. This suggest that political accountability can limit the room the state-governors have to favor their co-partisans at local level with transfers. This result goes in line with the findings by \cite{curto, brollo} who find that partisan favoritism in non-contested state-governors is XX times higher than in contested governments. 
%When I focus on the relation between the political alignment effect and the electoral cycle, I find that alignment is more sensitive to the electoral cycle of state-governors rather than the electoral cycle of municipal mayors. The effect of mayor’s electoral cycle (column 6) is XX higher than the effect of alignment in any other year, while in the case of the governor’s electoral cycle (column 8) I find that the effect of alignment is XX times higher. Overall, this suggest that favoring their co-partisans is a decision that respond to a vote-seeking behavior from the side of state governors. This matches the results found in \cite{brollo} and \cite{chile}.

