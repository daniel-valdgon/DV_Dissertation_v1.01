\documentclass[dv_diss_main.tex]{subfiles}

\begin{document}

\subsection{Ruling Out Violence and Corruption}

\subsubsection{Violence}   Resource abundance is associated with increased conflict between population subgroups trying to capture rents from the fiscal windfalls. An increase in violence could undermine the positive effects of higher spending and explain why I observe a slowdown in total employment. Table~\ref{tab:violence} estimate the effect of political alignment on homicides. Columns 1 and 2 measures the effect of alignment on the probability of a homicide during each year of the mayor's term; while columns 3 and 4 study the effects on the growth of the homicide rate. In summary, I do not find consistent and statistically significant evidence that alignment increases homicides. However, the standard errors are relatively large, so I cannot reject that political alignment increases the homicide rate.

To conclude about the effect of alignment on violence, I rely on the institutional context of Mexico. During my study period, 1998-2006, homicides were on a declining trend. Therefore, the abrupt changes that I observe in economic activity from increases in political alignment are not likely to be explained by historically low levels of homicides. This test also rules out that political alignment increases violence because it facilitates the implementation of anti-drug  crackdowns.\footnote{\citep{dell2015trafficking} finds that political alignment facilitated the implementation of anti-drug crackdown policies, which ended up increasing violence. I do not find this effect because I focus on a different period (2007-2009) and under a different definition of political alignment, i.e. she focused on the alignment between mayors and the president's party.} 

%Drug cartels %%%%%%que digo aca%%The second test evaluates if the presence of drug cartel can explain the is a sufficient threat to aligned government to appropiate frs is a sufficient threat  are known as the main actors behind the increase in violence that Mexico experienced after 2006. It could be that homicides did not increase but still, resources are appropriated by those illegal groups who have the monopoly of violence. I separate the sample between states with and without the presence of a drug cartel. Columns 3 and 4 show the effect of political alignment on states with a drug cartel presence, while columns 5 and 6 show the results for states without the presence of drug cartels. I find that XXXX


\subsubsection{Corruption} 
Another channel through which higher spending could create lower economic growth is corruption. \cite{brollo2013political} shows that fiscal windfalls spur corruption, which can reduce local economic growth \cite{colonnelli2020corruption}. The basic argument is that fiscal windfalls increase corruption either because the incumbent politicians tend to be more corrupt (moral hazard effect), or because past fiscal windfalls attract low-quality politicians in the subsequent elections (selection effect). 

The effect of alignment on employment takes place immediately after the increase in transfers. Therefore, changes in the quality of candidates of subsequent elections cannot explain my results.\footnote{ 
The research design assumes no systematic difference between aligned and misaligned municipalities in baseline characteristics, among those the pool of candidates before alignment is decided. Although I do not have information on the valence of candidates to test this assumption, I use a wide set of baseline characteristics in Figure~\ref{fig:baselinebalance}.} 
Still, the results could be explained by a moral hazard effect; that is, it could be that the elected mayor changes his or her behavior in response to the excess of resources and becomes more corrupt. This increase in corruption may be sufficiently large to undermine the potentially positive effect obtained through additional resources.   

I use data on audits to local governments to test if corruption is a mediation mechanism that explains the negative impact of political alignment on total employment. The audits are performed by an autonomous watchdog agency (Auditoria Superior de la Federación) that is part of the federal government and out of reach of the state governors. 

The audits are conducted on a subsample of municipalities and report continuous measures of corruption and malfeasance for each municipality audited.\footnote{The selection of municipalities to be audited is not random, it obeys to population and total municipal budget.\footnote{ See \cite{arias2018priors} and \cite{chong2015does} for more details} They report the percentage of audited spending that is not supported by receipts,  flagged as corruption. Also, the municipal audits focus on evaluating the malfeasance of Ramo-33 transfers; in particular, they focus on the FISM transfers. Their measure of malfeasance is defined as the percentage of spending that is not spent on the goods or services that are in line with the purpose of the earmarked transfers.}

Table~\ref{tab:corruption} shows the results of political alignment on the probability of being audited (columns 1 and 2); the likelihood that more than 10\% of the audited spending is not documented, i.e., being found guilty of corruption (columns 3 and 4); and the probability that more than 10\% of audited spending does not correspond with the purpose of the earmarks, i.e., malfeasance (columns 5 and 6). Three conclusions emerge from this table: First, it is reassuring that political alignment does not affect the probability of being audited (columns 1 and 2). The point estimates are positive (1.9 percentage points in Panel A column 2) but not statistically significant.  This is in line with the fact that the watchdog agency that performs the audits is autonomous and not influenced by state governors. Second, I find that alignment does not increase the probability of being accused of corruption (columns 3 and 4), but it does reduce the likelihood of being accused of malfeasance (columns 5 and 6). The latter effect suggests that the likelihood of being accused of malfeasance falls by 42 percentage points at the cut-off, which is particularly strong especially considering that the control's mean is 22 percentage points.

The surprising result that political alignment reduces malfeasance could be interpreted in two opposing ways. One interpretation is that governors exert more control and provide more guidelines to aligned mayors, which improves their management practices and the results of the audits. A second interpretation is that state governors influence the audits' results, asking the watchdog agency to be more lenient with politically aligned mayors. 

Since distinguishing between these two hypotheses is not possible unless other data is available,\footnote{
For example, \cite{chu2020hometown} obtains a similar result. They find that auditors reduce the proportion of questionable spending reports when evaluating their hometowns. The authors collect firm-level data of state-owned enterprises and compute real activity manipulation measures to disentangle between a discipline or a manipulation effect. Their evidence supports a manipulation effect. This data is not available for the case of Mexico. Still, it is definitively an area of future research.} 
I argue that audit reports are not politically manipulated for three reasons. First, the constitution gives economic and political independence to the watchdog agency. By being a centralized institution, it is also more difficult for governors to exert control over the audit's results. Second, several studies consider this data a valid corruption measure \citep{chong2015does,arias2018priors,ajzenman2021power}. Third, suppose aligned municipalities are actually less likely to be accused of malfeasance. In that case, I should observe more spending on the infrastructure projects (the primary purpose of the earmarks), which is confirmed by the data.

\end{document}