% File 'template.tex'

\documentclass[12pt]{report}

\usepackage{url}
\usepackage[round, sort]{natbib}
\usepackage[dvips]{graphicx}
\usepackage{amsmath}
\usepackage{amssymb}
\usepackage{algorithm}
\usepackage{algorithmic}
\usepackage{csquotes}



%%%%%%%%%%%%%%%%%%%%%%%%%%
% ARR's Added            %
%%%%%%%%%%%%%%%%%%%%%%%%%%
\usepackage[hyperfootnotes=false]{hyperref}
        % See that the hyperfootnotes=false turns on and off the hyper-noted-footnotes. In this case, cause the hyperref package has issues with the hyper-noting of footnotes, i'll just turn it off. 
\usepackage{xcolor}
	\definecolor{gublue}{RGB}{4, 30, 66} % Georgetown's university blue
\hypersetup{
    colorlinks=true,
    linkcolor=black,
    filecolor=gublue,      
    urlcolor=gublue,
    pdftitle={Thesis_Daniel_Valderrama_PhD_Economics_2022}, 
        %title of the PDF output file, to be displayed in the title bar of the window. 
    pdfpagemode=FullScreen,
    citecolor=gublue,
            % This changes the color of the citations.
    }

\usepackage{booktabs} 
        % Enhanced tables
\usepackage{subfiles}
        % To create standalone documents
\usepackage{ragged2e} 
        % To set ragged text and to allow hyphenation
\usepackage{tikz}
        %ffor figures       
\usepackage{subcaption}
        % Important so we could use the \begin{subfigures}<->\end{subfigures} command. 
% Expected value

%-------------------------
%--------- Expected value 
%-------------------------
%%%% for the expected value
\usepackage{amsthm, amsmath, amsfonts, mathtools, amssymb} %% Math packages 
%\usepackage{pifont} %% It provides commands for Pi fonts (dingmacro, symbols, etc.)
\newcommand{\expect}{\operatorname{E}\expectarg}
\DeclarePairedDelimiterX{\expectarg}[1]{[}{]}{%
  \ifnum\currentgrouptype=16 \else\begingroup\fi
  \activatebar#1
  \ifnum\currentgrouptype=16 \else\endgroup\fi
}
\newcommand{\innermid}{\nonscript\;\delimsize\vert\nonscript\;}
\newcommand{\activatebar}{%
  \begingroup\lccode`\~=`\|
  \lowercase{\endgroup\let~}\innermid 
  \mathcode`|=\string"8000
}

%%%
\usepackage[toc]{appendix}
\usepackage{etoolbox}
        %%% This line of code is used to fix the issue involving the hyperref package not working with the Appendix's TOC. 
\patchcmd{\footref}{\ref}{\ref*}{}{}
        % This line of code is used together with the [hyperfootnotes=false] command.
%\usepackage[compatibility=false,labelfont=it,textfont={bf,it}]{caption}
%\captionsetup{labelfont=bf,textfont=bf}
\usepackage{titlefoot}
        % To add the blind footnotes on chapter 2. Helps with giving credit to other authors. 

\usepackage{multirow}
        % Without this, the command \multirow won't work and the online appendix from chapter 3 would give us problems when compiling the document. 
\usepackage{makecell}
        % Without this, the command \makecell won't work and the online appendix from chapter 3 would give us problems in formatting. 
\usepackage{comment}
        % To comment a Full Stack.

% If there are any other \usepackage commands, put them here

\newtheorem{theorem}{Theorem}[section]
\newenvironment{proof}[0]{\textit{Proof.}}{}
\newcommand{\qed}{\hfill $\Box$}

% To comment out multiple lines of text.
\long\def\comment#1{}

\usepackage{guthesis}



\title{Essays on the Political Economy and Economic Impact of Fiscal Policies}

\author{Daniel Valderrama-Gonzalez}

\previousdegree{M.Sc.}

\thisdegree{Doctor of Philosophy}  % or Doctor of Philosophy, etc.

\thisdiscipline{Economics}

\thesistype{Dissertation}     % or Dissertation

% defense or approval date, not today's date...
\thesisday{19}
\thesismonth{Apr}
\thesisyear{2022}

\professor{Laurent Bouton, PhD}
\secondprofessor{Martin Ravallion, PhD}   % Only if you have 2 major professors!

\fulltitle{Full Title}

\indexwords{Political Favoritism, Place Based Policies, Infrastructure Earmarks, Defense Spending, Local Fiscal Multiplier, Rent Seeking, Mexico, United States}

\dean{Timothy A.\ Barbari}

\memberi{First I.\ Last}
\memberii{First I.\ Last}
% Use \memberiii, \memberiv, \memberv for up to 3 more members if needed.

\begin{document}

\pagenumbering{roman}

\maketitle   

\begin{abstract}
Economists get the opportunity and the responsibility of helping to design public policies. This implies that not only the big picture is necessary but also the details that will make the policy to actually work. This dissertation is all about the details! It shows how the economic efficiency of fiscal policy, measured by the size of the fiscal multiplier, is affected by three overlooked details, namely: political favoritism, elite capture and composition of government purchases. 

In the first chapter, \enquote{\textit{The Unintended Consequences of Politically Motivated Spending: Evidence from Mexico}}, I study the economic effects of politically motivated spending. I use a close-election research design that exploits variation in political alignment, between governors and mayors, during a period when political alignment implied a substantial increase in intergovernmental transfers. I find that political alignment increases intergovernmental transfers and public spending while slowing down private-sector employment. I find suggestive evidence that this slowdown is explained by the opportunity cost of reallocating economic activity toward rent-seeking activities. 

In the second chapter, \enquote{\textit{Distributional Effects of Intergovernmental Transfers in Mexico}}---co-authored with Carlos, Kiyomi, and Laura---,
we estimate the welfare effects of placed-based policies (PBP) that are targeted at lagging economic areas. We study the case of FAIS in Mexico, one of the largest PBP in the world that provides infrastructure-earmarks to lagging economic areas. We use a simulated instrument research design and find that an increase of FAIS translates into an increase in the coverage of social infrastructure (e.g. electricity, sewerage, piped water). However, the benefits in terms of infrastructure are followed by higher household income nor lower monetary poverty. The missing effect on welfare is explained by the bulk of the economic gains being captured by the non-poor residents of the initially targeted poor places.

In the third chapter, \enquote{\textit{Heterogeneous Spending, Heterogeneous Multipliers}}---co-authored with Pedro and Umberto---, we ask whether the size of the local employment and earnings multipliers depends on the composition of the government purchases. We answer this question by building a panel of military spending at the product-MSA-year level. We use this dataset in a shift-share research design to exploit the heterogeneous sensitivity of local military spending to national military buildups and drawdowns. We find that the goods and services that the government purchases determine the size of the local fiscal multiplier. Local fiscal multipliers are larger when public spending focuses its demand on labor-intensive industries.


\end{abstract}


\chapter*{Dedication}

Our goals are not our own, are completely bounded to others. I want to dedicate this work to four groups of people. \\
First, this work is dedicated to Diana also known as zelfi/sanchez/royce/posa. She did not need to read a single line of the thesis to give me the love, courage and spirit that I needed to continue pushing through it. 

Second, I can not imagine a dedication that does not include my family. So, Pa (Jaime), Ma (Estella), Pipe and Kta Gracias! Every single effort and mistake that you made when I was a child brought me here, so the degree is only yours. 

Third, I born in the late 80s, let me emphasize, the very late 80s. Among 3.35 million of people who born in Colombia during that period, only 0.006 has a PhD. Some people may said it was my effort, good luck, I think it was a good God. Thanks God

Finally to Marcelo and Danielle whom I truly expect to meet.

\chapter*{Acknowledgments}

%I am indebted to Laurent. A man of few words that received me under his wing saying "happy to help". He taught me about the humanity of being a researcher, how important is learning by doing, that there is not a bad idea if we put enough effort on it, and that a blind passion can deceive us but also keep us fighting. Also he fought on my side for this PhD as if it were his. He was patient with me asking for recommendation letters at the eleventh hour, and tried to eradicate the perfectionism from me at all cost. There is no perfect advisor, but he was close to be perfect.

%Thanks to Martin, before joining Georgetown I was seeing impossible to know him. I remember in the first year saying I am a fan while receiving a copy of his book (rally embarrassing). Now, after five years, I cherish the moments when he emailed me to talk about my papers or about his papers. I also will cherish the moments when he pushed me to go farther because of his confidence on me. Also, that in the eleventh our, in my dream job, he did the most he can to help me get it.

%Thanks to Toshihiko, he arrived to Georgetown during my third year. The first professor that create a completely open reading group. We ended up being more than 15 students under your wind, and simultaneously you continued publishing, teaching, mentoring and being DGS. I use to say in the corridor that you were a Samurai, the combination of strength, wisdom and duty. Every time I advance one epsilon for my thesis was thanks to your group. Also, every cell of macro economist that I have, which are few, is because of you. Thanks for that. 

%Garance, thanks for always being willing to talk and receiving me in your office with a smile. I wish i would have go more times to your office. In spite of joining the comitte in the last year, you made feel completely supported. You were worried for every aspect of my market, from the title of the paper, the punch line, to the economic content of it. Thank you for cheering me up during holidays after my mock interview, for telling me to be myself, and do not let that other's opinions discourage me. 

%Finally but not least, I want to thank Carlos. Thanks for believing in me, for encouraging me to start and finish this journey. Thanks for always caring for my career and well being. You define the course of me as a researcher, and believe in me at a level that only Diana does. I own you several of the job offers, articles and prizes I have won. I can not imagine how my research agenda would be if I would have not met you. Thanks you!

%Many classmates and faculty at Georgetown provided invaluable feedback and
%support. Thank you especially to Pedro, Umberto, Carlos,  all my co-authors, Pedro, Umberto, Pedro and UmbertoDario Sansone and Mike Packard. I would also like to acknowledge Nicola Persico, Philip Marx, Gaurav Bagwe, and Juan Margitic for useful feedback on previous drafts of these chapters



%\pseudochapter{Preface}

%A preface is not an introduction, and most theses do not need them.


\tableofcontents

\listoffigures  % Optional - Omit this line if you don't want a list of figures.
\listoftables   % Optional - Omit this line if you don't want a list of tables.

\newpage

\pagenumbering{arabic}  % Ordinary pages have Arabic numerals.

\chapter{Politics, Spending and Local Economics Growth:
Evidence From Mexico}\label{chap:c1}

\subfile{1_introduction.tex}
\subfile{2_Institutional_Context.tex}\label{sec:context}
\subfile{3a_Identification.tex}\label{sec:identification}
\subfile{4_Data.tex}\label{sec:data}
\subfile{3b_validity.tex}\label{sec:validity}
\subfile{5_Results.tex}\label{sec:results}
\subfile{5c_alt_emp.tex}
\subfile{6_Channels.tex}\label{sec:channels}
\subfile{6a_Economic_channels.tex}\label{sec:channels}
\subfile{6b_Channels.tex}\label{sec:channels}
\subfile{7_Conclusions.tex}


\chapter{Distributional Effects of Intergovernmental Transfers in Mexico}\unmarkedfntext{This chapter of my dissertation is part of a joint research project with Kiyomi Cadena, Laura Moreno-Herrera \& Carlos Rodríguez-Castelán.}\label{chap:c2}

\subfile{Chapter_2.tex}

\chapter{Heterogeneous Spending, Heterogeneous Multipliers}\unmarkedfntext{
This chapter of my dissertation is part of a joint research project with Pedro Juarros and Umberto Muratori.}\label{chap:c3}

\subfile{Chapter_3.tex}



\appendixtocoff
\appendices

\chapter{Appendix Chapter 1}
\section{Figures \& Tables}
\subfile{figures_c1.tex}\label{FirstAppendixC1}
\subfile{tables_c1.tex}\label{SecondAppendixC1}

\chapter{Appendix Chapter 2}
\section{Figures \& Tables}
\subfile{figures_c2.tex}\label{FirstAppendixC2}
\subfile{tables_c2.tex}\label{SecondAppendixC2}
\section{Online Appendix}
\subfile{online_appendix_c2.tex}\label{ThirdAppendixC2}

\chapter{Appendix Chapter 3}
\section{Figures \& Tables}
\subfile{figures_c3.tex}
\subfile{tables_c3.tex}
\section{Online Appendix}
\subfile{online_appendix_c3.tex}



\nocite{*} 
        % To show all references. (not cited on the document, ideally it's just to show them initially.)



    \bibliographystyle{plainnat}
    \bibliography{BibFile}  % thesis.bib



\typeout{***}
\typeout{*** Note!}
\typeout{*** Because this document has a table of contents,}
\typeout{*** you must run LaTeX TWICE to get it to print correctly.}
\typeout{***}

\end{document}

