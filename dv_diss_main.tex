% File 'template.tex'

\documentclass[12pt]{report}

\usepackage{url}
\usepackage[round, sort]{natbib}
\usepackage[dvips]{graphicx}
\usepackage{amsmath}
\usepackage{amssymb}
\usepackage{algorithm}
\usepackage{algorithmic}
\usepackage{csquotes}



%%%%%%%%%%%%%%%%%%%%%%%%%%
% ARR's Added            %
%%%%%%%%%%%%%%%%%%%%%%%%%%
\usepackage[hyperfootnotes=false]{hyperref}
        % See that the hyperfootnotes=false turns on and off the hyper-noted-footnotes. In this case, cause the hyperref package has issues with the hyper-noting of footnotes, i'll just turn it off. 
\usepackage{xcolor}
	\definecolor{gublue}{RGB}{4, 30, 66} % Georgetown's university blue
\hypersetup{
    colorlinks=true,
    linkcolor=black,
    filecolor=gublue,      
    urlcolor=gublue,
    pdftitle={Thesis_Daniel_Valderrama_PhD_Economics_2022}, 
        %title of the PDF output file, to be displayed in the title bar of the window. 
    pdfpagemode=FullScreen,
    citecolor=gublue,
            % This changes the color of the citations.
    }

\usepackage{booktabs} 
        % Enhanced tables
\usepackage{subfiles}
        % To create standalone documents
\usepackage{ragged2e} 
        % To set ragged text and to allow hyphenation
\usepackage{tikz}
        %ffor figures       
\usepackage{subcaption}
        % Important so we could use the \begin{subfigures}<->\end{subfigures} command. 
% Expected value

%-------------------------
%--------- Expected value 
%-------------------------
%%%% for the expected value
\usepackage{amsthm, amsmath, amsfonts, mathtools, amssymb} %% Math packages 
%\usepackage{pifont} %% It provides commands for Pi fonts (dingmacro, symbols, etc.)
\newcommand{\expect}{\operatorname{E}\expectarg}
\DeclarePairedDelimiterX{\expectarg}[1]{[}{]}{%
  \ifnum\currentgrouptype=16 \else\begingroup\fi
  \activatebar#1
  \ifnum\currentgrouptype=16 \else\endgroup\fi
}
\newcommand{\innermid}{\nonscript\;\delimsize\vert\nonscript\;}
\newcommand{\activatebar}{%
  \begingroup\lccode`\~=`\|
  \lowercase{\endgroup\let~}\innermid 
  \mathcode`|=\string"8000
}

%%%
\usepackage[toc]{appendix}
\usepackage{etoolbox}
        %%% This line of code is used to fix the issue involving the hyperref package not working with the Appendix's TOC. 
\patchcmd{\footref}{\ref}{\ref*}{}{}
        % This line of code is used together with the [hyperfootnotes=false] command.
%\usepackage[compatibility=false,labelfont=it,textfont={bf,it}]{caption}
%\captionsetup{labelfont=bf,textfont=bf}
\usepackage{titlefoot}
        % To add the blind footnotes on chapter 2. Helps with giving credit to other authors. 

\usepackage{multirow}
        % Without this, the command \multirow won't work and the online appendix from chapter 3 would give us problems when compiling the document. 
\usepackage{makecell}
        % Without this, the command \makecell won't work and the online appendix from chapter 3 would give us problems in formatting. 
\usepackage{comment}
        % To comment a Full Stack.

% If there are any other \usepackage commands, put them here

\newtheorem{theorem}{Theorem}[section]
\newenvironment{proof}[0]{\textit{Proof.}}{}
\newcommand{\qed}{\hfill $\Box$}





%---------------------------------------------%
%----------------Figures----------------------
%---------------------------------------------%



\newcommand{\rddplot}{
NOTE--This plot aggregate data into bins of half percentage points and estimate a third order polynomial regression between the running variable and the bins on each side of the cut-off. 
}


\newcommand{\rdd}{
NOTE--This table reports the estimates of political alignment from equation (2). The sample includes post electoral years of all municipalities with close elections during the period 1998-2003. The outcome variables are measure as a three year changes. Controls refers to state fixed effects, election-year fixed effects, and baseline political characteristics (incumbency status, previous political alignment, previous political party). Mean dep var refers to the sample average of the outcome variable for the non-aligned municipalities. 
}



\newcommand{\sharerev}{
Percentage of revenues is the sample average share of each source of revenue on total revenues for the non-aligned counterparts
}

\newcommand{\shareemp}{
Percentage of employment is the sample average share of each sector on total employment for the non-aligned counterparts
}

\newcommand{\event}{
NOTE--The figure plots the coefficients obtained from the estimation of equation (3) discussed in Section 4. The sample includes all municipalities with close elections during the period 1998-2003. The unit of observation is  the municipal-election pair, for each pair I follow the outcome measures in [-4 +4] years window. The outcome variables are measure in inverse hyperbolic sine points. The tick(thin) lines are 90\%(95\%) confidence intervals. The specification controls by municipality-election and election-year fixed effects. 
}

%inverse hyperbolic sine (IHS) transformation


\newcommand{\stars}{
Standard errors clustered at municipality level.  *** p<0.01, ** p<0.05, * p<0.1.
}

\newcommand{\census}{
}

\newcommand{\household}{
}


    %all table footnotes
% To comment out multiple lines of text.
\long\def\comment#1{}

\usepackage{guthesis}

\title{Essays on the Political Economy and Economic Impact of Fiscal Policies}

\author{Daniel Valderrama-Gonzalez}

\previousdegree{M.Sc.}

\thisdegree{Doctor of Philosophy}  % or Doctor of Philosophy, etc.

\thisdiscipline{Economics}

\thesistype{Dissertation}     % or Dissertation

% defense or approval date, not today's date...
\thesisday{19}
\thesismonth{April}
\thesisyear{2022}

\professor{Laurent Bouton, PhD}
\secondprofessor{Martin Ravallion, PhD}   % Only if you have 2 major professors!

\fulltitle{Full Title}

\indexwords{Political Favoritism, Place Based Policies, Infrastructure Earmarks, Defense Spending, Local Fiscal Multiplier, Rent Seeking, Mexico, United States}

\dean{Timothy A.\ Barbari}

\memberi{First I.\ Last}
\memberii{First I.\ Last}
% Use \memberiii, \memberiv, \memberv for up to 3 more members if needed.

\begin{document}

\pagenumbering{roman}

\maketitle   

\begin{abstract}

Economists have the opportunity and the responsibility of helping to design public policies. In the policy design, we need to have clear both the big picture and the details. The details convert a good policy in paper to a successful implemented policy in practice. 

This dissertation is all about the details! It shows how the economic efficiency of fiscal policy, measured by the size of the fiscal multiplier, is affected by three overlooked \enquote{details}, namely: political favoritism, elite capture and composition of government purchases. 

In the first chapter, \enquote{\textit{The Unintended Consequences of Political Alignment: Evidence from Mexico}}, I study the economic effects of politically motivated spending. I use a close-election research design that exploits variation in political alignment, between governors and mayors, during a period when political alignment implied a substantial increase in intergovernmental transfers. I find that political alignment increases intergovernmental transfers and public spending while slowing down private-sector employment. I find suggestive evidence that this slowdown is explained by the opportunity cost of reallocating economic activity toward rent-seeking activities. 

In the second chapter, \enquote{\textit{Distributional Effects of Intergovernmental Transfers in Mexico}}---co-authored with Carlos, Kiyomi, and Laura---,
we estimate the welfare effects of placed-based policies (PBP) that are targeted at lagging economic areas. We study the case of the Fund of Social Infrastructure (FAIS) in Mexico, one of the largest PBP in the world that provides infrastructure-earmarks to lagging economic areas. We use a simulated instrument research design and find that an increase of FAIS translates into an increase in the coverage of social infrastructure (e.g. electricity, sewerage, piped water). However, the benefits in terms of infrastructure are not followed by higher household income or lower monetary poverty. The missing effect of FAIS on welfare is explained by the bulk of the economic gains being captured by the non-poor residents of the initially targeted poor places.

In the third chapter, \enquote{\textit{Heterogeneous Spending, Heterogeneous Multipliers}}---co-authored with Pedro and Umberto---, we ask whether the size of the local employment and earnings multipliers depends on the composition of the government purchases. We answer this question by building a panel of military spending at the product-MSA-year level. We use this dataset in a shift-share research design to exploit the heterogeneous sensitivity of local military spending to national military buildups and drawdowns. We find that the goods and services that the government purchases determine the size of the local fiscal multiplier. Local fiscal multipliers are larger when public spending focuses its demand on labor-intensive industries.


\end{abstract}


\chapter*{Dedication}
Our achievements are not our own. From womb to tomb, we are in debt to others. I dedicate this work to several people.

First, I dedicate this to Sanchi, also known as Diana. She did not need to read a single line of this work to give me the love, courage, and spirit I needed to continue pushing through it. Also, this has to be dedicated to Marcelo and/or Danielle.

Second, I dedicate this to my family. Pa (Jaime), Ma (Estella), Pipe and Kta Gracias! Every effort, sacrifice, and mistake you made brought me here, so the degree is only yours. 

Third, only a few hundred out of millions of people with the same opportunities I had obtained a Ph.D. Some would say it was because of my effort, madness, or good luck. I would say it was a good God. I dedicate it to you God.

\chapter*{Acknowledgments}

First, I want to thank my advisors. I am indebted to Laurent. A mentor of few words but who gave me uncountable lessons. Thanks for showing me the humanity behind being a researcher, never stopping believing in me, and always being there. There is no perfect advisor, but he was close to being perfect. I want to also thank Martin for his advice in the moments when I needed them most and for sharing with me the \textit{treasure} of his knowledge of poverty and inequality. I want to also thank Carlos for his unconditional support and mentoring before, during, and after the Ph.D.

I want to also thank my committee members. Thanks to Garance for her genuine interest and efforts in my job market. I would have loved to have her as co-advisor from the beginning. Also, I thank Toshi for adopting me into his reading group, for his integrity as a person, his wisdom in the art of research, and his sense of duty. 

Thanks to the family that Georgetown gave me, I would never make it without them. They were there for me for the comps, the proposal, the parties, the thanksgiving, and the unexpected rollercoaster of the job market. Thanks to  Pedro, Rodi, Carolina, Mari, Umberto, Juan, Arturo, JJ, Kevin, Linis, Jacq, Minji, Deno, Bingxi, Sub, Gaurav, Mariel, Madhu, Allison.

Thanks to friends and mentors who were spread all over the world. All the conversations with you reminded me of why I got into graduate school and gave me the energy to believe that I could do it. Thank you Raul Andres, Andres, Jose Daniel, Thiago, German, Checho, Dario, July, Adri, Paul, Jorge, Hector, Del, Maria Isabel, Monik, Ricardo, Joana, Niko and Juana. 

I would love to thank my family. Gracias a Pa (Jaime), Ma (Estella), Pipe and Kta. I am just the consequence of their love, mistakes, and sacrifice. Impossible to tell my story without telling theirs. The saying goes, "...you do not choose your family" well, I would not have chosen a better one. Thanks for waiting for me, for taking care of yourselves when it was my responsibility. 

Finally, again and again, I thank God. 

%Many classmates and faculty at Georgetown provided invaluable feedback and%support. Thank you especially to Pedro, Umberto, Carlos,  all my co-authors, Pedro, Umberto, Pedro and UmbertoDario Sansone and Mike Packard. I would also like to acknowledge Nicola Persico, Philip Marx, Gaurav Bagwe, and Juan Margitic for useful feedback on previous drafts of these chapters

%\pseudochapter{Preface}

%A preface is not an introduction, and most theses do not need them.


\tableofcontents

\listoffigures  % Optional - Omit this line if you don't want a list of figures.
\listoftables   % Optional - Omit this line if you don't want a list of tables.

\newpage

\pagenumbering{arabic}  % Ordinary pages have Arabic numerals.


\chapter{The Unintended Consequences of Political Alignment: Evidence from Mexico}\label{chap:c1}

\subfile{1_introduction.tex}
\subfile{2_Institutional_Context.tex}\label{sec:context}
\subfile{3a_Identification.tex}\label{sec:identification}
\subfile{4_Data.tex}\label{sec:data}
\subfile{3b_validity.tex}\label{sec:validity}
\subfile{5_Results.tex}\label{sec:results}
\subfile{5c_alt_emp.tex}
\subfile{6_Channels.tex}\label{sec:channels}
\subfile{6a_Economic_channels.tex}\label{sec:channels}
\subfile{6b_Channels.tex}\label{sec:channels}
\subfile{7_Conclusions.tex}

\chapter{Distributional Effects of Intergovernmental Transfers in Mexico}\unmarkedfntext{This chapter of my dissertation is part of a joint research project with Kiyomi Cadena, Laura Moreno-Herrera \& Carlos Rodríguez-Castelán.}\label{chap:c2}

\subfile{Chapter_2.tex}

\chapter{Heterogeneous Spending, Heterogeneous Multipliers}\unmarkedfntext{
This chapter of my dissertation is part of a joint research project with Pedro Juarros and Umberto Muratori.}\label{chap:c3}

\subfile{Chapter_3.tex}

\appendixtocoff
\appendices

\chapter{Appendix Chapter 1}
\newpage
\section{Figures and Tables}
\subfile{figures_c1.tex}\label{FirstAppendixC1}
\newpage
\subfile{tables_c1.tex}\label{SecondAppendixC1}

\newpage
\chapter{Appendix Chapter 2}
\section{Figures and Tables}
\subfile{figures_c2.tex}\label{FirstAppendixC2}
\newpage
\subfile{tables_c2.tex}\label{SecondAppendixC2}
\newpage
\section{Online Appendix}
\subfile{online_appendix_c2.tex}\label{ThirdAppendixC2}
\newpage


\chapter{Appendix Chapter 3}
\section{Figures and Tables}
\subfile{figures_c3.tex}
\newpage
\subfile{tables_c3.tex}
\newpage
\section{Online Appendix}
\subfile{online_appendix_c3.tex}



\nocite{*} 
        % To show all references. (not cited on the document, ideally it's just to show them initially.)



    \bibliographystyle{plainnat}
    \bibliography{BibFile}  % thesis.bib



\typeout{***}
\typeout{*** Note!}
\typeout{*** Because this document has a table of contents,}
\typeout{*** you must run LaTeX TWICE to get it to print correctly.}
\typeout{***}

\end{document}

