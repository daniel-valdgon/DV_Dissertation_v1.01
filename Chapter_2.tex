\documentclass[dv_diss_main.tex]{subfiles}

\usepackage{comment}
        % To comment a Full Stack.

\begin{document}

\section{Introduction} \label{sec:Introduction}

Over past decades, several developing countries started a process of decentralization of public spending by transferring earmarked resources to subnational governments with the objective of providing a wide range of public goods and services, including health care, education, roads, and social infrastructure. There is a large body of evidence that this reform process has resulted in aggregate gains in the stock of public goods. However, there is less consensus on the distributional implications of  decentralization, particularly on who benefits more from these transfers.  While the informational and accountability gains of decentralization may lead to better targeting and pro-poor public spending, the vulnerability of local politicians to elite capture can result in opposite effects, leading to greater concentration of benefits for the non-poor.

To contribute to this debate, this paper answers the following two questions: What are the impacts of earmarked transfers on aggregate income and public service delivery? And what are the multiplier effects of these transfers along the income distribution? To respond to these questions, we leverage a novel municipal level dataset with information on municipal income distribution and public finances in Mexico over the last 25 years. We exploit plausible exogenous variation in the formula used to assign a large  earmarked-transfer, known as FAIS\footnote{The FAIS (Fondo de Aportaciones para la Infraestructura Social) is the third largest earmarked fiscal fund going to subnational governments in Mexico, but the largest going to municipalities. The objective of FAIS is to finance investments in infrastructure to benefit the poor at the municipal level. It is the largest source of earmarked income among municipalities. In 2014, 22 percent of the total fiscal funds received by municipalities originated from the fund; a larger share went to poorer municipalities, equivalent to total expenditures of US $\$230$ billion in 2011 purchasing power parity US dollars. The FAIS is equivalent to 2.53 percent of assignable federal tax revenues.}, across municipal governments.

The objective of FAIS is to transfer fiscal funds from the national government to municipalities for investment in social infrastructure (including urbanization, water and sanitation, housing, health, and education) with a formula to distribute resources that favors municipalities with higher poverty rates. Hence, any analysis that fails to account for such endogeneity would likely yield biased estimates of the impact of FAIS on local welfare.\footnote{The formula builds on the global poverty index in each municipality based on the density of inhabitants with incomes below an extreme poverty line or large deficits in education, housing, sanitation, and electricity.} To mitigate this issue, we implement an instrumental variables strategy that exploits the statutory variation in the federal formula published when FAIS was launched in 1998 for the allocation of the transfers across municipalities. Because the formula is a nonlinear function of the distribution of long predetermined municipality characteristics, we are able to control for the independent impact that these characteristics may have on contemporaneous outcomes of interest. In addition, since it is probable that municipalities receiving larger shares of FAIS according to the formula are on a different trend relative to municipalities receiving lower levels of  FAIS,   we control directly for pre-trends in our outcome of interest to address this additional source of bias.

Our results show a significant positive effect of the FAIS on most nonmonetary measures of well-being, precisely linked to the main categories of investment of the transfer, such as access to basic services (electricity, sewerage, piped water) and quality of housing (floors). In particular, an increase of 10 percent in the FAIS is associated with a coverage increase of 0.10, 0.77, and 0.57 percentage points in electricity, connection to sewerage, and access to piped water, respectively. On the quality of housing, an increase in 10 percent in the FAIS has an effect of 0.48 percentage point increase in the quality of floors, but no significant effect on access to sanitation.

We fail to reject the null hypothesis of no effects of the FAIS on monetary poverty.  Our results show that the main reason behind this puzzling result is the fact that the majority of the monetary gains generated by the FAIS are captured by non-poor residents of poor places. In particular, our results show that a 10 percent increase in the FAIS had a larger positive effect of 0.57 percent on the top decile of the income distribution, but no effect on other deciles. The FAIS thus accounts for a 0.21 increase in the Gini coefficient.

To explain this result, we look at treatment heterogeneity by area of residence. In contrast to the overall results nationwide, a 10 percent increase in the FAIS was associated with a modest reduction in food poverty in urban areas of 0.17 percentage points. The results in rural municipalities were not significantly different from zero. These results are consistent with previous descriptive analysis done by \cite{wellenstein2006social}. They observe that the infrastructure projects funded by the FAIS were usually made in the periphery of cities, which tend to have a higher share of low-income residents. In contrast, because of the nature of the social infrastructure investments financed by FAIS which rely on economies of scale (e.g. piped water  and sewerage connections), the FAIS resources in rural and semiurban areas, which comprise the largest number of municipalities in Mexico, are usually invested in \textit{cabeceras} (similar to county capitals), where people with relatively higher income live. Our findings are also consistent with the findings of \cite{smith2018aumento}, who shows that local governments with more within-state capacity are the ones that can take better advantage of public investment and show greater convergence in Mexico, which are precisely those governments in urban areas.

Our analysis highlights both the positive aspects and the unintended consequences of an earmarked infrastructure funds such as the FAIS. While the FAIS resulted in increased average monetary and non-monetary welfare across municipalities, it also has an unequalizing effect in monetary outcomes resulting on an increase in local income inequality. These results call for complementary policies to ensure that the poorest households in rural and remote areas also benefits from investments in social infrastructure.

\iffalse
\begin{comment}
\textbf{Related literature1.---} This paper contributes to several strands of the literature. On the discussion about the type of resources and public service delivery. We show the limits of earmarked transfers. 
Need to build a section that compares our estimates .... Martinez-COL and Litschig-BRA , other in Africa and other in Asia

In a similar paper on Brazil, \cite{litschig2013impact} find that intergovernmental transfers had a positive effect in schooling and literacy rates, as well as on lower poverty rates.

*Come back to education outcomes and may be mortality rates

\textbf{Related literature2.---}
On the impact of transfer multipliers/ better poverty!!!!!. We find very little household per-capita income increasing, 

Look other studies besides Litschig-BRA, RAvi !!!!!
i not it is still open the idea  of looking at income 

\textbf{Related literature3.---}
On the literature that focus on decentralization and state capture of local public goods. We shows how private gains can be concentrated 
What can be said here???
\end{comment}
\fi
The next section presents a brief literature review of the empirical research on the effects of federal transfers on local welfare in various parts of the world. Section~\ref{sec:institucinal} describes the FAIS and offers background information. Section~\ref{sec:data} outlines the data and the sources used in the analysis. Section~\ref{sec:strategy} details the methodology of the analysis. Results and robustness are presented in Sections~\ref{sec:results} and \ref{sec: Robustness}, while Section \ref{sec: Concluding remarks} concludes.



\section{Literature Review} \label{sec:literarure}

Although the intergovernmental transfer fund studied here is earmarked, the paper refers to the decentralization literature to the extent that the FAIS allows ample discretion to municipal governments in the type of public good they spend this transfer,and, more importantly, municipalities select the location of those projects. Therefore, our findings are relevant to the literature on the distributional effects of fiscal federalism and decentralization \citep{oates1972fiscal,musgrave1983should,wallis1988decentralization,ostrom1993relational,litvack1998rethinking,habibi2003decentralization,jimenez2011impact,martinez2015impact}. Our evaluation adds to previous studies because it focuses on a wide array of welfare outcomes, and it is performed on a larger set of smaller administrative units relative to previous studies, that is, among municipalities instead of states. In a similar paper on Brazil, \cite{litschig2013impact} find that intergovernmental transfers had a positive effect in schooling and literacy rates, as well as on lower poverty rates.

A vast literature documents the importance of and the challenges involved in increasing government revenues in developing countries (for example, see \cite{besley2014developing}). The way in which revenues are spent—programmatically, as well as by whom and where—is also important. During the 1980s and 1990s, several countries around the world began a shift toward decentralization by transferring earmarked resources to subnational governments. The objective was to allow subnational governments to provide public services locally, including health care, education, and social infrastructure, such as water and sanitation services or support for urbanization \citep{conning1999community,jimenez1999community,bardhan2005decentralizing}.

The relative failure or success of decentralization has been associated with various factors. These include the design and implementation of the reforms, the incentives facing local and national stakeholders, local accountability and responsiveness, the budgetary process, and the availability of adequate local capacity \citep{ahmad2005decentralization,robinson2007does,martinez2011decentralization,loayza2014more}. 

There is also a large literature on the effects of federal transfers on local socioeconomic outcomes. The bulk of this literature focuses on variations in transfers that are indexed to the value of natural resources that are taxed and produced locally \citep{loayza2013poverty,enamorado2014regional,zambrano2014global},but findings of these studies may be associated with external validity problems related to the impact of transfers in non-mining districts. A few studies focus on transfers that are allocated to distressed areas \cite{araujo2008local}or offer compelling casual evidence about the effects of federal transfers \cite{caselli2009oil}; and \cite{corbi2019regional}. 

In general, the empirical literature on the effects of decentralization on income distribution is inconclusive. A summary of the literature carried out by \cite{jutting2004decentralisation} finds that the effects of decentralization on poverty were positive in some cases, including China, Ghana, Philippines, and South Africa, but null or somewhat negative in others, including Brazil, Egypt, and India.

In Mexico, most studies have been devoted to the two largest earmarked fiscal funds transferred to subnational governments. One, corresponding to expenditures for health care, is the Health Care Services Transfer Fund (Fondo de Aportaciones para los Servicios de Salud, FASSA) \citep{moreno2001descentralizacion,merino2003descentralizacion,molina2014decentralization,martinez2011decentralization}. The other, in education, is the Basic Education Transfer Fund (Fondo de Aportaciones para la Educación Básica y Normal, FAEB) \citep{esquivel1999gasto,latapi2000financiamiento,avendano2012evaluacion}. But a significant difference of these two fiscal funds with respect to FAIS is that these are only transferred to states governments rather than municipalities.

On the FAIS, \cite{guadarrama2008gasto} shows that, in rural municipalities, about half the FAIS resources in 2004 were allocated to the municipality \textit{cabeceras} (similar to county capitals), which is rarely the location of the most disadvantaged communities.\footnote{ In Mexico, states are the first-level administrative unit. The second level is represented by municipalities. The \textit{cabecera} is the location designated as the capital of a municipality.} \cite{ramones2014efectos} find a weak positive relationship between the resources allocated by the FAIS and the reduction of asset and multidimensional poverty measures across states in 2000–10.\footnote{ The analysis is performed across states, which significantly reduces the number of observations from 2,443 municipalities to 31 states, and, thus, the explanatory power of the model. Additionally, the study does not control for the unobservable and invariant characteristics of states. Rather than taking advantage of the panel data of states (with 62 to 93 total observations), the study uses a pooled regression analysis and aggregates the same units of measurement over time.}

Several studies on fiscal federalism in Mexico have emphasized the problems behind the lack of an effect of fiscal transfers on poverty reduction. Inadequate subnational political accountability has been identified as the main reason for the lack of results in the decentralization of the efforts to relieve poverty \citep{diaz2016fiscal,hernandez2016poverty}. Other factors include: voters being unaware of mayoral responsibilities \citep{chong2015does}, and concentration of power at the local level \citep{smith2016micro}. 

To contribute to the literature on the distributional effects of intergovernmental fiscal transfers in developing countries and particularly in Mexico, we conduct a detailed analysis on the impacts of the FAIS on local income distribution and investments in social infrastructure. 

\section{Institutional Context} \label{sec:institucinal}
\subsection{General Context} \label{subsec:general}


After a long history of centralized administration in Mexico, the federal government unveiled a series of institutional reforms during the 1990s to decentralize provision of basic services. In particular, responsibilities of federal spending and investments began to be carried out through earmarked transfers administered by states and municipalities. In 1997, a range of intergovernmental transfer funds were created under a new national budget line referred as the Federal Transfer Fund for States and Municipalities umbrella, known as Ramo 33. The FAIS is one of the two earmarked transfer funds that were designated for decentralized investment decisions by municipalities.\footnote{Technically, among FAIS resources, 88 percent are managed by local governments, and federal states administer 12 percent. The other program that decentralizes spending decisions to local governments is known as FORTAMUN (Fondo de Aportaciones para el Fortalecimiento de los Municipios, Contribution Fund for the Strengthening of Municipalities). The allocations under this fund were proportional to population.} Also, unlike other programs of Ramo 33, the FAIS allocations across local governments were not constrained by previous commitments of public spending from the pre-decentralization period, such as in the case of intergovernmental transfers to finance public education and health care.\footnote{ The other main programs of Ramo 33 are the FAEB education fund and the FASSA health care fund, which, together with FAIS, represent about 77 percent of Ramo 33 proceeds. Because these resources are allocated to pay public workers in the education and health sectors, the actual allocations of spending across municipalities did not changed as result of the fiscal decentralization of 1998. By contrast, the new rules governing the FAIS have changed the allocation of infrastructure spending completely.}

The annual FAIS proceeds amount to 2.53 percent of the assignable federal tax revenue.\footnote{The assignable federal tax revenue (\textit{recaudación federal participable}) is defined as the taxes collected through the federal income tax, the value added tax, and ordinary taxes and fees associated with oil extraction and production. In practice, with the exception of the extraordinary fees on oil and other extraordinary fiscal revenues of the federal government, most of the federal taxes are subject to sharing with subnational governments through earmarked and unconditional fiscal transfers.} The FAIS is distributed to all municipalities and is earmarked to infrastructure projects that directly benefit populations that are living in conditions of extreme poverty and social deprivation. According to the rules of operation, there are a wide variety of project categories for which these resources can be assigned, such as those improving households’ access to basic services (for example, water, sewerage, and electrification), those involving the construction of health care clinics and schools, and those aimed at promoting local economic development through roadbuilding, urbanization and other productive infrastructure investments (See Appendix~\ref{ap:faisspending}  for a complete list of FAIS catalog of spending). Thus, although the FAIS is an earmarked transfer, it gives a substantial degree of freedom to local governments in the administration of resources, and this offers a relevant case study in the literature on the welfare effects of the decentralization of public spending.

Articles 34 and 35, which were added to the Fiscal Coordination Law (FCL) on December 29, 1997, provide a precise formula for the allocation of the FAIS resources across states and across municipalities. In 2013, this formula was amended, although the new allocation rule did not alter significatively the allocation system under the previous rule after 2014. Therefore, the initial formula adopted in 1998 governed the period under study here, that is, 2005–15.\footnote{The modification to the formula in 2013 did not abruptly change the allocation system under the old formula for two reasons. First, the new formula was only applied on additional resources received after 2013. Second, the main difference between the old formula and the new one required information from the 2015 population count, which was not released until 2017.} 

  Compliance with the FAIS formula also occurred gradually since the establishment of this intergovernmental transfer fund. Between 1998 and 2002, the share of municipalities that reported not receiving the amount of the FAIS transfers that they were supposed to receive based on the formula of distribution oscillated between 26 percent and 90 percent (Panel A of Figure~\ref{fig:des1A}). Evidence suggests that the missing transfers may be partially explained by political factors.\footnote{\cite{diaz2016political} and \cite{Valderrama} find evidence that FAIS transfers were allocated according to political factors during the first years of the system. Anecdotical evidence is offered by the Supreme Court case that came to be known as the Bartlett Law, which aimed to solve a dispute between the governor of the state of Puebla and the federal government that revolved around a request of the governor that the FCL be changed to allow discretionary decisions in the allocation of the FAIS.} But after 2002, compliance increased substantially. 

Beyond political economy reasons, there are two additional explanations of why the compliance with the formula to distribute fiscal resources to municipalities was limited during the initial years of the implementation of the FAIS. First, the federal government created a transitional period between 1998 and 2001 during which the formula was only partially followed. During this period, a share of the FAIS budget was equally distributed to all states, while the rest was supposed to be distributed following the formula. The transitional period ended in 2002.\footnote{ \cite{mogollon2002discrecion} and \cite{diaz2016political} document the transition period, during which a fraction of the FAIS was distributed equally across all states instead using the formula. This transition period ended in in 2002, when the entire FAIS pot was allocated on the sole basis of the formula. Although this process occurred at the state level, states may have allocated across their municipalities in a similar fashion.} Second, article 35, added to the FCL in 1997, allowed states to delay the implementation of allocation of resources to municipalities based on the formula until they considered to have sufficient data to implement the formula properly. In case a state opted to delay the application of the formula, it was required to follow a predetermined alternative formula.\footnote{This alternative formula was a simple average of four municipal characteristics measured in the latest available population census: the population earning less than two minimum wages, the illiterate population, sewerage coverage, and electricity coverage.}

Panel B of Figure~\ref{fig:des1A}, shows trends in the main intergovernmental transfers introduced during the spending decentralization of 1998. The average transfers from the FAIS grew at a rapid pace following the introduction of decentralization in 1998 until 2002. Most of the growth in disbursement of the FAIS over this period arose because of changes in the extensive margin, particularly a reduction in the proportion of municipalities that reported not receiving any funds described in panel A of Figure~\ref{fig:des1A}.

The second period of expansion of the FAIS took place starting in 2005. The annual resource allocations of the FAIS rose from Mex\$30 billion to Mex\$42.6 billion between 2005 and 2015. Cumulative public spending represented about 2.4 percent of the country’s GDP in 2015.

As of 2015, the FAIS represented the main revenue source for building public infrastructure across most municipalities in the country. Municipalities are typically characterized by low fiscal capacity.\footnote{ The other main source of revenues for municipalities with low levels of fiscal capacity correspond to unconditional intergovernmental transfers (\textit{participaciones}; Ramo 28 of the federal budget)}According to the 2015 marginalization index, defined by the National Population Council of Mexico, the average share of the FAIS transfer in total revenues for municipalities with a high level of marginalization was 47.1 percent. The share was 11.9 percent among municipalities with a low level of marginalization.

Several aspects of this section are important in explaining the identification strategy in Section~\ref{sec:strategy}. First, the formula was set in December 1997 by the federal government and practically did not change over the period under study. Second, we also exploit the variation of a large growth of FAIS transfers that took place after 2005. Third, FAIS transfers are particularly important for municipalities with low levels of local development. We therefore expect to find effects on local aggregate welfare.



\subsection {Policy Allocation Rule } \label{subsec:Policy}

The formula for the distribution of the FAIS resources to subnational governments described by article 34 of the FCL is computed and published every year by the federal government. According to the law, it should be based on the information available through the latest population census that we will index by $k$, which can be expressed as function of the year $t, k(t)$. The objective of the formula is to measure the size and intensity of poverty in each municipality. Allocations are computed in two steps. First, a measure of deprivation is computed at the household level. This is determined through a nonlinear function of the joint distribution of five dimension of well-being defined at household level: household income, educational attainment, household crowding, sewerage, and cooking fuels. Each dimension uses as input at least one household characteristic. Second, the deprivation measure at household level is aggregated across households to obtain a measure at the municipality level. ´

Equation~(\ref{eq:1}) shows the first step. Where $D_{j,m,k(t)}$ is the deprivation mass of household $j$, living in municipality $m$, interviewed in the population census $k(t)$. $\omega^g$ is the policy weight for the dimension $g$. $W_{j,m,k(t)}^g$ of household characteristics used to compute dimension $g$ for household $j$, interviewed in the population census $k(t)$. $f_g(.)$ is a step function that maps characteristics into a normalized measure\footnote{ Normalize to range between -0.5 to 1, where higher values are higher levels of deprivation.}  of household deprivation for the dimension g (this is the first non-linear transformation of households’ characteristics).\footnote{Technically, one of these characteristics (poverty) depends on parameters that are time variant (poverty line), but, for the sake of simplicity, we ignore this from the notation now. It will be introduced below.}$^{,}$\footnote{See Appendix~\ref{ap:faisformula} for details on $\omega^g,W_{j,m,k(t)}^{g}$ and $f_g (.)$}
\begin{equation}\label{eq:1}
D_{j,m,k(t)}=\sum_g \omega^g f_g \big(W_{j,m,k(t)}^{g}\big)   
\end{equation}
Equation~(\ref{eq:2}) shows the second stage. To compute the deprivation at municipality level, it is necessary to sum over the squared of the household’s deprivation mass distribution truncated from below by 0, i.e., only sum across household with positive deprivation mass. The sum weights each household by the number of household members $n_{j,m,k(t)}$. 
\begin{equation}\label{eq:2}
D_{m,t}=\sum_{j,j\in m} D_{j,m,k(t)}^2 \times 1(D_{j,m,k(t)}>0) \times n_{j,m,k(t)} \end{equation}
The squared and the indicator function imply a second and a third non-linear transformation over the initial vector of formula’s inputs $W_{j,m,k(t)}$. The equation~(\ref{eq:3}) shows how to use $D_{m,t}$ to compute the percentage of FAIS resources in year t assigned to each state-- $S_{s(m),t}$, and municipality--$M_{m,t}$:
\begin{equation}\label{eq:3}
S_{s(m),t}=\frac{\sum_{m,m \in s}D_{m,t}}{\sum_m D_{m,t}}\;\;\;\;,\;\;\;\; M_{m,t}=\frac{D_{m,t}}{\sum_{m,m \in s}D_{m,t}}   
\end{equation}
\noindent two aspects of the formula are fundamental for our identification strategy. First, the formula that allocates resources in period $t$ is computed according to long predetermined characteristics based on the latest available census $k(t)$. This implies that, in our period of analysis (2005–15), the formula inputs are at least five years apart from the outcomes of interest. Second, the formula depends on the joint distribution of an ad-hoc non-linear function of predetermined characteristics (see equations (\ref{eq:1}) and (\ref{eq:2})), which is a source of variation that is independent of any linear combination of municipal measures of the vector of formula’s inputs $W_{m,k(t)}$.

Beyond cross-sectional variation, the formula also includes a within–municipal variation from two sources: first, because one of the dimensions used to compute the deprivation mass includes a policy parameter, a federal poverty line, which changes every year; and, second, because of changes in the sources of the formula inputs, which occurs each time a new population census becomes available. (See Figure~\ref{fig:B1} and ~\ref{fig:B2} in Appendix~\ref{ap:overtime})

\section{Data} \label{sec:data}

Our main data consist of a panel of comparable poverty maps for municipalities merged with administrative records on local public finance. The panel includes information on 2,120 municipalities (of 2,443 municipalities across Mexico) for 1990, 2000, 2005, 2010, and 2014.\footnote{ We exclude 307 municipalities because either they have missing FAIS data (121) or low precision in their small area income estimates (186). To flag unprecise SAIE we use the coefficient of variation of the mean household income. Main results are quantitatively similar with all municipalities with non-missing FAIS transfers, actually results on the reduction of poverty improve when all observations are considered (results available upon request). Table \ref{poly_inf} compare descriptive statistics of the observations that are deleted, they tend to be poorer, more rural and with lower levels of education.} Data come from three main sources: (a) nonmonetary welfare variables on access to basic services and social infrastructure from population censuses and intercensalus population counts and surveys; (b) household income, monetary poverty, and inequality measures from comparable poverty map estimates; and (c) public finance data from administrative records.

\subsection {Nonmonetary Poverty Measures} \label{subsec:non}

Nonmonetary measures during our period of study come from population censuses and intercensal counts in 2005, 2010, and 2015. To measure trends before the policy change, we use the population censuses of 1990 and 2000. The relevant indicators for social infrastructure include (a) the share of the population with access to basic services in their dwellings (electricity, piped water, sewerage) and (b) the share of the population with adequate quality floors and sanitation in their dwellings. We also use population censuses and the closest economic censuses for each population census (1999, 2004, 2009, and 2014) to construct control variables and for robustness exercises.

\subsection {Longitudinal Data on Local Monetary Poverty and Inequality Indicators}\label{subsec:monentary}

In Mexico, poverty, inequality, and income deciles are available at granular geographical levels in contrast to most developing countries. The National Council for the Evaluation of Social Development Policy (CONEVAL) adopted a methodology on small area estimation that produces statistically accurate information on several points of the income distribution in small areas in Mexico.\footnote{The poverty mapping technique, developed by \cite{elbers2003micro}, allows the estimation of poverty at high levels of disaggregation. Several steps are required to apply the technique successfully. First, income or consumption econometric models are estimated using microdata from household surveys. Second, parameters estimated in the first step are exported to the census data to obtain predictions of household welfare, which are not available in the census data. Finally, poverty measures are estimated for specific geographical areas based on the welfare measures predicted in the second step} CONEVAL, jointly with the World Bank, has produced a panel of municipal poverty maps for 1990, 2000, 2005, and 2010. To complement this time series, the World Bank has produced a comparable municipality-level poverty map for 2014 combining the intercensal survey of 2015 with the 2014 household survey. This long-term panel of poverty maps allows an analysis of local trends and drivers of income, poverty, and inequality in Mexico.

Following the definitions used by CONEVAL, we measure poverty using three poverty lines: (a) the food poverty line (FPL)—comparable with an extreme poverty line—which is defined as the income needed to acquire a basic food basket; (b) the capabilities poverty line (CPL)—defined as the income needed to acquire a basic food basket and basic health care and education; and (c) the asset poverty line (APL)—defined as the income needed to acquire a basic food basket, basic health care and education, and basic housing and transportation.

Between 2000 and 2014, Mexico experienced income convergence across municipalities, but inequality increased within municipalities \cite{lopez2019poverty}. While the average real per capita income growth of the bottom 10 percent of municipalities was 25 percent between 2000 and 2014, the top 10 percent of municipalities grew at an average rate of  7 percent over the same period. Furthermore, the average income growth of municipalities in the 10th percentile and the median was positive between 2000 and 2014, unlike the municipalities in the top 90th percentile for which the average income of declined in real terms between 2005 and 2010 (See Figure~\ref{fig:des2}). In contrast, local income inequality within municipalities increased between 2010 and 2014. Before 2010, most of the inequality reduction occurred in municipalities with higher income per capita, while, in 2010–14, less than 20 percent of the municipalities in the highest quantiles experienced a reduction in the Gini coefficient (See Figure~\ref{fig:des3}).



    \subsection {Administrative Data on Public Finances} \label{subsec:adminis}


We use administrative data on public expenditure among municipalities from the State and Municipal System Databases (SIMBAD) produced by the National Institute of Statistics and Geography (INEGI). SIMBAD provides access to public expenditure variables from 1988 until 2015, including investment spending. Our analysis aggregates the fiscal accounts into the following groups: (1) own fiscal resources, including local taxes and other fees; (2) unconditional intergovernmental transfers; (3) earmarked conditional transfers, excluding the FAIS; and (4)  the  FAIS  (A detailed breakdown of the fiscal accounts is presented in {Appendix}~\ref{Ap:fiscalcategories})

To minimize any possible effect of measurement error of the reported data on the FAIS in our analysis, we generated a variable that aggregates the monthly average of the FAIS per capita for the last three consecutive years of the observation. For example, the variable FAIS in 2013 corresponds to the monthly average per capita FAIS outlay in 2011–13. This three-year average also purges transfers from nonrandom variation because of political business cycles.\footnote{ Local elections in Mexico are staggered. Any specification with time fixed effects will therefore not completely control for the local political business cycle}

The FAIS is the main source of revenue for municipalities with low total revenues per capita (See Figure~\ref{fig:des4}). In 2014, the FAIS accounted for 38.3 percent of the fiscal revenues among municipalities with the lowest own revenues. In contrast, the FAIS represented only 5.8 percent of the resources in municipalities with the highest values of own fiscal revenues.



\section {Identification Strategy} \label{sec:strategy}
\subsection {OLS Specification} \label{subsec:ols}


Equation (\ref{eq:4}) allows to get the Ordinary Least Square (OLS) estimates of the effect of the FAIS on the outcomes of interest---$Y_{m,t}$, as follows:
\begin{equation}\label{eq:4}
Y_{m,t}= \beta\; FAIS_{m,t-1}+\delta_t+\gamma_m+\epsilon_{m,t}   
\end{equation}
\noindent where the policy variable, $FAIS_{m,t-1}$, is measured as the log of per capita FAIS transfers received by municipality $m$ during the year that precedes $t$.\footnote{As explained in Section~\ref{sec:data}, the measure of FAIS for year t is the actual average spending in the last three years. For example, the variable FAIS in 2013 corresponds to the monthly average per capita of FAIS in 2011–13} This lagged structure allows for some delay in the emergence of the effect of the FAIS transfers in our outcomes of interest and also helps mitigate concerns about reverse causality.\footnote{The results are robust to the definition of averages at different lengths. Notice that we did not implement a fully flexible distributed lag model because this cannot be combined with an instrumental variable setup. A distributed lag will lead to an exactly identified model of several endogenous variables with several instruments (as many as the added leads and lags). The asymptotic distribution of more than three instruments has not been completely characterized, thereby creating inference problems} Since the specification includes municipality--$\gamma_m$, and time fixed effects--$\delta_t$, we identify $\beta$ by exploiting within municipality variation in FAIS transfers over time.

A potential concern about the estimates of equation (\ref{eq:4}) is that politicians could, for example, use the FAIS to mitigate the effects of economic shocks that also affect the outcomes of interest, leading to a downward bias in the estimates. Also, given the \textit{de facto} discretion that state governments have over the allocation of FAIS, we could expect that they will target municipalities based on political traits that affect the electoral returns each municipality has to offer. In that case, the OLS estimates could be picking up the effect of the political capital of the municipalities, leading to an upward bias in the estimates. To mitigate the effects of this source of endogeneity, we propose an instrumental variable research design in the next subsection.

\subsection{Instrumental Variables Specification} \label{subsec:variables}

We instrument the per-capita FAIS transfers with the amount transfer that should be allocated to each municipality according to the formula described in article 34 of the FCL enacted in 1998.\footnote{The census microdata that are publicly available either through INEGI or International Census Microdata Harmonization Projects (IPUMS) is not designed to provide accurate measures of the formula, because the latter depends on the joint distribution of 7 household characteristics. We obtain the formula coefficients directly from the archives of the Ministry of Development, SEDESOL.} The per-capita law-implied allocation--$Z_{m,t}$, is defined as follows:
\begin{equation}\label{eq:5}
  Z_{m,t}=\frac{FAIS_t \times S_{s(m),t} \times M_{m,t}}{pop_{m,t}}
\end{equation}
\noindent where $FAIS_t$ is the national pot of $FAIS$ in year $t$.
$S_{s(m),t}$ is the law-implied share of FAIS pot that correspond to state $s$ in year $t$, and that is allocated by the federal government.\footnote{This equation implies that the perfect compliance from the side of the federal government could be sufficient to have a strong first stage if the no compliance with $M_{m,t}$ is uncorrelated with variation in $S_{s(m),t}$.} $M_{m,t}$ is the law-implied share of state pot of FAIS that correspond to the municipality $m$ in year $t$, and that is allocated by the state government.

This instrument belongs to the family of \textit{synthetic instruments} as originally proposed by \cite{gruber2002elasticity} to estimate the elasticity of taxable income using statutory variation in tax reforms and recently implemented by \cite{criscuolo2019some} to study the effect of investment subsidies on employment and wages using variations in the formula that allocates the investment subsidies across wards in UK.

Since the formula's inputs are long predetermined municipal characteristics, we close any backdoor path between any unobserved contemporaneous shock and our instrument. The remaining threat to identification is the presence of unobserved persistent shocks that affect both formula inputs and the trends in the outcome variables. To bolster the validity of the instrument, we augment equation~(\ref{eq:4}) with two sets of controls that aim to circumvent for this identification threat, namely, the formula's inputs and a measure to capture pre-policy trends in the outcome of interest.

The two-stage least squares (2SLS) estimates are obtained by the following system of equations. Where equation~(\ref{eq:6}) allow us to obtain our IV estimates---$\beta$, while  equation~(\ref{eq:7}) corresponds to the first stage, where $\pi$ estimate the level of compliance with the allocation of FAIS mandated by law. 
\begin{equation} \label{eq:6}
Y_{m,t}=\delta_t + \gamma_m + \beta FAIS_{m,t-1} + \psi \textbf{W}_{m,k(t-1)} + \rho_t \Delta Y_m^{00} + \epsilon_{m,t}
\end{equation}

\begin{equation}\label{eq:7}
FAIS_{m,t-1}=\kappa_t + \alpha_m + \pi Z_{m,t-1} + \xi \textbf{W}_{m,k(t-1)} + \eta_t \Delta Y_m^{00} + \nu_{m,t}
\end{equation}
\noindent where the identification of our 2SLS estimates comes from within municipality variation in observed and law-implied FAIS transfers over time, since the corresponding set of municipal---$\gamma_m$ and $\alpha_m$, and time fixed effects---$\kappa_t$ and $\eta_t$,  are properly included. As mentioned, to bolster the validity of our instrument, our 2SLS equations adds  $\textbf{W}_{m,k(t-1)}$ and $\Delta Y_m^{00}$ as additional sets of controls. 

 $\textbf{W}_{m,k(t-1)}$ is a vector of the municipal characteristics that are used as inputs to compute the formula-implied transfers that should be allocated period $t - 1$. These formula's inputs are measured, as mandated by the law, by the latest population census available $k(t -1)$. The highly nonlinear nature of the formula affords us the opportunity to control for the direct effect that its input's may have on our outcome of interest. This allows us to have a more credible identification assumption compared to a case in which the instrument were  a linear combination of predetermined characteristics, which is the predominant case in the applied economics literature.\footnote{We refer particularly to the different papers that used shift-share desings to identify causal effects of immigration \citep{tabellini2020gifts}, import competition from China \citep{dell2019violent}, discretionary fiscal spending \cite{chodorow2012does}, exposure to robots \cite{acemoglu2020robots}. The instrument employed in this papers cannot afford to directly control for the independent effect that the predetermined characteristics may have on their outcomes of interest.} A word of caution remains, because by including this controls linearly, we impose an untestable assumption about the functional form between formula’s inputs and the outcomes of interest.

 $\Delta Y_m^{00}$  corresponds to the average change in our outcome of interest between 1990 and 2000.\footnote{Technically FAIS was implemented in 1998, which implies that the period of 1990-2000 has a certain overlap with the post-policy period. Our period choice responds to data limitations regarding pre-policy measures of the local income distribution. Still, we argue that this is a valid pre-policy measure under the assumption that most of the effect of FAIS takes time to materialize and as a consequence the bulk of the variation in our outcomes during the 90’s is not explained by the creation of FAIS.}
 This aims to mitigate the concerns related to correlation between our instrument and long term pre-policy trends. Notice that the corresponding  coefficient---$\rho_t$, varies over time, which allows for the presence of dissipation effects of the pre-trends over time. 


Finally for valid inference, we cluster standard errors at municipality level to control for serial correlation of the impact of FAIS over time. Also, since we test multiple outcomes, we present p values that account for multiple testing in Appendix~\ref{ap:E}. 

\subsection{Instrument Validity} \label{subsec:validity}

To be a valid instrument, $Z_{m,t}$ needs to satisfy three identification assumptions: First, actual allocations should respond, at least partially, to what is mandated by law. Second, it needs to be exogenous once we condition on pre-trends and the formula inputs. Third, the only way the formula can affect our outcomes of interest is through the observed FAIS transfers. Below we develop further each these assumptions.

\subsubsection{Relevance} The relevance of the instrument is proportional to the compliance with the formula described by the law, which should not be taken for granted.\footnote{Section \ref{sec:institucinal} describes several reasons of why we should expect compliance, at least in the allocation from the federal to state goverments. However there are also at least two reasons that may explain deviations from the formula's allocation: First, the FCL allows a certain amount of discretion in the timing of the adoption of the formula by states. Second, political distortions, such as partisan alignment and political business cycles, may lead to imperfect compliance.} {Figure}~\ref{fig:des5} shows statistical evidence of the instrument’s relevance. Panel A shows the fit of a nonparametric regression between the observed FAIS and the law-implied FAIS transfers (both measured as mean-standardized residuals), while panel B shows the distribution of the latter. Panel A suggest a nondecreasing function between the observed FAIS and law-implied FAIS for both specifications, with and without time varying controls. The difference between the 25th and the 75th percentile of the distribution of the residualized law-implied transfers is 23.2 percent, which is similar to the 19.1 percent of difference on the observed FAIS transfers by the local linear regression.


The histogram allows us to illustrate how the variance of our instrument looks before and after adding time-varying controls. Before adding these controls, we have a mean value of the law-implied transfers of 3.8 log points with a standard deviation of 0.17 log points. Once we add time-varying controls, the standard deviation decreases to 0.10 log points.

Panel A of {Table}~\ref{tab:1} reports the estimates of the first stage, described in {equation}~(\ref{eq:7}). Column (1) reports the results of a bivariate regression between the observed transfers and the law-implied transfers. Columns 2–4 present different specifications that add sequentially our baseline controls. It is reassuring that the results of the first stage are strong and quantitatively similar across all specifications. We discuss the results of Panel B and C in the next subsection.


\subsubsection{Conditional Independence} Our instrument exploits variations in the interaction of nationwide policy parameters and long-term predetermined municipal characteristics measured in the latest population census available. Because the policy parameters and sources of information were defined in 1997, we can be confident in not suffering from concerns related to policy endogeneity, that is, a situation in which policy parameters changed because of potential confounders, such as shifts in the federal or state government that tend to favor municipalities with certain traits.

Moreover, we can be confident that the instrument is not affected by contemporaneous political or economic shocks that may affect the demand and supply of the FAIS transfers. Specifically, the cross sectional variation of the instrument depends on the formula's inputs, since these inputs are measured in the latest population census, the gap between unobserved contemporaneous shocks, captured in the residual of the second-stage equation, and the formula's inputs is at least five years. 

A potential identification threat is that municipalities with high and low exposure to law-implied FAIS transfers were in different trajectories before the policy implementation. This correlation between pre-policy trends and our outcomes is expected because, by definition, the formula assigns more transfers to local areas with higher poverty levels, and most of the poorer municipalities did not become poor from day to night but as a result of long term process. 

To circumvent this threat to our research design, our baseline specification controls for both the independent effect of the formula inputs on our outcomes and the pre-policy trends in the outcome of interest. Therefore our conditional independence assumption states that once we condition on having similar pre-policy trends and formula's inputs the remaining within municipality variation in the law-implied transfers is orthogonal to any unobserved variable that may affect our outcome of interest.

We perform two indirect tests that may inform the validity of this assumption. The first test evaluates how controlling only for the formula inputs helps alleviate the strong and expected relationship between the FAIS transfers and pre-policy trends in the outcome variables. The second test explores the sensibility of the first stage to a large set of economic, socio-demographic and political controls. Both tests follow the logic of \cite{altonji2005selection} and \cite{oster2019unobservable} in which changes in point estimates when conditioning on observable confounders is informative about the potential bias that can be produced by the confounders we can not observe.  


{Table}~\ref{tab:2} explores the cross-sectional correlation between pre-policy trends (1990–2000) in the outcome variables and the values of our instrument in 2005. Column (1) highlights the presence of differential trends between municipalities with high and low exposure to law-implied FAIS transfers, which as it was said was expected because of the same objective of the formula. It is reassuring to see in column (2) that most of these correlations become not statistically significant or weakened once we condition on the formula's inputs.\footnote{ The only case where results are neither statistically insignificant or economically irrelevant are for inequality. This suggests that we should pay close attention to the sensibility of the estimates of FAIS on inequality. In the results section, we will see that the estimates of the effect of FAIS on inequality are particularly robust and stable across specifications.} 
These results give confidence to our conditional independence assumption. Particularly we find that an important observed confounder, pre-policy trends, seems not to be a relevant identification threat once control by the formula's inpus and get identification from the non-linear variation of the formula (i.e. after we control by formula's inputs). It is important to reiterate that in our final specification, we directly control by the observed pre-trends; the main point of this exercise was to asses how the strength and direction of those pre-trends before and after conditioning on the formula's inputs. 

The second indirect test to the conditional independence assumption involves observing the sensibility of our first stage estimates to the inclusion of different sets of controls. Panel A of {Table} ~\ref{tab:1} shows the baseline estimates of the first stage coefficients and explores the sensibility of them to include the controls of our main specification across columns. The overall conclusion is that once we control for the formula's inputs, our first stage coefficients are stable. Panel B and C of {Table}~\ref{tab:2} include a large set of time-varying socio-demographic and economic controls. Suppose the variation of our instrument is orthogonal to unobserved contemporaneous shocks partially capture by those time-varying controls. In that case, we should not observe changes in the first stage estimates when comparing across rows within the same column. The striking stability of first stage coefficients makes it harder to argue the presence of an important back-door path between our instrument.

\subsubsection{Exclusion Restriction} The conditional independence of our instrument is sufficient for a causal interpretation of the reduced form (RF) estimates of the impact of statutory FAIS allocations.\footnote{The reduced form estimates are obtained from estimating equation (\ref{eq:6}) using OLS and replacing the observed FAIS with the instrument (law-implied FAIS).} The causal interpretation of our IV estimates will depend on the validity of the exclusion restriction, that is, the formula affects welfare outcomes only through the observed FAIS transfers. This assumption will not hold if the formula is used to allocate other types of revenue sources to municipalities. After a detailed verification on the different laws that define the allocation of other revenue sources and the main federal programs, we can rule out that other transfers to municipalities follow a similar allocation rule. (See  Appendix~\ref{intertransfer})

Our exclusion restriction does not rule out that other public revenue sources may be affected by our instrument, but it only imposes that the instrument affects other revenue sources only through actual FAIS transfers, our endogenous variable.\footnote{ There are several reasons why other FAIS transfers can either crowd out or crowd in other revenue sources. For example, a crowding out may occur because politicians make use of discretionary intergovernmental transfers to compensate the municipalities that receive lower FAIS transfers. On the other hand, a crowding in will be easily explained because the FAIS allows local governments to participate in development projects that are partially funded by the state and federal governments}Still, it is relevant to quantify to what extent our results may be explained by indirect effects, that is, by the potential effect that the FAIS has on other revenue sources. We explore this including other revenue sources as additional control variables.\footnote{ In particular, we control by a set of time varying revenue sources (tax revenues and conditional and unconditional intergovernmental transfers)} As discussed in the section of results , our main estimates are not explained by the indirect effects of the FAIS through an increase in other public revenues.



\section{Results} \label{sec:results}
\subsection{Social Infrastructure Outcomes}\label{subsec:outcomes}

{Table}~\ref{tab:3} presents the results of the effects of the FAIS on social infrastructure. Column (1) shows the OLS estimates obtained through equation~(\ref{eq:4}). The fixed effect estimates show that, between 2005 and 2014, increases in the FAIS allocations were positively associated with increases in several social infrastructure outcomes, such as access to basic services (electricity, sewerage, pipped water) and the quality of the dwelling (the quality of the floors).

Column (2) shows the RF estimates, which are obtained from regressing the outcome variables directly on the law-implied FAIS (plus the same controls used in equation~\ref{eq:6}). The coefficients on access to electricity, connection to sewerage, access to piped water, and quality of the floor are all larger than in column (1), suggesting that the OLS equation is downward-biased.

Column (3) reports the IV estimates that instrument the observed per capita FAIS transfers with the law-implied FAIS, while controlling for the formula's inputs. Column (4) adds pre-trends controls, particuarly it adds the interaction of the annual change in the outcome of interest during the pre-policy period (1990–2000) with year dummies, which allows for dissipation effects of these pre-trends. To assess quantitatively the role of other revenue sources as mediation channels of the effect of the FAIS, column (5) includes time varying measures of all other sources of municipal revenues, such as local taxes, unconditional federal transfers, and other conditional federal transfers.

The F-stats for the IV results indicate a particularly strong instrument, with values above 500, which results from the high compliance with the formula for our period of study. This explains why the IV results are not far from the RF effects shown in column (2).

The 2SLS estimates confirm that the FAIS had a positive effect on several nonmonetary measures of poverty. In our preferred specification with the full set of controls (column (4)), we find that a 10 percent increase in the FAIS is associated with a coverage increase of 0.10, 0.77, 0.57 and 0.48 percentage points in access to electricity, connection to sewerage, access to piped water, and a high-quality floor in the home. The FAIS did not have a significant effect on access to sanitation inside the dwelling. These results are expected for a fund that is earmarked for social infrastructure.



\subsection{Income and Poverty Outcomes}\label{subsec:income}

Table~\ref{tab:4} presents estimates of the impact of the FAIS on the logarithm of income per capita and the poverty headcount measured by different poverty lines. The OLS estimates show a small, positive, and statistically significant coefficient for the FAIS on income per capita (first row). The coefficient is slightly larger and statistically significant in the 2SLS specification if no pre-trend controls and fiscal revenues are added (See column (3)). The coefficient is reduced as controls are added (See columns (4) and (5)) and falls out of significance. We note that the coefficients in columns (4) and (5) are similar in magnitude to the coefficient in the OLS specification, but imprecisely estimated because of the increase in the standard errors. So, we cannot rule out a positive, but small effect on income per capita.

Overall, the FAIS did not have significant impacts on poverty headcounts measured using two lower poverty lines (as the three levels shown in columns (3)–(5)), with the possible exception ofexcept for the poverty headcount based on the highest poverty line (the APL) which is significant in the OLS model. The IV coefficient is larger than the OLS coefficient, butcoefficient but does not achieve statistical significance.

The fact that the effects of the FAIS on poverty seem larger if we use the APL, which is the highest poverty line, is consistent with the effects of the transfer concentrated in the top half of the income distribution, which would suggest an increase in inequality (see the next section).

\subsection{Inequality Outcomes}\label{subsec:inequiality}

Table~\ref{tab:5} shows the effects of the FAIS on within-municipality inequality. We use several metrics that measures inequality in different parts of the income distribution. The most global of them is the Gini coefficient, since it uses information from all the distribution. Other measures are distribution-wide income ratio (90/10 ratio), and income ratios that cover inequality in the bottom-half and the top half of the income distribution (50/10 ratio, 90/50 ratio) to characterize the changes in income inequality.

The FAIS had an unequalizing effect within municipalities. Table~\ref{tab:6} shows that a 10 percent increase in the FAIS had a larger positive effect of 0.57 percent on the top decile of the income distribution, but no effect on the rest of the deciles (using our preferred specification). Controlling for our full set of controls (column (4)), we find that the FAIS significantly increased inequality as measured by the Gini coefficient; the increase in the 90/10 ratio is positive, although imprecisely estimated. The observed increase in inequality seems to be explained by an increase in the top half of the income distribution, because the 90/50 ratio increases in a sizable manner. In particular, between 2005 and 2014, a 10 percent increase in the FAIS is associated with an increase in inequality of 0.21 Gini points. Finally, we find that the FAIS increased inequality in the top half of the income distribution because the coefficient of the 90/50 ratio shows that a 10 percent increase in the FAIS leads to a $0.012$0. increase in the 90/50 income ratio.

Overall, the IV coefficients (column (3)) are roughly comparable in magnitude to the OLS coefficients (column (1)), though the IV coefficient is larger for the Gini and the 90/50 ratio, while the OLS is larger for the rest. Adding as controls the lagged dependent variables (column (4)) and other fiscal resources (column (5)) has a small effect on the point estimates.

\subsection{Heterogeneous Impacts: Urban and Rural} \label{subsec:heterogeneous} 

The effects on social infrastructure outcomes tend to be positive across municipality types (see Table~\ref{tab:7}). In urban areas, the FAIS had positive and significant effects on all social infrastructure outcomes except for access to electricity and access to sanitation. Except for electricity, the coefficients are, in all cases, larger for all outcomes relative to the outcomes in rural areas.

Access to electricity is the only outcome that is larger and significant in rural areas in contrast to urban settings. The higher effect on access to electricity in rural areas likely arises because of coverage at the urban level was close to 100 percent already. So, there was little room for a FAIS impact in urban areas.

 The higher effect in connection to sewerage, access to piped water, and quality of the floor in urban areas is more in line with the more equalizing impact of the FAIS in urban areas and the unequalizing effect in rural areas, which we also observe in the monetary outcomes discussed below.

Table~\ref{tab:8} shows that the FAIS had a positive effect on average household income per capita in urban areas, but not in rural areas. However, the results are not statistically significant using our preferred specification with the full set of controls.

Although we did not find an effect on poverty at the aggregate level, the results show a marginally significant reduction in poverty in urban areas. In particular, a 10 percent increase in the FAIS was associated with a 0.169 percentage point reduction in food poverty between 2005 and 2014. In contrast with urban areas, the results in rural municipalities show coefficients of the opposite sign (and statistically insignificant). The relative effectiveness in urban areas may be related to the greater capacity to implement projects that benefit the poor or the better targeting of public investments in urban areas, which can be explained by the earlier finding that FAIS investments in rural areas are centered on the \textit{cabecera} municipal where the relatively more well-off live.

The unequalizing effect of the FAIS within municipalities is driven by rural areas. While the FAIS had a positive, significant, and greater effect on the Gini coefficient in rural areas, the effect on urban areas is close to zero and statistically insignificant (See Table~\ref{tab:9}). The same holds for the 90/50 ratio, although it is positive and marginally significant in urban areas as well. Moreover, the effect on the 50/10 ratio in urban areas is negative and significant, which suggests that the FAIS had an equalizing effect between the middle and the bottom of the income distribution in urban areas (a reduction of 0.09 points), whereas it had an unequalizing effect on these groups in rural areas (an increase of 0.08).

Both the higher effects on social infrastructure and the relatively more equalizing effect of the FAIS in urban areas is associated with the fact that the resources are usually invested in the periphery of cities, which tend to have a higher share of low-income residents. This is possible because the peripheries of urban municipalities are sufficiently large that economies of scale allow for such investments. In contrast, because of the need for economies of scale, FAIS resources in rural and semiurban areas are usually invested in the \textit{cabeceras}, where people with relatively higher incomes live, as documented by \cite{wellenstein2006social}.


\section{Robustness}\label{sec: Robustness}

\subsubsection{Pre-trends} Our two-way fixed effects design allows us to control for the fact that the policy allocates higher transfers to poorer municipalities. However, our results could be downward bias because areas that were more exposed to FAIS transfers may be on declining long-term trends in welfare prior to implementation of FAIS. We present the sensibility of our estimates to different strategies that aim to account for pre-trends in Appendix~\ref{ap:c} (Tables \ref{trends_inf} to \ref{trends_ineq}). Our baseline approach to account for pre-trends is presented in Column (2) of these tables while the rest of columns present alternative approaches. Column (3) controls for pre-trends using the lagged change in the outcome of interest. Column (4) adds the lagged value of the outcome of interest, which parametrically aim to compare only municipalities in the same level before the contemporaneous FAIS transfers. 

The results under the alternative approaches are stronger for the case of infrastructure, inequality, and poverty outcomes. In spite poverty turn out to be statistically significant under these alternative approaches, we prefer our baseline specification because the other measures of pre-trend controls use information from the post-policy period and therefore provide less exogenous variation.

\subsubsection{Late} Since our first stage shows a very strong IV with high compliance, our IV estimates are less local than one may expect from an identification strategy based on instrumental variables. However, it is interesting to know to what extent the IV estimates represent a population different from the OLS estimates or even from the observational data.

 \cite{aronow2016does} shows that regression model gives different weight to each observation depending on the conditional variance of the regressor of interest, i.e. observed FAIS. Therefore, municipalities with high within municipality variation in conditional FAIS will be given higher weight to estimate our treatment effect estimates. 

Appendix \ref{ap:D} presents the descriptive statistics using observational and regression-based weights for pre-policy variables measure in the 2000 population census. {Table}~\ref{fs} compares descriptive statistics with three weighting schemes: Nominal sample, effective sample according to the OLS, and effective sample according to the IV estimation. To obtain an effective sample of the IV estimation, we implemented a control function approach. 

Panel A and B of {figure}~\ref{fig:late} shows the test of differences in means between the nominal sample and each of the regression-based samples. We find small differences in magnitudes in the descriptive statistics across samples. The OLS-based weights tend to produce descriptive statistics more different from the nominal sample than their IV-based counterparts. This suggest the compliers of the IV regression tend to be relatively representative of the nominal sample. Putting these results together with our strong first stage we can suggest that our estimates are fairly representative of the population. 

\subsubsection{Multiple Hypothesis Testing} Because our specification estimate effects of FAIS over multiple outcomes variables, the probability that we incorrectly reject at least one null hypothesis is greater than the significance level used for each individual hypothesis test. When appropriate, we address this multiple inference concern in two ways:

 First, by controlling for the family-wise error rate (FWER) as in \cite{jones2019workplace}. We control for the FWER within families of outcomes instead of using one single family with all the outcomes pooled. We define three mutually exclusive families of hypotheses that encompass all of our outcome variables: i) social infrastructure, ii) income and poverty, and iii) inequality.\footnote{The infrastructure family is defined by access to electricity, connection to sewerage, access to water, quality of floor and access to sanitation; poverty family is formed by log of per capita income and poverty rates measured by three lines (food, capabilities and assets), and the inequality family is defined by Gini index and income ratios 90/10, 50/10 and 90/50.}  Table~\ref{ap:E.1} shows the main result of all our outcomes with family-wise p-value adjustment included. As it can be seen, all ofall the coefficients that were significant at the 5 level or higher remain significant at that level.  

The second way to account for multiple inference is to construct indices our groups of outcome variables. We standardized and centered all our outcome variables for all years for which we have information, and we implemented the procedure proposed by \cite{anderson2008multiple}.\footnote{Thanks to Bouguen and Varejkova (2020) for the implementation code in stata} We also transformed the variables to have similar directions. As it can be seen in {Table}~\ref{ap:E.2}, the results of our main specification using the indices for infrastructure, poverty and inequality are in the same direction, when a municipality experience a large increase in FAIS it also observed higher increase in the index that measured coverage in infrastructure and inequality. But, in spite the higher infrastructure there is not decrease in the poverty index. This lack of reduction in poverty could be explained by the uneven distribution of the monetary benefits from FAIS, which is evident in the increase in the inequality index. 

\section{Concluding Remarks} \label{sec: Concluding remarks}

Within the growing literature on fiscal federalism, a rigorous understanding of the welfare impacts of fiscal transfers to subnational governments is an important research issue. This paper contributes to this body of work by identifying the causal effects of a large earmarked fiscal transfer fund on local welfare in a developing country. It focuses on an overlooked social infrastructure fund in Mexico known as the FAIS as a case study to illustrate a larger problem that has implications at the global level, that is, how to raise the effectiveness of fiscal transfers to local governments to achieve higher-quality spending and, ultimately, improve social outcomes.

To identify the distributional effects of this large  earmarked fund on municipalities in Mexico and understand who may have benefited from this policy, we exploit a novel dataset combining data from a panel of poverty maps with information on the disbursement of intergovernmental transfers across municipalities in Mexico. Consistent with the main objective of this intergovernmental fiscal fund, we find a positive effect on several nonmonetary measures of well-being, such as access to services (electricity, sewerage, piped water) and the quality of the dwelling (floor materials). Our findings also show that this program has been associated with higher local income inequality because the relatively more well off in the poorest and rural municipalities benefited the most from these investments.

Adequate formulas and interventions to distribute and monitor intergovernmental transfers can create incentives to close coverage gaps, create appropriate incentives in the design of policies, boost the quality of social spending, and ultimately improve local economic activity. However, although designing optimal formulas to distribute fiscal resources to subnational governments is important, our study shows that this may not be enough, particularly if there are barriers to canalizing these investments to boost the social and productive inclusion of the poorest populations.




\end{document}	




%%%%%%%%%%%%%%%%%%%%%%%%%%%%%%%%%%%%%%%%%%%%%%%%%%%%%%%%%%%%%%%%%%%%%%
%%%%%%%%%%%%%%%%%%%%%%%%%%%%%%%%%%%%%%%%%%%%%%%%%%%%%%%%%%%%%%%%%%%%%%%
%%%%%%%%%%%%%%%%%%%%%%%%%%%%%%%%%%%%%%%%%%%%%%%%%%%%%%%%%%%%%%%%%%%%%%%
%\addcontentsline{toc}{section}{Online Appendix}
	
%\section{Referee Report Annexes}

%%%%%%%%%%%%%%%%%%%%%%%%%%%%%%%%%%%%%%%%%%%%%%%%%%%%%%%%%%%
%\subsection{Functional Form}
%%%%%%%%%%%%%%%%%%%%%%%%%%%%%%%%%%%%%%%%%%%%%%%%%%%%%%%%%%%


%\begin{table}[H]
%	\centering
%	\small
%	\caption{Functional Form: Impact of FAIS on Social Infrastructure}
%	\label{poly_inf}
%	\resizebox{11cm}{!}{
%		\begin{tabular}{lccc}

\toprule



\multicolumn{1}{l}{} & \multicolumn{3}{c}{\footnotesize{Observed FAIS Transfers (2SLS)}} \\ 


\cmidrule(lr{1mm}){2-2} 
\cmidrule(lr{1mm}){3-3} 
\cmidrule(lr{1mm}){4-4}  % "\\" was before, it was creating an extra additional line


\multicolumn{1}{l}{} &  \multicolumn{1}{c}{(1)} &
						\multicolumn{1}{c}{(2)} & 
						\multicolumn{1}{c}{(3)} \\
						

\midrule

%\multicolumn{7}{c}{\textit{Panel   A: Direct policy outcomes on infrastructure}} \\                                                          

% 1st row 
\textit{Access to electric lighting}   &  0.912***   &
						   0.481   &
						   0.392   \\

\vspace{4pt} &  \footnotesize{(0.350)}  &
			    \footnotesize{(0.372)}  &
			    \footnotesize{(0.382)}  \\

\vspace{4pt} &  \footnotesize{[830]} &
				\footnotesize{[677]} &
				\footnotesize{[665]} \\
				




% 2nd row 
\textit{Connection to sewerage}   &  6.160***   &
						   2.857***   &
						   2.200**   \\

\vspace{4pt} &  \footnotesize{(0.975)}  &
			    \footnotesize{(0.941)}  &
			    \footnotesize{(0.938)}  \\

\vspace{4pt} &  \footnotesize{[850]} &
				\footnotesize{[687]} &
				\footnotesize{[678]} \\
				


% 3rd row  			
\textit{Access piped water}   &  4.945***   &
						   1.728*   &
						   1.088   \\

\vspace{4pt} &  \footnotesize{(0.951)}  &
			    \footnotesize{(1.004)}  &
			    \footnotesize{(1.026)}  \\

\vspace{4pt} &  \footnotesize{[855]} &
				\footnotesize{[691]} &
				\footnotesize{[679]} \\
				

% 4th row  				
\textit{Housing (floor quality)}   &  4.175***   &
						   0.774   &
						   0.520   \\

\vspace{4pt} &  \footnotesize{(0.742)}  &
			    \footnotesize{(0.716)}  &
			    \footnotesize{(0.732)}  \\

\vspace{4pt} &  \footnotesize{[761]} &
				\footnotesize{[654]} &
				\footnotesize{[649]} \\



% 5th row  				

\textit{Access to sanitation}   &  -0.288   &
						   -1.176**   &
						   -1.606***   \\

\vspace{4pt} &  \footnotesize{(0.586)}  &
			    \footnotesize{(0.546)}  &
			    \footnotesize{(0.561)}  \\

\vspace{4pt} &  \footnotesize{[815]} &
				\footnotesize{[699]} &
				\footnotesize{[688]} \\

\midrule
{\bf Controls}    					&	   &   
										   & 
										   \\


\textit{Municipality and year FE}    &	$\checkmark$   &  
										$\checkmark$   &  
										$\checkmark$   \\

\textit{Dep. var pre-trends}  & $\checkmark$   &    
								$\checkmark$   &  
								$\checkmark$   \\
								
								
{\bf Functional form Formula inputs}       &	&   
												& 
												\\

\textit{1st order pol}  	& 	$\checkmark$ &  
											 &	 
											 \\
\textit{2nd order pol}  	& 				 &  
								$\checkmark$ &
											\\
\textit{3rd order pol}  	& 				 & 
											 &	 
								$\checkmark$ \\
\midrule		

Observations 			&	 6123   &  
							 6123   & 
							 6123   \\

Municipalities  		&    2119   &   
							 2119   & 
							 2119   \\
\bottomrule

\end{tabular}%

%	}		
%	\parbox{\textwidth}{\small 
%		\vspace{2eX}
%		\scriptsize	
%		\maintable
%		\polynomial 
%	}
%\end{table}
%\begin{table}[H]
%	\centering
%	\small
%	\caption{Functional Form: Impact of FAIS on Poverty and Mean Income}
%	\label{poly_pov}
%	\resizebox{11cm}{!}{
%		\begin{tabular}{lccc}

\toprule



\multicolumn{1}{l}{} & \multicolumn{3}{c}{\footnotesize{Observed FAIS Transfers (2SLS)}} \\ 


\cmidrule(lr{1mm}){2-2} 
\cmidrule(lr{1mm}){3-3} 
\cmidrule(lr{1mm}){4-4}  % "\\" was before, it was creating an extra additional line


\multicolumn{1}{l}{} &  \multicolumn{1}{c}{(1)} &
						\multicolumn{1}{c}{(2)} & 
						\multicolumn{1}{c}{(3)} \\
						

\midrule

%\multicolumn{7}{c}{\textit{Panel   A: Direct policy outcomes on infrastructure}} \\                                                          

% 1st row 
\textit{Mean household income (log)}   &  0.027   &
						   0.020   &
						   0.014   \\

\vspace{4pt} &  \footnotesize{(0.019)}  &
			    \footnotesize{(0.021)}  &
			    \footnotesize{(0.022)}  \\

\vspace{4pt} &  \footnotesize{[837]} &
				\footnotesize{[678]} &
				\footnotesize{[667]} \\
				




% 2nd row 
\textit{Food poverty rate}   &  -0.542   &
						   0.123   &
						   0.335   \\

\vspace{4pt} &  \footnotesize{(0.868)}  &
			    \footnotesize{(0.923)}  &
			    \footnotesize{(0.929)}  \\

\vspace{4pt} &  \footnotesize{[845]} &
				\footnotesize{[684]} &
				\footnotesize{[672]} \\
				


% 3rd row  			
\textit{Capabilities poverty rate}   &  -0.795   &
						   -0.061   &
						   0.175   \\

\vspace{4pt} &  \footnotesize{(0.869)}  &
			    \footnotesize{(0.927)}  &
			    \footnotesize{(0.934)}  \\

\vspace{4pt} &  \footnotesize{[844]} &
				\footnotesize{[683]} &
				\footnotesize{[671]} \\
				

% 4th row  				
\textit{Assets poverty rate}   &  -0.875   &
						   -0.568   &
						   -0.328   \\

\vspace{4pt} &  \footnotesize{(0.784)}  &
			    \footnotesize{(0.854)}  &
			    \footnotesize{(0.867)}  \\

\vspace{4pt} &  \footnotesize{[839]} &
				\footnotesize{[679]} &
				\footnotesize{[667]} \\



\midrule
{\bf Controls}    					&	   &   
										   & 
										   \\


\textit{Municipality and year FE}    &	$\checkmark$   &  
										$\checkmark$   &  
										$\checkmark$   \\

\textit{Dep. var pre-trends}  & $\checkmark$   &    
								$\checkmark$   &  
								$\checkmark$   \\
								
								
{\bf Functional form Formula inputs}       &	&   
												& 
												\\

\textit{1st order pol}  	& 	$\checkmark$ &  
											 &	 
											 \\
\textit{2nd order pol}  	& 				 &  
								$\checkmark$ &
											\\
\textit{3rd order pol}  	& 				 & 
											 &	 
								$\checkmark$ \\
\midrule		

Observations 			&	 6123   &  
							 6123   & 
							 6123   \\

Municipalities  		&    2119   &   
							 2119   & 
							 2119   \\
\bottomrule

\end{tabular}%

%	}		
%	\parbox{\textwidth}{\small 
%		\vspace{2eX}
%		\scriptsize	
%		\maintable
%		\polynomial 
%	}
%\end{table}

%\begin{table}[H]
%	\centering
%	\small
%	\caption{Functional Form: Impact of FAIS on Inequality}
%	\label{poly_ineq}
%	\resizebox{11cm}{!}{
%		\begin{tabular}{lccc}

\toprule



\multicolumn{1}{l}{} & \multicolumn{3}{c}{\footnotesize{Observed FAIS Transfers (2SLS)}} \\ 


\cmidrule(lr{1mm}){2-2} 
\cmidrule(lr{1mm}){3-3} 
\cmidrule(lr{1mm}){4-4}  % "\\" was before, it was creating an extra additional line


\multicolumn{1}{l}{} &  \multicolumn{1}{c}{(1)} &
						\multicolumn{1}{c}{(2)} & 
						\multicolumn{1}{c}{(3)} \\
						

\midrule

%\multicolumn{7}{c}{\textit{Panel   A: Direct policy outcomes on infrastructure}} \\                                                          

% 1st row 
\textit{Gini index}   &  1.805***   &
						   1.584***   &
						   1.618***   \\

\vspace{4pt} &  \footnotesize{(0.443)}  &
			    \footnotesize{(0.497)}  &
			    \footnotesize{(0.501)}  \\

\vspace{4pt} &  \footnotesize{[836]} &
				\footnotesize{[681]} &
				\footnotesize{[670]} \\
				




% 2nd row 
\textit{Income Ratio 90/10}   &  0.231   &
						   0.217   &
						   0.208   \\

\vspace{4pt} &  \footnotesize{(0.154)}  &
			    \footnotesize{(0.177)}  &
			    \footnotesize{(0.178)}  \\

\vspace{4pt} &  \footnotesize{[838]} &
				\footnotesize{[680]} &
				\footnotesize{[669]} \\
				


% 3rd row  			
\textit{Income Ratio 50/10}   &  0.028   &
						   0.046   &
						   0.050   \\

\vspace{4pt} &  \footnotesize{(0.029)}  &
			    \footnotesize{(0.033)}  &
			    \footnotesize{(0.033)}  \\

\vspace{4pt} &  \footnotesize{[832]} &
				\footnotesize{[676]} &
				\footnotesize{[666]} \\
				

% 4th row  				
\textit{Income Ratio 90/50}   &  0.108***   &
						   0.075**   &
						   0.072**   \\

\vspace{4pt} &  \footnotesize{(0.031)}  &
			    \footnotesize{(0.035)}  &
			    \footnotesize{(0.035)}  \\

\vspace{4pt} &  \footnotesize{[837]} &
				\footnotesize{[679]} &
				\footnotesize{[668]} \\



\midrule
{\bf Controls}    					&	   &   
										   & 
										   \\


\textit{Municipality and year FE}    &	$\checkmark$   &  
										$\checkmark$   &  
										$\checkmark$   \\

\textit{Dep. var pre-trends}  & $\checkmark$   &    
								$\checkmark$   &  
								$\checkmark$   \\
								
								
{\bf Functional form Formula inputs}       &	&   
												& 
												\\

\textit{1st order pol}  	& 	$\checkmark$ &  
											 &	 
											 \\
\textit{2nd order pol}  	& 				 &  
								$\checkmark$ &
											\\
\textit{3rd order pol}  	& 				 & 
											 &	 
								$\checkmark$ \\
\midrule		

Observations 			&	 6123   &  
							 6123   & 
							 6123   \\

Municipalities  		&    2119   &   
							 2119   & 
							 2119   \\
\bottomrule

\end{tabular}%

%	}		
%	\parbox{\textwidth}{\small 
%		\vspace{2eX}
%		\scriptsize	
%		\maintable
%		\polynomial 
%	}
%	
%\end{table}


%Other types of tables for notes : we decide to change because it was complicated to have wider notes %https://tex.stackexchange.com/questions/164589/table-width-with-threeparttable-smaller-than-notes-and-caption

%\begin{table}[H]
%	\resizebox{11cm}{!}{
%		\begin{threeparttable}
%			\begin{tabular}{lcccc}

\toprule



\multicolumn{1}{l}{} & \multicolumn{4}{c}{\footnotesize{Observed FAIS Transfers (2SLS)}} \\ 

\cmidrule(lr{1mm}){2-5}  % \\ was before, it was creating an extra additional line


\multicolumn{1}{l}{} &  \multicolumn{1}{c}{(1)} &
						\multicolumn{1}{c}{(2)} & 
						\multicolumn{1}{c}{(3)} & 
						\multicolumn{1}{c}{(4)}  \\ 

\midrule


% 1st row 
\textit{Gini index}   &  1.870***   &
						   1.805***   &
						   2.499***   &  
   						   2.395***   \\

\vspace{4pt} &  \footnotesize{(0.447)}  &
			    \footnotesize{(0.443)}  &
			    \footnotesize{(0.563)}  &
				\footnotesize{(0.561)}  \\

\vspace{4pt} &  \footnotesize{[844]} &
				\footnotesize{[836]} &
				\footnotesize{[584]} &
				\footnotesize{[584]} \\
				




% 2nd row 
\textit{Income Ratio 90/10}   &  0.229   &
						   0.231   &
						   0.308   &  
   						   0.262   \\

\vspace{4pt} &  \footnotesize{(0.154)}  &
			    \footnotesize{(0.154)}  &
			    \footnotesize{(0.190)}  &
				\footnotesize{(0.189)}  \\

\vspace{4pt} &  \footnotesize{[844]} &
				\footnotesize{[838]} &
				\footnotesize{[590]} &
				\footnotesize{[591]} \\
				


% 3rd row  			
\textit{Income Ratio 50/10}   &  0.028   &
						   0.028   &
						   0.041   &  
   						   0.035   \\

\vspace{4pt} &  \footnotesize{(0.029)}  &
			    \footnotesize{(0.029)}  &
			    \footnotesize{(0.036)}  &
				\footnotesize{(0.036)}  \\

\vspace{4pt} &  \footnotesize{[844]} &
				\footnotesize{[832]} &
				\footnotesize{[590]} &
				\footnotesize{[591]} \\
				

% 4th row  				
\textit{Income Ratio 90/50}   &  0.109***   &
						   0.108***   &
						   0.142***   &  
   						   0.134***   \\

\vspace{4pt} &  \footnotesize{(0.031)}  &
			    \footnotesize{(0.031)}  &
			    \footnotesize{(0.038)}  &
				\footnotesize{(0.038)}  \\

\vspace{4pt} &  \footnotesize{[844]} &
				\footnotesize{[837]} &
				\footnotesize{[587]} &
				\footnotesize{[584]} \\
				



\midrule
{\bf Controls}    					&	   &   
										   & 
										   & 
										   \\


\textit{Municipality and year FE}    &	$\checkmark$   &   
										$\checkmark$   & 
										$\checkmark$   & 
										$\checkmark$   \\

\textit{Formula inputs}  	& 	$\checkmark$    &   
								$\checkmark$    & 
								$\checkmark$    & 
								$\checkmark$    \\


{\bf Pre-trends Control}       &	   &   
									   & 
									   &
									   \\


\textit{$\Delta Y^{2000}_m$ $\times$ $1(year=t)$}  & 	
												   & $\checkmark$	
												   & 
												   & \\


\textit{$\Delta Y^{t-1}_m$} 						&	
													&   
													& $\checkmark$	
													& \\

\textit{$Y_{m,t-1}$}  								&
													& 
													& 
													& $\checkmark$	\\

\midrule		


Observations 			&	 6123   &   
							 6123   & 
							 6107   & 
							 6107   \\

Municipalities  		&    2119   &   
							 2119   & 
							 2118   & 
							 2118   \\

\bottomrule

\end{tabular}%

%			\begin{tablenotes}
%				\scriptsize	
%				\item \maintable
%				\item \trend 
%			\end{tablenotes}	
%		\end{threeparttable}
%	}
%\end{table}
%
%
%
%\begin{table}[ht] %ht was used before not sure why, just search


%\begin{table}[H]
%	\centering
%	\small
%	\caption{Estimates on Indexes:Urb}
%	\label{ds}
	
%	\resizebox{11cm}{!}{
		
		
%		\begin{tabular}{lccc}

\toprule



\multicolumn{1}{l}{} & \multicolumn{3}{c}{\footnotesize{Observed FAIS Transfers (2SLS)}} \\ 


\cmidrule(lr{1mm}){2-2} 
\cmidrule(lr{1mm}){3-3} 
\cmidrule(lr{1mm}){4-4}  % "\\" was before, it was creating an extra additional line


\multicolumn{1}{l}{} &  \multicolumn{1}{c}{All} &
						\multicolumn{1}{c}{Urban} & 
						\multicolumn{1}{c}{Rural} \\
\multicolumn{1}{l}{} &  \multicolumn{1}{c}{(1)} &
						\multicolumn{1}{c}{(2)} & 
						\multicolumn{1}{c}{(3)} \\
						

\midrule

%\multicolumn{7}{c}{\textit{Panel   A: Direct policy outcomes on infrastructure}} \\                                                          

% 1st row 
\textit{Social infrastructure index}   	&  0.077***   
							&  0.075***  
							&  0.056*   \\

\vspace{4pt} &  \footnotesize{(0.019)}   & 
			    \footnotesize{(0.025)}   & 
			    \footnotesize{(0.030)}    \\          


\vspace{4pt} &  \footnotesize{[841]}   & 
			    \footnotesize{[430]}   & 
			    \footnotesize{[339]}    \\          




% 2nd row 
\textit{Poverty and mean income index}   	&  0.021   
							&  -0.017  
							&  0.117   \\

\vspace{4pt} &  \footnotesize{(0.046)}   & 
			    \footnotesize{(0.054)}   & 
			    \footnotesize{(0.076)}   \\          


\vspace{4pt} &  \footnotesize{[845]}   & 
			    \footnotesize{[400]}   & 
			    \footnotesize{[344]}   \\          


% 3rd row  			
\textit{Inequality index}   	&  0.196***   
							&  -0.052  
							&  0.294***   \\

\vspace{4pt} &  \footnotesize{(0.073)}   & 
			    \footnotesize{(0.101)}   & 
			    \footnotesize{(0.111)}   \\          


\vspace{4pt} &  \footnotesize{[836]}   & 
			    \footnotesize{[408]}   & 
			    \footnotesize{[337]}   \\          


\midrule
{\bf Controls}    					&	   &   
										   & 
										   \\


\textit{Municipality and year FE}    &	$\checkmark$   & 
										$\checkmark$   & 
										$\checkmark$   \\

\textit{Dep. var pre-trends}  & $\checkmark$   &   
								$\checkmark$   & 
								$\checkmark$   \\
								
								
								
\midrule		

Observations 			&	 6123   &  
							 2949   & 
							 3063   \\

Municipalities  		&   2119    &   
							 1020   & 
							 1089    \\
\bottomrule

\end{tabular}%

		
%	}		
%	\parbox{\textwidth}{\small 
%		\vspace{2eX}
%		\scriptsize	
%		\maintable
%		\trend 
%	}
	
%\end{table}

%\begin{table}[H]
%	\centering
%	\small
%	\caption{Estimates on Social Infrastructure: Pre-trends}
%	\label{trends_inf}
%	\resizebox{11cm}{!}{
%		\begin{tabular}{lcccc}

\toprule



\multicolumn{1}{l}{} & \multicolumn{4}{c}{\footnotesize{Observed FAIS Transfers (2SLS)}} \\ 

\cmidrule(lr{1mm}){2-5}  % \\ was before, it was creating an extra additional line


\multicolumn{1}{l}{} &  \multicolumn{1}{c}{(1)} &
						\multicolumn{1}{c}{(2)} & 
						\multicolumn{1}{c}{(3)} & 
						\multicolumn{1}{c}{(4)}  \\ 

\midrule


% 1st row 
\textit{Social infrastructure index}   &  0.102***   &
						   0.077***   &
						   0.109***   &  
   						   0.098***   \\

\vspace{4pt} &  \footnotesize{(0.020)}  &
			    \footnotesize{(0.019)}  &
			    \footnotesize{(0.025)}  &
				\footnotesize{(0.022)}  \\

\vspace{4pt} &  \footnotesize{[844]} &
				\footnotesize{[841]} &
				\footnotesize{[595]} &
				\footnotesize{[594]} \\
				




% 2nd row 
\textit{Poverty and mean income index}   &  -0.001   &
						   0.021   &
						   0.009   &  
   						   -0.029   \\

\vspace{4pt} &  \footnotesize{(0.046)}  &
			    \footnotesize{(0.046)}  &
			    \footnotesize{(0.056)}  &
				\footnotesize{(0.057)}  \\

\vspace{4pt} &  \footnotesize{[844]} &
				\footnotesize{[845]} &
				\footnotesize{[593]} &
				\footnotesize{[588]} \\
				


% 3rd row  			
\textit{Inequality index}   &  0.193***   &
						   0.196***   &
						   0.244***   &  
   						   0.244***   \\

\vspace{4pt} &  \footnotesize{(0.072)}  &
			    \footnotesize{(0.073)}  &
			    \footnotesize{(0.090)}  &
				\footnotesize{(0.090)}  \\

\vspace{4pt} &  \footnotesize{[844]} &
				\footnotesize{[836]} &
				\footnotesize{[591]} &
				\footnotesize{[588]} \\
			
\midrule
{\bf Controls}    					&	   &   
										   & 
										   & 
										   \\


\textit{Municipality and year FE}    &	$\checkmark$   &   
										$\checkmark$   & 
										$\checkmark$   & 
										$\checkmark$   \\

\textit{Formula inputs}  	& 	$\checkmark$    &   
								$\checkmark$    & 
								$\checkmark$    & 
								$\checkmark$    \\


{\bf Pre-trends Control}       &	   &   
									   & 
									   &
									   \\


\textit{$\Delta Y^{2000}_m$ $\times$ $1(year=t)$}  & 	
												   & $\checkmark$	
												   & 
												   & \\


\textit{$\Delta Y^{t-1}_m$} 						&	
													&   
													& $\checkmark$	
													& \\

\textit{$Y_{m,t-1}$}  								&
													& 
													& 
													& $\checkmark$	\\

\midrule		


Observations 			&	 6123   &   
							 6123   & 
							 6107   & 
							 6107   \\

Municipalities  		&    2119   &   
							 2119   & 
							 2118   & 
							 2118   \\

\bottomrule

\end{tabular}%

%	}		
%	\parbox{\textwidth}{\small 
%		\vspace{2eX}
%		\scriptsize	
%		\maintable
%		\trend 
%	}
%	
%\end{table}
%	
%\begin{table}[H]
%		\centering
%		\small
%		\caption{Estimates on Indexes:Urb}
%		\label{ds}
%		
%		\resizebox{11cm}{!}{
%			
%			
%			\begin{tabular}{lccc}

\toprule



\multicolumn{1}{l}{} & \multicolumn{3}{c}{\footnotesize{Observed FAIS Transfers (2SLS)}} \\ 


\cmidrule(lr{1mm}){2-2} 
\cmidrule(lr{1mm}){3-3} 
\cmidrule(lr{1mm}){4-4}  % "\\" was before, it was creating an extra additional line


\multicolumn{1}{l}{} &  \multicolumn{1}{c}{(1)} &
						\multicolumn{1}{c}{(2)} & 
						\multicolumn{1}{c}{(3)} \\
						

\midrule

%\multicolumn{7}{c}{\textit{Panel   A: Direct policy outcomes on infrastructure}} \\                                                          

% 1st row 
\textit{Social infrastructure index}   &  0.077***   &
						   0.013   &
						   -0.007   \\

\vspace{4pt} &  \footnotesize{(0.019)}  &
			    \footnotesize{(0.019)}  &
			    \footnotesize{(0.019)}  \\

\vspace{4pt} &  \footnotesize{[841]} &
				\footnotesize{[685]} &
				\footnotesize{[672]} \\
				




% 2nd row 
\textit{Poverty and mean income index}   &  0.021   &
						   -0.040   &
						   -0.030   \\

\vspace{4pt} &  \footnotesize{(0.046)}  &
			    \footnotesize{(0.051)}  &
			    \footnotesize{(0.052)}  \\

\vspace{4pt} &  \footnotesize{[845]} &
				\footnotesize{[683]} &
				\footnotesize{[671]} \\
				


% 3rd row  			
\textit{Inequality index}   &  0.196***   &
						   0.195**   &
						   0.202**   \\

\vspace{4pt} &  \footnotesize{(0.073)}  &
			    \footnotesize{(0.082)}  &
			    \footnotesize{(0.083)}  \\

\vspace{4pt} &  \footnotesize{[836]} &
				\footnotesize{[679]} &
				\footnotesize{[668]} \\
				

\midrule
{\bf Controls}    					&	   &   
										   & 
										   \\


\textit{Municipality and year FE}    &	$\checkmark$   &  
										$\checkmark$   &  
										$\checkmark$   \\

\textit{Dep. var pre-trends}  & $\checkmark$   &    
								$\checkmark$   &  
								$\checkmark$   \\
								
								
{\bf Functional form Formula inputs}       &	&   
												& 
												\\

\textit{1st order pol}  	& 	$\checkmark$ &  
											 &	 
											 \\
\textit{2nd order pol}  	& 				 &  
								$\checkmark$ &
											\\
\textit{3rd order pol}  	& 				 & 
											 &	 
								$\checkmark$ \\
\midrule		

Observations 			&	 6123   &  
							 6123   & 
							 6123   \\

Municipalities  		&    2119   &   
							 2119   & 
							 2119   \\
\bottomrule

\end{tabular}%

%			
%		}		
%		\parbox{\textwidth}{\small 
%			\vspace{2eX}
%			\scriptsize	
%			\maintable
%			\trend 
%		}
%		
%	\end{table}







%%%subsection and section renames 
%Color for title
%\definecolor{ChadBlue}{rgb}{.1,.1,.5}  
%@startsection
%\let\LaTeX@startsection\@startsection 
%\renewcommand{\@startsection}[6]{\LaTeX@startsection%
%{#1}{#2}{#3}{#4}{#5}{\color{ChadBlue}\raggedright #6}} 

%% % Fix periods at end of section numbers
%\renewcommand \thesection {\@arabic\c@section.}
%\renewcommand\thesubsection   {\thesection\@arabic\c@subsection}%.}
%\renewcommand\thesubsubsection{\thesubsection %\@arabic\c@subsubsection}%.}

%\newcommand\section{%
%\@startsection{section}{1}{\z@}%
%{13pt \@plus 2pt \@minus 2pt}%
%{\belowsectionskip}%
%{\bfseries\sectionsize\center}}

%\newcommand\subsection{%
%\@startsection{subsection}{2}{\z@}%
%{13pt \@plus 2pt \@minus 2pt}%
%{13pt \@plus 2pt \@minus 2pt}%
%{\itshape\sectionsize\center}}

%\newcommand\subsubsection{%
%\def\headingsuffix{. ---\ }
%\@startsection{subsubsection}{3}{\subsubsectionindent}%
%{13pt \@plus 2pt \@minus 2pt}%
%{13pt \@plus 2pt \@minus 2pt}%
%{\scshape\sectionsize\center}}

%----------Section 
%\renewcommand\section{ %
% \@startsection{section}%  name
%    {1} % level
%    {0pt} % indent
%    {13pt \@plus 2pt \@minus 2pt} % beforeskip
%    {13pt \@plus 2pt \@minus 2pt} % afterskip
%    {\bf\raggedright\huge} % style
%}
%---------- Sample not modified anymore, keep the standards of AEJ
%\renewcommand\subsection{%
%  \@startsection{subsection}%  name
%    {2}% level
%    {0pt}% indent
%    {13pt \@plus 2pt \@minus 2pt}% beforeskip
%    {13pt \@plus 2pt \@minus 2pt}% afterskip
%    {\centering\Large\scshape}% style
%}

%This will put subsection titles in small caps type, inline with the paragraph.
%\renewcommand\subsection{%
%  \@startsection{subsection}%  name
%    {2}% level
%    {0em}% indent
%    {-1ex plus 0.1ex minus -0.05ex}% beforeskip
%    {-1em plus 0.2em}% afterskip
%    {\scshape}% style
%  }

