\documentclass[dv_diss_main.tex]{subfiles}

\begin{document}
\section{Conclusions}

This paper studies the impact of politically induced public spending on private-sector economic activity. To do so, I causally estimate the effect of political alignment on private employment in a context in which governors are able to disproportionally allocate an economically significant amount of public spending to municipalities that elect a mayor from the governor's own political party. 

I find that political alignment with state governors increases public spending by about 10 percentage points due to larger intergovernmental transfers received by aligned municipalities. Municipalities that experience the disproportional increase in spending suffer from a slowdown in private sector job creation. The results imply that the growth rate of private sector jobs in non-aligned municipalities is 10 percentage point lower in politically aligned municipalities. 
%Under the assumption that the main policy lever that the governors had during our study period is to increase the local public spending, our estimates imply an elasticity of private-formal employment to local spending of -0.89. 

I do not find evidence that higher corruption, a larger public sector enlargement, or an increase in the construction of infrastructure  projects explain the observed results. The lack of a similar negative effect on our proxy of total economic activity, measured by nighttime light and electricity consumption, suggests that the results can be interpret as a crowding out effect, where there is not effects on output but some substitution between public sector spending and private economic activity. 

The existing literature that studies the impact of political alignment on economic welfare has found positive effects for economies where politicians' policy levers are related to regulation rather than higher spending  \citep{asher2017politics}. In our context, the main policy lever is spending which, in line with the findings of \cite{cohen2011powerful}, seems to crowd out private sector economic activity.

These findings imply that politically induced public spending can have unintended consequences that may negatively affect welfare. Although the result does not imply lower welfare in the short term (at least when measure trough nigh lights and electricity consumption), it may affect welfare in the long term trough three channels. First, lower formal employment implies lower tax collection, which leads to a higher national deficit in the long run. Second, less workers in the formal sector implies lower pension savings. Third, higher informality can reduce the future incentives of firms to expand in order to remain informal.



\end{document}