\documentclass[dv_diss_main.tex]{subfiles}

\begin{document}
\subsection{Informality}
The previous analysis relies on social security records. Therefore, it remains mute about the impact of political alignment on employment in the informal sector and, consequently, on the total number of jobs.

A potential alternative explanation to the results is that public spending increases demand in sectors that disproportionally hire informal workers, like construction and services. In that case, the reduction of formal jobs is explained by a shift of workers from the formal to the informal sector rather than a slowdown in total employment. This explanation would be consistent with the evidence that the bulk of workers in the informal sector are informal by choice and not by a lack of opportunities for being employed in the formal sector \citep{alcaraz2015informality,maloney1999does}.

By definition, informal jobs are illegal; therefore, administrative records that would allow me to measure informal or total employment (informal and formal) at the municipal-year level do not exist. To circumvent this measurement problem, I use two alternative sources of data: household force surveys and economic censuses. 

The household survey data collects comprehensive individual-level information for a cross-section of individuals on a quarterly basis. This data allows me to measure the conditional probability of being employed and decompose it into the formal and informal sectors. 
\footnote{ I cannot compute municipal level aggregates with the household surveys because the data collects information for a non-representative subsample of households living in each municipality. The survey design makes the data representative at the metropolitan area level, which is a larger geographical unit than the municipality level }
In particular, I use this rich individual-level data to estimate the effect of political alignment on the probability of belonging to one of the following mutually exclusive labor statuses: formal worker, informal worker, unemployed, and out of the labor force. To do so, I estimate separate regression for each labor market status on an augmented version of equation (2) that includes a rich set of individual-level controls: age, education, gender, and household size. The data from the economic census reports the total number of jobs, among other establishment-level characteristics.\footnote{I only observe municipal-sector level aggregates of this establishment-level information.} I use the same specification of equation (2) where the outcome variable is the five-year change in total employment. This equation is estimated for all the close elections that took place in between any close election year. 

Panel A of Table~\ref{tab:censuslfs} reports the results for total employment. Columns 1 and 2 present the effect of political alignment using the household surveys. They show that political alignment reduces the probability of being employed by 3.3 to 3.4 percentage points. This result suggests that the effects observed using only the formal sector have consequences on total employment. Columns 3 and 4 present the estimated  effects of alignment on the change in total employment between 1998 and 2008. Although the estimates are not statistically significant, they are economically meaningful, suggesting that political alignment reduces the total employment growth rate by 5-7 percentage points.\footnote{The lower precision from the estimates that use data from the economic census may be explained by the fact that the outcome measures cannot match to measure employment before and after each mayor's term. In particular, two different measurement problems may lead a downward bias in the estimates: first, the outcome of interest is measure as a ten-year growth rate, while the treatment's period (mayor's term) is defined as a three-year window. Second, the difference between treatment year (election year) and the baseline year (1998 round of the economic census) varies across municipalities.}

Table~\ref{tab:lfs1} decomposes the result from household surveys into two exclusive components, each computed as a probability among the population between 15 and 65 years of age. Columns 1 and 2 show the effects of political alignment on the likelihood of working in the formal sector; in particular, Panel A shows that political alignment reduces the formal employment rate by 2.1-2.3 percentage points. In contrast, columns 3 and 4 show that the impact of political alignment on the informal employment rate is also negative, although imprecisely estimated. The sum of these two effects corresponds to the effect on the total employment, presented again for illustrative purposes in columns 5 and 6. Overall, the results suggest that alignment reduces the size of both the formal and the informal sector. Therefore, I can rule out that my main results are explained by shifts of workers from the formal to the informal sector. 

\end{document}
