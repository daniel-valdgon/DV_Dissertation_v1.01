\documentclass[dv_diss_main.tex]{subfiles}

\begin{document}

\section{Validity of the Research Design}\label{sec:ValResDes}

%presents first some descriptive statistics to characterize the size and characteristics of the sub-sample of razor-close elections that we use for identification. 
This section evaluates the internal validity of the identification strategy, which hinges on the fact that any other variables that affect my outcomes of interest change smoothly along the threshold. I perform two indirect tests for that purpose. I evaluate whether there are discontinuities along the cut-off on either the density of the running variable or baseline characteristics. I also provide a raw look at the spatial clustering of the data.

\subsection{Manipulation Test} The fact that political alignment brings benefits to local officials is a sufficient reason to think that local governments may self-select into political alignment with the governor. I test for this by evaluating whether the density of the vote margin changes abruptly at the cut-off. Figure~\ref{fig:hist} shows the histogram of the vote margin for the elections in which the governor's party compete. Overall, I do not find evidence of municipalities sorting on either side of the alignment threshold. This strong symmetry in the result of close elections is also apparent in the raw data. Particularly, I find that 467 out of 1867 elections were decided by a margin of less than five percentage points and, among those, 241 were won by the opposition and 226 by the governor's party. \footnote{This symmetry remains when I look at narrower bandwidths: I observe 282 (189) municipalities that were decided by a margin of less than three (one) percentage points, and among those, 149 (100) were won by the opposition and 133 (89) by the governor's party.} 

I implement the \cite{mccrary2008manipulation} test to formally evaluate if there are discontinuities in the vote margin at the threshold. Specifically, I estimate the density function of the vote margin separately on each side of the cut-off and test if the two expected values of the density function at the cut-off are statistically different from zero. The results, presented in Figure~\ref{fig:mccrary}, precisely suggest that there is no discontinuity of the density function around the cut-off. The p-value of the McCrary-test is 0.7, which indicates that neither the governor's party nor the opposition systematically wins close elections.\footnote{ \cite{calonico2020optimal} and \cite{bugni2021testing} have proposed variations of the \cite{mccrary2008manipulation} test. I obtain the same conclusion with any of these results (available upon request).}  

% It is possible that the Manipulation of the running variable depends on the party who is in power at the central level or in the incumbency status of either the state-governor or the municipal-mayor, for example, if manipulating an election requires solid political connections, it is reasonable to think that incumbent governors have a comparative advantage relative to new entrants. Figure XX shows the results of \cite{mccrary2008manipulation} test across different parties and across different status of incumbency advantage for both state-governor and local mayors. Overall, there is no strong evidence of particular parties manipulating elections or incumbency status being a key factor that allows observing sorting around the cut-off.  

\subsection{Discontinuity of Baseline Characteristics} Another indirect test to the identification assumption is to evaluate whether baseline characteristics jump discontinuously at the alignment threshold. A discontinuity on baseline characteristics would suggest that municipalities where the politically aligned candidate narrowly won are systematically different from municipalities where the politically aligned candidate narrowly lost.
To perform this test, I estimate the causal effect of political alignment using a variant of equation \eqref{eq:didrdd}.  The main difference is that now the baseline characteristics (measured in levels) are the main outcomes of interest, implying that they are excluded from the set of controls. Since I use the sample of close elections that took place during the study period (1998-2003),\footnote{The data set to perform this test is at the close election level, therefore municipalities with more than one close election show up more than once. I account for this by clustering standard errors at the municipality level; results are robust to only including the first close election for each municipality.} and the main outcomes are measured before 1998, I should not find discontinuities of these outcomes at the cut-off.

Figure~\ref{fig:baselinebalance} shows the results of this continuity test on several economic, socio-demographic, geographic, and political characteristics measured several years before every election took place.\footnote{All socioeconomic characteristics are measured from the 1990s Population and 1989 Economic Censuses. The political characteristics are measured from the previous electoral period.} 
I standardize all non-binary variables and present estimates in terms of standard deviation units to facilitate comparison across variables. The figure reports the point estimates and 95\% confidence intervals of each regression.
%\footnote{The Online Appendix provides analogous results for alternative bandwidths, kernels, and polynomials specifications.} 
It is reassuring to observe that there is no evidence of any discontinuous jump in the baseline characteristics. The confidence interval for each variable crosses zero, except for the share of workers in manufacturing. The p-value of the joint hypothesis test that all baseline characteristics are statistically equal to zero is 0.8. 

%While conducting several placebo test allows to be more confident about the no presence of omitted variable that jumps at the aligned-misaligned threshold it comes at the price of false discoveries. When testing multiple baseline variables, it is always possible that I find false positives even when our identification assumption holds \cite{de2016misunderstandings}.\footnote{When one test for discontinuities in 25 variables, and all null hypotheses are true at 5\% level of significance, one will have on average 1.25 false rejections \cite{de2016misunderstandings}.}
%To control for multiple testing, appendix~\ref{} shows Family-Wise-Error p-values (Westfall and Young, 1993 and Jones et al., 2019). Overall I confirm that baseline characteristics are continuous at the cut-off.\footnote{In addition to the continuity test, \cite{canay2018approximate} propose a permutation test to validate the continuity of the distribution, rather than the mean, of baseline characteristics at the cut-off. The results are presented in Appendix~\ref{} and fail to find evidence of selection into alignment among close elections.} 

%%%%%%%%%%

%\subsection{Size and distribution of close elections} The regression discontinuity design focus on close elections. This implies that in the presence of heterogeneous treatment effects, our estimates should be understood as a local average treatment effect. Therefore it is important to see what is the fraction of close elections in the sample and how similar are municipalities with high political competition to municipalities without it. 

%Table~\ref{} shows the difference in means between  municipalities with close elections and all the municipalities in Mexico on several characteristics. I observe that contested municipalities differ strongly from uncontested municipalities: they tend to have 2 percentage points more XXX, 3 percentage points more of XXXX, and 3 percentage points more of XXXX. 

%This suggests that I can not extrapolate our estimates to municipalities outside the cut-off. Another strong argument to abstain from such comparison is provided by a stylized model of political competition with two layers of government proposed by \cite{brollo}. The main argument of the model is that it is too costly for the central politician to swing voters in non-contested places. Therefore the optimal response of central government with uncontested places does not vary with partisan alignment. See \cite{brollo2012tying} for formal proof of this argument. 

\subsection{Spatial Concentration of Close Elections} Another concern is that the close elections subsample is geographically clustered. This would bias the estimates in the presence of  either spatial spillover effects or heterogeneous effects by state characteristics. Figure~\ref{fig:map} shows the map of close elections during the study period using a bandwidth of 5 percentage points. If a municipality has more than one close election, I map the result of the first election. As it can be observed, close elections are spread throughout the country. All 32 states had at least one close election won by either the governor's party or the main opposition. 

%The test on spatial clustering shows that close elections are not particularly clustered 


%\subsubsection{Are close elections persistent} Other importance concern is that our sample of close elections affect always same municipalities.  close elections may the results in our estimates may be persistent over time. This goes in line with the incumbency advantage. our treatment may have persistent effects over time. Panel A shows the evolution of close elections over time Morevoer, the panel B of Figure XX shows the map 

\end{document}